% Allow latex commands
\usepackage{ltxcmds}
\usepackage{listings}

% Allow us to define page borders
% \ifsetCustomMargin\
  % \RequirePackage[left=37mm,right=30mm,top=35mm,bottom=30mm]{geometry}
  % \setFancyHdr\  % To apply fancy header after geometry package is loaded
% \fi
% Package for displaying algorithms in a document as pseudo code
% \usepackage{algorithmicx} % Replaced the algorithm package
\usepackage{algorithm}
\usepackage{algpseudocode}

% Package for having multiple figures in one
\usepackage{subfig}  % replaced the subfigure package
\usepackage{float}

% Have multiple rows per column in a table
\usepackage{multirow}
\usepackage{xparse}

% Special bold and uppercase math characters
\usepackage{mathtools} % Replaced the amsmath package
\usepackage{amssymb}
\usepackage{mathptmx} % Replaced the times package
% Keep the old calligraphic math font
\DeclareMathAlphabet{\mathcal}{OMS}{cmsy}{m}{n}

% *****************************************************************************
% ******************* Fonts (like different typewriter fonts etc.)*************

% Add `customfont' in the document class option to use this section

\ifsetCustomFont
  % Set your custom font here and use `customfont' in options. Leave empty to
  % load computer modern font (default LaTeX font).
  %\RequirePackage{helvet}
  % \RequirePackage{newtxtext,newtxmath}
  \RequirePackage{lmodern}
  % \RequirePackage{tgtermes,newtxmath}

  % For use with XeLaTeX
  %  \setmainfont[
  %    Path              = ./libertine/opentype/,
  %    Extension         = .otf,
  %    UprightFont = LinLibertine_R,
  %    BoldFont = LinLibertine_RZ, % Linux Libertine O Regular Semibold
  %    ItalicFont = LinLibertine_RI,
  %    BoldItalicFont = LinLibertine_RZI, % Linux Libertine O Regular Semibold Italic
  %  ]
  %  {libertine}
  %  % load font from system font
  %  \newfontfamily\libertinesystemfont{Linux Libertine O}
\fi
% Package for having nice bold and uppercase characters
% in formulae
% Already loaded by the PHD thesis class
%\usepackage{mathptmx} % Replaced the times package

% Add colour to your text and images
% Already loaded by the PHD thesis class
%\usepackage{color} 

% Allow us to easily customize page setup
% The fancyhdr package was already defined in the class
%\usepackage{fancyhdr}
%\pagestyle{fancyplain}
%\rfoot{\thepage}
% This clears old style settings
%\fancyhead{}
%\fancyfoot{}
%\sloppy

% Other misc packages
%\usepackage{ifpdf}
\usepackage{afterpage}
% \usepackage[labelfont=bf,textfont=it]{caption}
\usepackage{pdflscape}
\usepackage{multicol}
%\usepackage{babel}  % For multilingual support

%\ifpdf
%   \usepackage[pdftex]{graphicx}
%\else
%   \usepackage{graphicx}
%\fi

% Set equation numbers <chapter>.<section>.<index>
\numberwithin{equation}{section} 
\usepackage{bm,bbm}
\usepackage{pdflscape}
\usepackage{multicol}
\usepackage{booktabs} % for top and bottom rules in tables
\usepackage{tabularx} % for variable width columns in tables
\usepackage[table]{xcolor}
%%%%%%%% Some Table Options %%%%%%%%%%%%%%
% define "struts", as suggested by Claudio Beccari in
% %    a piece in TeX and TUG News, Vol. 2, 1993.
\newcommand\Tstrut{\rule{0pt}{2.6ex}}         % = `top' strut
\newcommand\Bstrut{\rule[-0.9ex]{0pt}{0pt}}   % = `bottom' strut
%%%%%%%%%%%%%%%%%%%%%%%%%%%%%%%%%%%%%%%%%%%


\usepackage[export]{adjustbox}[2011/08/13]
% Set equation numbers <chapter>.<section>.<index>
\numberwithin{equation}{section} 
% Include tikz
\usepackage{tikz,pgfplots}
\usetikzlibrary{matrix,positioning,arrows}
\usetikzlibrary{decorations.markings}
\usetikzlibrary{dsp,chains,fit}
%\usetikzlibrary{external}\tikzexternalize         % Allow tikz images to be
                                                  % compiled once
% Make examples in your document
\usepackage{amsthm}
%\usepackage{shadethm}
%\theoremstyle{definition}
%\newshadetheorem{exmp}{Example}[section]
%\definecolor{shadethmcolor}{HTML}{EDF8FF}
\usepackage{mdframed}
\usepackage[shortlabels,inline]{enumitem}
\newmdtheoremenv[
  hidealllines=true,
  innerleftmargin=8pt,%
  innerrightmargin=8pt,%
  innertopmargin=12pt,%
  innerbottommargin=12pt,%
  backgroundcolor=blue!10,%
  skipbelow=\baselineskip,%
  skipabove=\baselineskip]{exmp}{Example}[section]
\newmdenv[linecolor=blue!10, backgroundcolor=blue!10,skipbelow=\baselineskip,
          skipabove=\baselineskip]{goals}
% Allow hyperlinks in our document. Redefine the section names to use the
% squigly S
\usepackage{nameref}
\usepackage{hyperref}
%%%%%%%%% Some Hyperref Options %%%%%%%%%%
\hypersetup{%
  colorlinks   = true,
  citecolor    = blue  
}
% \renewcommand{\sectionautorefname}{\S}
% \renewcommand{\subsectionautorefname}{\S}
% \renewcommand{\subsubsectionautorefname}{\S}
% \renewcommand{\pageautorefname}{p.}
% \renewcommand{\chapterautorefname}{Chapter}
% \newcommand{\subfigureautorefname}{\figureautorefname}

\usepackage{array}
\newcolumntype{b}{X}
\newcolumntype{s}{>{\hsize=.5\hsize}X}
\newcommand{\heading}[1]{\multicolumn{1}{c}{#1}}
%\newcolumntype{L}[1]{>{\raggedright\let\newline\\\arraybackslash\hspace{0pt}}m{#1}}
% \newcolumntype{C}[1]{>{\centering\let\newline\\\arraybackslash\hspace{0pt}}m{#1}}
% \newcolumntype{R}[1]{>{\raggedleft\let\newline\\\arraybackslash\hspace{0pt}}m{#1}}
\robustify{\subref}
\usepackage{breqn}
\newif\ifcuboidshade
\newif\ifcuboidemphedge
\newif\ifcuboiddrawxdims
\newif\ifcuboiddrawydims
\newif\ifcuboiddrawzdims

\tikzset{
  cuboid/.is family,
  cuboid,
  shiftx/.initial=0,
  shifty/.initial=0,
  dimx/.initial=3,
  dimy/.initial=3,
  dimz/.initial=3,
  scale/.initial=1,
  densityx/.initial=1,
  densityy/.initial=1,
  densityz/.initial=1,
  rotation/.initial=0,
  anglex/.initial=0,
  angley/.initial=90,
  anglez/.initial=225,
  scalex/.initial=1,
  scaley/.initial=1,
  scalez/.initial=0.5,
  front/.style={draw=black,fill=white},
  top/.style={draw=black,fill=white},
  right/.style={draw=black,fill=white},
  shade/.is if=cuboidshade,
  shadecolordark/.initial=black,
  shadecolorlight/.initial=white,
  shadeopacity/.initial=0.15,
  shadesamples/.initial=16,
  emphedge/.is if=cuboidemphedge,
  emphstyle/.style={thick},
  drawzdims/.is if=cuboiddrawzdims,
  dimzval/.initial=C,
  drawxdims/.is if=cuboiddrawxdims,
  dimxval/.initial=W,
  drawydims/.is if=cuboiddrawydims,
  dimyval/.initial=H,
}

\newcommand{\tikzcuboidkey}[1]{\pgfkeysvalueof{/tikz/cuboid/#1}}

% Commands
\newcommand{\tikzcuboid}[1]{
    \tikzset{cuboid,#1} % Process Keys passed to command
  \pgfmathsetlengthmacro{\vectorxx}{\tikzcuboidkey{scalex}*cos(\tikzcuboidkey{anglex})*28.452756}
  \pgfmathsetlengthmacro{\vectorxy}{\tikzcuboidkey{scalex}*sin(\tikzcuboidkey{anglex})*28.452756}
  \pgfmathsetlengthmacro{\vectoryx}{\tikzcuboidkey{scaley}*cos(\tikzcuboidkey{angley})*28.452756}
  \pgfmathsetlengthmacro{\vectoryy}{\tikzcuboidkey{scaley}*sin(\tikzcuboidkey{angley})*28.452756}
  \pgfmathsetlengthmacro{\vectorzx}{\tikzcuboidkey{scalez}*cos(\tikzcuboidkey{anglez})*28.452756}
  \pgfmathsetlengthmacro{\vectorzy}{\tikzcuboidkey{scalez}*sin(\tikzcuboidkey{anglez})*28.452756}
  \begin{scope}[
    xshift=\tikzcuboidkey{shiftx}, 
    yshift=\tikzcuboidkey{shifty}, 
    scale=\tikzcuboidkey{scale}, 
    rotate=\tikzcuboidkey{rotation}, 
    x={(\vectorxx,\vectorxy)}, 
    y={(\vectoryx,\vectoryy)}, 
    z={(\vectorzx,\vectorzy)}]

    \pgfmathsetmacro{\steppingx}{1/\tikzcuboidkey{densityx}}
    \pgfmathsetmacro{\steppingy}{1/\tikzcuboidkey{densityy}}
    \pgfmathsetmacro{\steppingz}{1/\tikzcuboidkey{densityz}}
    \newcommand{\dimx}{\tikzcuboidkey{dimx}}
    \newcommand{\dimy}{\tikzcuboidkey{dimy}}
    \newcommand{\dimz}{\tikzcuboidkey{dimz}}
    \pgfmathsetmacro{\secondx}{2*\steppingx}
    \pgfmathsetmacro{\secondy}{2*\steppingy}
    \pgfmathsetmacro{\secondz}{2*\steppingz}
    \foreach \x in {\steppingx,\secondx,...,\dimx} { 
      \foreach \y in {\steppingy,\secondy,...,\dimy} {   
        \pgfmathsetmacro{\lowx}{(\x-\steppingx)}
        \pgfmathsetmacro{\lowy}{(\y-\steppingy)}
        \filldraw[cuboid/front] (\lowx,\lowy,0.5*\dimz) -- (\lowx,\y,0.5*\dimz) -- (\x,\y,0.5*\dimz) -- (\x,\lowy,0.5*\dimz) -- cycle;
      }
    }
    \foreach \x in {\steppingx,\secondx,...,\dimx} { 
      \foreach \z in {\steppingz,\secondz,...,\dimz} {   
        \pgfmathsetmacro{\lowx}{(\x-\steppingx)}
        \pgfmathsetmacro{\lowz}{(\z-\steppingz-0.5*\dimz)}
        \pgfmathsetmacro{\highz}{(\z-0.5*\dimz)}
        \filldraw[cuboid/top] (\lowx,\dimy,\lowz) -- (\lowx,\dimy,\highz) -- (\x,\dimy,\highz) -- (\x,\dimy,\lowz) -- cycle;
      }
    }
    \foreach \y in {\steppingy,\secondy,...,\dimy} { 
      \foreach \z in {\steppingz,\secondz,...,\dimz} {
        \pgfmathsetmacro{\lowy}{(\y-\steppingy)}
        \pgfmathsetmacro{\lowz}{(\z-\steppingz-0.5*\dimz)}
        \pgfmathsetmacro{\highz}{(\z-0.5*\dimz)}
        \filldraw[cuboid/right] (\dimx,\lowy,\lowz) -- (\dimx,\lowy,\highz) -- (\dimx,\y,\highz) -- (\dimx,\y,\lowz) -- cycle;
      }
    }
    \ifcuboidemphedge
      \draw[cuboid/emphstyle] (0,\dimy,-0.5*\dimz) -- (\dimx,\dimy,-0.5*\dimz) -- (\dimx,\dimy,0.5*\dimz) -- (0,\dimy,0.5*\dimz) -- cycle;%
      \draw[cuboid/emphstyle] (0,\dimy,0.5*\dimz) -- (0,0,0.5*\dimz) -- (\dimx,0,0.5*\dimz) -- (\dimx,\dimy,0.5*\dimz);%
      \draw[cuboid/emphstyle] (\dimx,\dimy,-0.5*\dimz) -- (\dimx,0,-0.5*\dimz) -- (\dimx,0,0.5*\dimz);%
    \fi
    \ifcuboiddrawxdims
      \draw[<->] (0, -0.5, 0.5*\dimz) -- (\dimx, -0.5, 0.5*\dimz) node[below,midway] {$\tikzcuboidkey{dimxval}$};
    \fi
    \ifcuboiddrawydims
      \draw[<->] (-0.5, 0, 0.5*\dimz) -- (-0.5, \dimy, 0.5*\dimz) node[left,midway] {$\tikzcuboidkey{dimyval}$};
    \fi
    \ifcuboiddrawzdims
      \draw[<->] (\dimx, -0.5, -0.5*\dimz) -- (\dimx, -0.5, 0.5*\dimz) node[below right,midway] {$\tikzcuboidkey{dimzval}$};
    \fi

    \ifcuboidshade
      \pgfmathsetmacro{\cstepx}{\dimx/\tikzcuboidkey{shadesamples}}
      \pgfmathsetmacro{\cstepy}{\dimy/\tikzcuboidkey{shadesamples}}
      \pgfmathsetmacro{\cstepz}{\dimz/\tikzcuboidkey{shadesamples}}
      \foreach \s in {1,...,\tikzcuboidkey{shadesamples}} {   
        \pgfmathsetmacro{\lows}{\s-1}
        \pgfmathsetmacro{\cpercent}{(\lows)/(\tikzcuboidkey{shadesamples}-1)*100}
        \fill[opacity=\tikzcuboidkey{shadeopacity},
              color=\tikzcuboidkey{shadecolorlight}!\cpercent!\tikzcuboidkey{shadecolordark}] 
            (0,\s*\cstepy,0.5*\dimz) -- (\s*\cstepx,\s*\cstepy,0.5*\dimz) -- (\s*\cstepx,0,0.5*\dimz) 
              -- (\lows*\cstepx,0,0.5*\dimz) -- (\lows*\cstepx,\lows*\cstepy,0.5*\dimz) -- (0,\lows*\cstepy,0.5*\dimz) -- cycle;
        \fill[opacity=\tikzcuboidkey{shadeopacity},
              color=\tikzcuboidkey{shadecolorlight}!\cpercent!\tikzcuboidkey{shadecolordark}] 
            (0,\dimy,\s*\cstepz-0.5*\dimz) -- (\s*\cstepx,\dimy,\s*\cstepz-0.5*\dimz) -- (\s*\cstepx,\dimy,-0.5*\dimz) 
              -- (\lows*\cstepx,\dimy,-0.5*\dimz) -- (\lows*\cstepx,\dimy,\lows*\cstepz-0.5*\dimz) -- (0,\dimy,\lows*\cstepz-0.5*\dimz) -- cycle;
        \fill[opacity=\tikzcuboidkey{shadeopacity},
              color=\tikzcuboidkey{shadecolorlight}!\cpercent!\tikzcuboidkey{shadecolordark}] 
            (\dimx,0,\s*\cstepz-0.5*\dimz) -- (\dimx,\s*\cstepy,\s*\cstepz-0.5*\dimz) -- (\dimx,\s*\cstepy,-0.5*\dimz) 
              -- (\dimx,\lows*\cstepy,-0.5*\dimz) -- (\dimx,\lows*\cstepy,\lows*\cstepz-0.5*\dimz) -- (\dimx,0,\lows*\cstepz-0.5*\dimz) -- cycle;
      }
    \fi 

  \end{scope}
}

