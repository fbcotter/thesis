% Allow latex commands
\usepackage{ltxcmds}
\usepackage{listings}

% Allow us to define page borders
% \ifsetCustomMargin\
  % \RequirePackage[left=37mm,right=30mm,top=35mm,bottom=30mm]{geometry}
  % \setFancyHdr\  % To apply fancy header after geometry package is loaded
% \fi
% Package for displaying algorithms in a document as pseudo code
\usepackage{algorithm2e} % Replaced the algorithm package

% Package for having multiple figures in one
\usepackage{subfig}  % replaced the subfigure package

% Have multiple rows per column in a table
\usepackage{multirow}

% Special bold and uppercase math characters
\usepackage{mathtools} % Replaced the amsmath package
\usepackage{amssymb}
\usepackage{mathptmx} % Replaced the times package
% Keep the old calligraphic math font
\DeclareMathAlphabet{\mathcal}{OMS}{cmsy}{m}{n}

% *****************************************************************************
% ******************* Fonts (like different typewriter fonts etc.)*************

% Add `customfont' in the document class option to use this section

\ifsetCustomFont
  % Set your custom font here and use `customfont' in options. Leave empty to
  % load computer modern font (default LaTeX font).
  %\RequirePackage{helvet}
  % \RequirePackage{newtxtext,newtxmath}
  \RequirePackage{lmodern}
  % \RequirePackage{tgtermes,newtxmath}

  % For use with XeLaTeX
  %  \setmainfont[
  %    Path              = ./libertine/opentype/,
  %    Extension         = .otf,
  %    UprightFont = LinLibertine_R,
  %    BoldFont = LinLibertine_RZ, % Linux Libertine O Regular Semibold
  %    ItalicFont = LinLibertine_RI,
  %    BoldItalicFont = LinLibertine_RZI, % Linux Libertine O Regular Semibold Italic
  %  ]
  %  {libertine}
  %  % load font from system font
  %  \newfontfamily\libertinesystemfont{Linux Libertine O}
\fi
% Package for having nice bold and uppercase characters
% in formulae
% Already loaded by the PHD thesis class
%\usepackage{mathptmx} % Replaced the times package

% Add colour to your text and images
% Already loaded by the PHD thesis class
%\usepackage{color} 

% Allow us to easily customize page setup
% The fancyhdr package was already defined in the class
%\usepackage{fancyhdr}
%\pagestyle{fancyplain}
%\rfoot{\thepage}
% This clears old style settings
%\fancyhead{}
%\fancyfoot{}
%\sloppy

% Other misc packages
%\usepackage{ifpdf}
\usepackage{afterpage}
% \usepackage[labelfont=bf,textfont=it]{caption}
\usepackage{pdflscape}
\usepackage{multicol}
%\usepackage{babel}  % For multilingual support

%\ifpdf
%   \usepackage[pdftex]{graphicx}
%\else
%   \usepackage{graphicx}
%\fi

% Set equation numbers <chapter>.<section>.<index>
\numberwithin{equation}{section} 
\usepackage{bm,bbm}
\usepackage{pdflscape}
\usepackage{multicol}
\usepackage{booktabs} % for top and bottom rules in tables
\usepackage{tabularx} % for variable width columns in tables
\usepackage[table]{xcolor}
%%%%%%%% Some Table Options %%%%%%%%%%%%%%
% define "struts", as suggested by Claudio Beccari in
% %    a piece in TeX and TUG News, Vol. 2, 1993.
\newcommand\Tstrut{\rule{0pt}{2.6ex}}         % = `top' strut
\newcommand\Bstrut{\rule[-0.9ex]{0pt}{0pt}}   % = `bottom' strut
%%%%%%%%%%%%%%%%%%%%%%%%%%%%%%%%%%%%%%%%%%%


\usepackage[export]{adjustbox}[2011/08/13]
% Set equation numbers <chapter>.<section>.<index>
\numberwithin{equation}{section} 
% Include tikz
\usepackage{tikz,pgfplots}
\usetikzlibrary{matrix,positioning,arrows}
\usetikzlibrary{decorations.markings}
\usetikzlibrary{dsp,chains}
%\usetikzlibrary{external}\tikzexternalize         % Allow tikz images to be
                                                  % compiled once
% Make examples in your document
\usepackage{amsthm}
%\usepackage{shadethm}
%\theoremstyle{definition}
%\newshadetheorem{exmp}{Example}[section]
%\definecolor{shadethmcolor}{HTML}{EDF8FF}
\usepackage{mdframed}
\usepackage[inline]{enumitem}
\newmdtheoremenv[
  hidealllines=true,
  innerleftmargin=8pt,%
  innerrightmargin=8pt,%
  innertopmargin=12pt,%
  innerbottommargin=12pt,%
  backgroundcolor=blue!10,%
  skipbelow=\baselineskip,%
  skipabove=\baselineskip]{exmp}{Example}[section]
\newmdenv[linecolor=blue!10, backgroundcolor=blue!10,skipbelow=\baselineskip,
          skipabove=\baselineskip]{goals}
% Allow hyperlinks in our document. Redefine the section names to use the
% squigly S
\usepackage{nameref}
% \usepackage{hyperref}
%%%%%%%%% Some Hyperref Options %%%%%%%%%%
% \hypersetup{%
  % colorlinks   = true,
  % citecolor    = blue  
% }
% \renewcommand{\sectionautorefname}{\S}
% \renewcommand{\subsectionautorefname}{\S}
% \renewcommand{\subsubsectionautorefname}{\S}
% \renewcommand{\pageautorefname}{p.}
% \renewcommand{\chapterautorefname}{Chapter}
% \newcommand{\subfigureautorefname}{\figureautorefname}

\usepackage{array}
\newcolumntype{b}{X}
\newcolumntype{s}{>{\hsize=.5\hsize}X}
\newcommand{\heading}[1]{\multicolumn{1}{c}{#1}}
%\newcolumntype{L}[1]{>{\raggedright\let\newline\\\arraybackslash\hspace{0pt}}m{#1}}
% \newcolumntype{C}[1]{>{\centering\let\newline\\\arraybackslash\hspace{0pt}}m{#1}}
% \newcolumntype{R}[1]{>{\raggedleft\let\newline\\\arraybackslash\hspace{0pt}}m{#1}}
\makeatletter
% \def\fullpath{\begingroup\everyeof{\noexpand}\@sanitize
  % \edef\x{\@@input|"find `pwd` -name \jobname.tex" }%
  % \edef\x{\endgroup\noexpand\zap@space\x\noexpand\@empty}\x}
% \makeatother
