\chapter{Conclusion}\label{ch:conclusion}
  This chapter aims to logically tie together the results from the previous
  chapter, outlining what has been promising and what has not been, offering
  explanations as to why we think that is the case.

\section{Discussion of Attempts so Far}
\subsection{Multiscale Scatternets + SVM}
  Much of the beginning of the project was spent understanding and analysing the
  Scatternet design proposed by Mallat and the proposed \DTCWT\ based
  multiscale Scatternet. It was found that the \DTCWT\ based Scatternets are
  much faster than the Mallat based Scatternets (3--4 times faster for tiny
  images, and 10--15 times faster for medium resolution images).
  Work by a colleague --- \citet{singh_multi-resolution_2016} showed that this
  came at the cost of no appreciable loss of performance.

  Attempts to use the four layer multiscale Scatternet with linear SVMs on
  \cifar\ data showed that little performance was gained from lots of extra
  coefficients from the fourth layer. For example, the H4 block could easily be
  removed with no appreciable difference in performance, but saving one third
  of the coefficients. Nonetheless, it was difficult to improve this accuracy
  past $70\%$ without making use of further feature extraction methods
  (orthogonal least squares was used in both \citep{oyallon_deep_2015} and
  \citep{singh_multi-resolution_2016}).

  This shows the multiscale Scatternet alone is not achieving
  the same quality of feature extraction  as
  a few convolutional layers in a CNN, as simply connecting a classifier on the
  output achieved a lower test accuracy.

  We believe that this was because the initial Scatternet design lacks the
  fully connectedness in the depth dimension that comes
  with the CNN\@. This would allow a network to learn how to add linear
  combinations of different output activations from the layer below. At the
  simplest case, at layer two, this would allow the CNN to have a filter that
  could combine two oriented wavelets, perhaps at a slight offset from each
  other, allowing it to detect a corner. 

\subsection{Reduced Multiscale Scatternets and CNNs}
  Focus switched in the latter half of the year to bringing back in one layer
  of CNNs with a scaled down multiscale Scatternet, followed by a neural
  network classifier.  Our initial results were quite poor compared to the pure
  CNN solution --- a $10\%$ reduction in accuracy from $87\%$ to $77\%$. This was
  still an improvement on the $\sim 69\%$ we were getting with the pure SVM
  method which, while comforting, does little to meet our project goals.
  However, subsequent refinement of our methods narrowed this gap
  significantly and started to show some promising results.

  As a side note, poor initial results do not necessarily indicate failings of
  the Scatternet. There are a huge variety of hyperparameters to choose when
  designing the convolutional side of our hybrid network, and on top of that,
  a selection of different extra layers/schemes that can be used to `normalize'
  the statistics of the data, all to promote learning. Replacing one layer of
  an optimized CNN with a Scatternet and getting a reduction in accuracy could
  be due to the later layers of the CNN now no longer being optimized. In fact,
  it was very surprising we had such a large reduction in accuracy. The ability
  for us to eventually narrow the gap only reinforces the necessity of the
  project, to better understand how we should train CNNs.

\subsubsection{Quicker Training Time}
  Our Scatternet and shallow CNN was able to train much faster (in fewer
  epochs) than a pure CNN method. We regard this as an early sign that we are
  on the right track. Unfortunately, our network does plateau earlier than
  a CNN, so we end up performing worse in the long run --- see
  \autoref{fig:comparison}.

  \begin{figure}
    \centering
      % \includegraphics[width=0.8\textwidth]{results/comparison.png}
      \caption[Comparison of test accuracy for our network vs a two layer CNN]
              {Comparison of test accuracy for our network vs a two layer
              CNN\@.
              The network used for the Scatternet+CNN design was the reduced
              Scatternet with data augmentation (see \autoref{sec:best_scat}). The
              standard CNN reference architecture was used (see
              \autoref{sec:reference_cnn}). Our network trains much faster but plateaus
              earlier than the CNN network. We must find out how to bring the
              asymptotes of these two curves closer together.}
      \label{fig:comparison}
  \end{figure}



\subsubsection{Data Augmentation}
  The other interesting thing we have found so far is our model is far less
  dependent on data augmentation methods. In particular, taking out the random
  cropping and shifting from the pure CNN method resulted in a drastic
  reduction of performance, while it only marginally affected the Scatternet
  plus CNN performance.

\subsection{Improved Analysis Methods}
  One benefit of having set filters in the Scatternet design versus purely
  learned ones can be seen in the difference between the axes labels of
  \autoref{fig:reference_filt_energies} and \autoref{fig:scat1_filt_energies}.
  We were able to label the slices in the Scatternet design, see that some of
  the slices were not being used much in the learned CNN, and design
  experiments that removed these slices, which improved training time and
  accuracy. This kind of targetted design would be
  hugely beneficial to the field of CNNs.

  Unfortunately, we have not yet fine tuned our visualizations. The figures in
  \autoref{fig:inv_scat} are only toy examples at this stage. They show that
  it is possible to get images like those from the work of
  \citet{zeiler_visualizing_compact_2014} from a subset of Scatternet
  coefficients. 

\section{Future Work}\label{sec:future_work}
  
\subsection{Closing the Gap}\label{sec:closing_the_gap}
  The first layer of a CNN \emph{can} be replaced with a first order
  Scatternet, with currently a small loss of performance.  Getting this result is
  promising, but it needs further work. If we were to reduce this gap, we would
  have developed a network that could train \emph{faster} than current methods.
  While this is not our primary research goal, it will nonetheless be a useful
  addition to the field.

  We need to spend some time researching the cause for the current performance
  gap. We must revisit the first layers of the CNN and draw inspiration from
  here. What is missing?

  The basis functions for the first order Scatternet are considerably fewer in
  number than used by the CNN\@. We are currently using $24$ compared to $64$
  in the pure CNN network. While relying on fewer operations is beneficial,
  this may be an easy way to increase our final accuracy. Two possibilities
  immediately come to mind --- we could add in more colour channels as
  currently, the assumption is that only low frequency is necessary for colour,
  but there certainly are some mid-low level frequency colour filters in both
  the AlexNet first layer \autoref{fig:alexnet_filters}, and the \cifar\ first
  layer \autoref{fig:tensorflow_filters}. Also, we could increase the number of
  scales we have per octave from one to two, to get a better coverage of the
  frequency space.
  
  Unfortunately, both of these additions would mean we would lose perfect
  reconstruction, but that does not mean we cannot still invert the
  representations, it just becomes slightly more difficult.

\subsection{Moving to a new Dataset}\label{sec:new_dataset}
  We believe that we are not playing wavelets to their strengths by using such
  small images as the ones found in \cifar. Further, they are very poor
  representations of the images found in the real world, and it would be
  lackluster to restrict ourselves to them. Fortunately, there are no shortages
  in choosing an alternative dataset, with PASCAL-VOC a good candidate due to
  its similar number of training samples.

  This is best done sooner rather than later. It will take time to get used to
  working on a new dataset, and then some more to fine tune our design. As
  a first step, we will look at a design like AlexNet, replacing the first layer with
  a Scatternet, as we have done for \cifar.

\subsection{Improved Visualizations}\label{sec:visualizations}
  We believe that visualizations are key to unlocking the secrets of CNNs. Our
  attempts so far at creating visualizations have taught us a great deal, but
  further work needs to be done. In particular, a recent paper by
  \citet{grun_taxonomy_2016} has given a detailed comparison and analysis of
  how visualizations have developed since the work of
  \citet{zeiler_visualizing_compact_2014}. We must spend time to study these
  works and use them to improve our visualizations method with Scatternets.

\subsection{Going Deeper}\label{sec:going_deeper}
  To really indicate a gained
  insight, we need to be able to replace more than one layer of a CNN.
  
  This does not mean that Scatternets are a poor feature extractor, they have already
  been proven to be state of the art in texture datasets
  \citep{sifre_rotation_2013, sifre_rigid-motion_2014}, but they do not help us
  gain insight into CNNs. 

  As we now have a framework that successfully mimics one layer of
  a CNN\@. We believe that the
  visualization tools that we are currently working on  will be the key to
  unlocking more information about the deeper layers of CNNs, which will in
  turn allow us to design enhancements to the Scatternets.
  
\subsection{Resiudal Scattering Layers}\label{sec:residual_scat}
  This somewhat continues on from \autoref{sec:closing_the_gap}, but also could
  lead to a complete rethink of the Scatternet design.
  We want to investigate to achieve this is inspired from
  the recent state of the art ResNets --- mentioned in
  \autoref{sec:cnn_other_trends}. Residual networks work on the basis that if
  the ideal mapping $\mathcal{H}(x)$ is close to the identity mapping, then
  it will be easier to learn $\mathcal{F}(x) := \mathcal{H}(x)-x$ than to learn
  $\mathcal{H}(x)$ directly. To do this, they construct a residual layer, shown
  again in \autoref{fig:original_resnet}.
  
  Assuming we have developed a near-optimal mapping with a first order
  Scatternet, $S^1_{J}$, then a sensible thing to do is to improve performance by
  adding in extra flexibility and the ability to better fit the nuances of the
  training set. We propose to do that with the residual layer shown in
  \autoref{fig:scatnet_res}.

  \begin{figure}
    % \subfloat[ResNet
    % Layer]{\makebox[0.5\textwidth][c]{\includegraphics[height=7cm]{images/Resnet.png}
      % \label{fig:original_resnet}}}
    % \subfloat[Scatternet
    % + Residual]{\makebox[0.5\textwidth]{\includegraphics[height=7cm]{images/Scatnet_res.png}
      % \label{fig:scatnet_res}}}
    \caption[Possible future work with residual mappings]
            {Possible future work with residual mappings.
            \subref{fig:original_resnet} shows the residual layer as used by
            \citep{he_deep_2015}. These networks started with the assumption
            that the the identity mapping $f(\cdot) = I$ was a good reference
            function, and the network should learn deviations from that.
            \subref{fig:scatnet_res} shows the proposed architecture. Assuming
            the first order Scatternet (wavelet transform + modulus + averaging) is a good
            base mapping, improve on it by allowing some deviation from it as
            a residual.}
    \label{fig:improved_design}
  \end{figure}

\subsection{Exploring the Scope of Scatternets}\label{sec:scat_scope}
  We believe that applications a designed architecture like the
  Scatternet is well suited for are unsupervised and low data tasks.
  More specifically, any method that is difficult to propagate back gradients
  (due to the lack of labelled data, or to only a few training points). We
  would like to spend some time analysing what progress can be made in these
  fields with the knowledge we have.

\subsection{Analysing Newer CNN Layers}\label{sec:new_cnn_layers}
  For the moment we have satisfied ourselves with examining architectures like
  Cuda Convnet and AlexNet, which while high end, are not state of the art.
  There have been many new layers and designs added in since then, which have
  taken modern CNNs even further. We have already covered one example, the
  Residual Unit. Another example is the `Inception Unit'
  \citep{szegedy_going_2015,szegedy_rethinking_2015}, which combines
  $5\x 5$, $3\x 3$, and $1\x 1$ convolutions of different.

  Also, there is a new trend of using stride-2 convolution with downsampling is
  something we would like to investigate more, as it could lead us to clues
  about how to use subsampling and scale in a CNN, ideas that fit very
  naturally in a wavelet based network.

\subsection{Revisiting Greedy Layer-Wise Training}\label{sec:greedy_layerwise}
  The work on using unsupervised Deep Belief Networks to progressively train
  the hidden units of a deep neural network by \citet{bengio_greedy_2007} has
  been largely abandoned recently. This is sad to see, as it looked like
  a promising, well-defined way to train deep networks. We would like to
  revisit this in the context of a Scatternet, and attempt to use this
  paradigm.

\section{Timeline}
We set our timeline for the next two years in \autoref{tab:plan}. The future
work we have discussed so far roughly falls into two categories: the first four
(\S\S~\ref{sec:closing_the_gap}--\ref{sec:going_deeper}) fall on what we can
consider the  main line of research into designing a learned CNN style network.
The second four (\S\S~\ref{sec:residual_scat}--\ref{sec:greedy_layerwise}) is
more speculative research, that we hope will allow us to explore and gain
inspiration from similar problems.

Ordering these tasks chronologically would not be very well thought out. For example, 
successfully achieving \autoref{sec:going_deeper} is the main goal of the PhD
after all, so it would be naive to say we believe we can achieve it by the end
of the second year. Instead, we believe that the main line of research will take us 
a long time, and during that time, we must explore the more speculative
research. Also, a few of the tasks will need to be revisited at certain points
--- just as importantly as not being too ambitious, we should not be too lazy.
I.e., we should not leave our first attempt at building deeper layers of a CNN
until our third year. The same goes for improving our visualizations.

\begin{table}
  \caption[Our plan for the remainder of the PhD]
          {Our plan for the remainder of the PhD. The first column lists the
          months we want to do the work in, and the second column describes the
          task, and the questions we want to answer in this time.}
  \label{tab:plan}
  %\begin{tabularx}{\textwidth}[t]{XX} \toprule
    \begin{tabularx}{\textwidth}{>{\hsize=.5\hsize}X>{\hsize=1.2\hsize}X>{\hsize=1.2\hsize}X}\toprule
    \textbf{Second Year} & & \Tstrut\Bstrut\\\midrule
    Date & {Task} & Second Task\Tstrut\Bstrut\\\midrule
    Oct--Nov & {Closing the Gap \autoref{sec:closing_the_gap}} &\\
    Dec--Jan & {Developing ResNet Layers \autoref{sec:residual_scat}} &\\
    Feb--Mar & {Improved Visualizations Research 
      \autoref{sec:visualizations}}   &\\
    Apr--May & {Moving to a new dataset (preparation)
      \autoref{sec:new_dataset}} &\\
      Jun--Jul & Moving to a new dataset (testing) \autoref{sec:new_dataset} & Going Deeper
      \autoref{sec:going_deeper}\\ 
    Aug--Sep & Going Deeper \autoref{sec:going_deeper} 
      & Recap on Second Year\Bstrut\\\midrule
% %
    \textbf{Third Year} & & \Tstrut\Bstrut\\\midrule
    Oct--Nov & {Exploring the Scope of Scatternets
      \autoref{sec:scat_scope}} & \\
    Dec--Jan & Revisiting Greedy Layer-Wise 
      Training \autoref{sec:greedy_layerwise} & Analysing Newer CNN Layers
      \autoref{sec:new_cnn_layers} \\
    Feb--Mar & Revisiting Greedy Layer-Wise
      Training\autoref{sec:greedy_layerwise} & Analysing Newer CNN Layers
      \autoref{sec:new_cnn_layers} \\
    Apr--May & {Improved Visualizations Research
      \autoref{sec:visualizations}}\\
    Jun--Jul & {Going Deeper \autoref{sec:going_deeper}}\\ 
    Aug--Sep & Going Deeper \autoref{sec:going_deeper} & Recap on Third Year\Bstrut\\\midrule
%
    \textbf{Fourth Year}  && \Tstrut\Bstrut\\\midrule
    Oct-Dec & {Writing up}\\ 
  \end{tabularx}
\end{table}
    

% \begin{table}
  % \caption[Our plan for the remainder of the PhD]
          % {Our plan for the remainder of the PhD. The first column lists the
          % months we want to do the work in, and the second column describes the
          % task, and the questions we want to answer in this time.}
  % \label{tab:plan}
  % %\begin{tabularx}{\textwidth}[t]{XX} \toprule
  % \begin{tabularx}{\textwidth}{>{\hsize=.5\hsize}X>{\hsize=1.5\hsize}X} 
    % \textbf{Second Year} & \Tstrut\Bstrut\\\midrule
    % Date & {Task}\Tstrut\Bstrut\\\midrule
    % Oct--Nov & Investigate expanding out our Scatternet design with
    % more filters.  Why are we achieving a lower performance?\Tstrut\\
    % December \& January & Add in Residual Layers to our Scatternet. What does
    % this gain us? Can it be used as a clue to measure how important the
    % layers of our Scatternet are?\\
    % February, March, April, \& May & Start to design and test fully learned
    % layers that can replace the \emph{second} layer of a CNN\@. Will need to
    % spend time examining and understanding the slices of the filters learned
    % for the second layer. Having a well defined first layer will help us, as we
    % will be able to\\
    % June \& July & Develop model for higher higher resolution images, preparing
    % the dataset and the code to  such as those from
    % Caltech or ImageNet. What needs to change in our model? How many learned
    % layers can we remove away in deeper architectures?\\\\
    % August \& September & \Bstrut\\\midrule
    % \textbf{Third Year}  & \Tstrut\Bstrut\\\midrule
    % October, November \& December & Changing the problem. Examine applying
    % Scatternets to small, high resolution datasets.\Tstrut\\
    % January, February \& March & \\
    % April, May \& June & \\
    % July, August \& September & \Bstrut\\\midrule
    % \textbf{Fourth Year} & \Tstrut\Bstrut\\\midrule
    % October, November, \& December & Writing up \Tstrut\\
  % \end{tabularx}
% \end{table}
    % 



