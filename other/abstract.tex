% ************************** Thesis Abstract *****************************
% Use `abstract' as an option in the document class to print only the titlepage and the abstract.
\begin{abstract}
  Image understanding has long been a goal for computer vision. It has proved
  to be an exceptionally difficult task due to the large amounts of variability
  that are inherent to objects in a scene. Recent advances in supervised learning
  methods, particularly convolutional neural networks (CNNs), have pushed forth the frontier
  of what we have been able to train computers to do. 

  Despite their successes, the mechanics of how these networks are able to
  recognize objects are little understood, and the networks themselves are often 
  very difficult and time-consuming to train. It is very important that we improve our
  current approaches in every way possible. 

  A CNN is built from connecting many learned convolutional layers in series.
  These convolutional layers are fairly crude in terms of signal
  processing - they are arbitrary taps of a finite impulse response filter,
  learned through stochastic gradient descent from random initial conditions. We
  believe that if we reformulate the problem, we may achieve many insights and
  benefits in training CNNs. Noting that modern CNNs are mostly viewed from and
  analyzed in the spatial domain, this thesis aims to view the convolutional
  layers in the frequency domain (viewing things in the frequency
  domain has proved useful in the past for denoising, filter
  design, compression and many other tasks). In particular, we use \emph{complex
  wavelets} (rather than the Fourier transform or the discrete wavelet
  transform) as basis functions to reformulate image understanding with deep
  networks. 

  In this thesis, we explore the most popular and well-developed form of 
  using complex wavelets in deep learning, the ScatterNet from Stephane Mallat.
  We explore its current limitations by building a DeScatterNet and found that
  while it has many nice properties, it may not be sensitive to the most
  appropriate shapes for understanding natural images.

  We then develop a \emph{locally invariant} convolutional layer, a combination of a complex wavelet
  transform, a modulus operation, and a learned mixing. To do this, we derive
  backpropagation equations and allow gradients to flow back through the
  (previously fixed) ScatterNet front end. Connecting several such
  locally invariant layers allows us to build \emph{learnable ScatterNet}, a more flexible and general
  form of the ScatterNet (while still maintaining its desired properties). 

  We show that the learnable ScatterNet can provide significant improvements
  over the regular ScatterNet when being used as a front end for a learning
  system. Additionally, we show that the locally invariant convolutional
  layer can directly replace convolutional layers in a deep CNN (and not just at the front-end). 
  The locally invariant convolutional layers naturally downsample the input
  (because of the complex modulus) while increasing the channel dimension (because of the multiple
  wavelet orientations used). This is an operation that often happens in a CNN
  by a combination of a pooling and convolutional layer. It was at these
  locations in a CNN where the learnable ScatterNet performed best, implying it
  may be useful as learnable pooling layer.
  
  Finally, we develop a system to learn complex weights that act directly on the
  wavelet coefficients of signals, in place of a convolutional layer. We call
  this layer the \emph{wavelet gain layer} and show it can be used alongside convolutional 
  layers. The network designer may then choose to learn in the pixel \emph{or}
  wavelet domains. This layer shows a lot of promise and affords more control over what 
  regions of the frequency space we want our layer to learn from. Our
  experiments show that it can improve on learning in the pixel domain for early
  layers of a CNN.

\end{abstract}
