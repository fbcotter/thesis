\chapter{Complex CNN Operations}\label{app:ch6:complex_backprop}
\def \path {freqlearn/}
\def \imgpath {freqlearn/images}
This appendix lists some of the forward and backward equations for the
complex operations we use in the \emph{gain layer}.

\section{Convolution}\label{sec:appE:complex_conv}
  %% Define the variables we will use in this section
\newcommand{\SigIn}{x}
\newcommand{\SigOut}{y}
\newcommand{\Filter}{h}
\newcommand{\SigInB}{\bmu{\SigIn}}
\newcommand{\SigOutB}{\bmu{\SigOut}}
\newcommand{\FilterB}{\bmu{\Filter}}
% \newcommand{\kk}{\bmu{k}}
  %% Now the body
  
Let us represent the complex input with $\SigInB = \SigInB_r + j\SigInB_i$,
where $\SigInB_r$ and $\SigInB_i$ are the real and imaginary parts of $\SigIn$,
and $\SigIn$ is of shape $\complexes[C\x n_1\x n_2]$. Similarly, we call $\SigOutB =
\SigOutB_r + j\SigOutB_i$ the result we get from convolving $\SigInB$ with
$\FilterB= \FilterB_r + j\FilterB_i$ where $\FilterB \in \complexes[C\x m\x m]$ and so 
$\SigOutB \in \complexes[1 \x (n_1 + m - 1) \x (n_2 + m -1)]$. 
With appropriate zero or symmetric padding, we can make
$\SigOutB$ have the same spatial shape as $\SigInB$. Now, consider the full
complex convolution to get $\SigOutB$:
\begin{equation}
 \SigOut[\nn] = \sum_{c=0}^{C-1} \sum_{\kk} \Filter[c, \kk]\ \SigIn[c, \nn - \kk]
\end{equation}
We can expand this with real and imaginary terms:
\begin{align}
  \SigOut[\nn]  =&  \sum_{c=0}^{C-1} \sum_{\kk} (\Filter_r[c, \kk] +j\Filter_i[c, \kk]) 
        (\SigIn_r[c, \nn-\kk]+ j\SigIn_i[c, \nn - \kk]) \nonumber \\
  =&  \sum_{c=0}^{C-1} \sum_{\kk} (\SigIn_r[c, \nn - \kk] \Filter_r[c, \kk] - 
                                   \SigIn_i[c, \nn - \kk] \Filter_i[c, \kk]) \nonumber \\
   &+ j\sum_{c=0}^{C-1} \sum_{\kk} (\SigIn_r[c, \nn-\kk] \Filter_i[c, \kk] + 
          \SigIn_i[c, \nn - \kk] \Filter_r[c, \kk]) \nonumber\\
\SigOut_r[\nn] + j\SigOut_i[\nn]  =& \sum_{c=0}^{C-1} \left((\SigIn_r \ast \Filter_r) - (\SigIn_i \ast \Filter_i)\right)[c, \nn] + 
          j\left((\SigIn_r \ast \Filter_i) + (\SigIn_i \ast \Filter_r)\right)[c, \nn] \label{eq:app:complex_conv}
\end{align}
Unsurprisingly, complex convolution is the sum and difference of 4 real convolutions.

We can use this fact to find the update and passthrough gradients for complex
convolution.

\textbf{Update Gradients:} We need to find $\dydx{L}{h_r}$ and $\dydx{L}{h_i}$.
From \eqref{eq:app:complex_conv} we can apply the chain rule and the properties
of real convolutions to write:
\begin{align}
  \dydx{L}{h_r} &= \dydx{L}{y_r}\dydx{y_r}{h_r} + \dydx{L}{y_i}\dydx{y_i}{h_r} \\
                &= \Delta y_r \star x_r + \Delta y_i \star x_i
\end{align}
where $\star$ is the correlation operation (to achieve spatial reversal compared
with convolution, see \autoref{sec:ch2:conv_grad}).
Similarly 
\begin{equation}
  \dydx{L}{h_i} = -\Delta y_r \star x_i + \Delta y_i \star x_r
\end{equation}

\textbf{Passthrough Gradients:} Again with application of the chain rule and the
properties of real convolution, we have:
\begin{align}
  \dydx{L}{x_r} &= \dydx{L}{y_r}\dydx{y_r}{x_r} + \dydx{L}{y_i}\dydx{y_i}{x_r} \\
                &= \Delta y_r \star h_r + \Delta y_i \star h_i \\
  \dydx{L}{x_i} &= -\Delta y_r \star h_i + \Delta y_i \star h_r
\end{align}

\section{Regularization}\label{sec:appE:complex_reg}
We must be careful with regularizing complex weights. We want to promote the
magnitude of the weights to be small but allow the phase to change unrestricted.

To do $\ell_2$ regularization we can apply $\ell_2$ to the real and imaginary parts
independently:
\begin{equation}
  \lnorm{r}{2}^2 = \lnorm{\sqrt{x_r^2+x_i^2}}{2}^2 = \frac{1}{2}\sum_{\nn} x_r[\nn]^2 + x_i[\nn]^2 = 
  \frac{1}{2}\sum_\nn x_r[\nn]^2 + \frac{1}{2}\sum_\nn x_i[\nn]^2 = \lnorm{x_r}{2}^2 + \lnorm{x_i}{2}^2
\end{equation}
But this does not hold for $\ell_1$ regularization as:
\begin{equation}
  \lnorm{r}{1} = \lnorm{\sqrt{x_r^2+x_i^2}}{1} = \sum_\nn \sqrt{x_r[\nn]^2 + x_i[\nn]^2} \neq \lnorm{x_r}{1} + \lnorm{x_i}{1}
\end{equation}

Also, for $\ell_1$ regularization the derivatives have a discontinuity at the complex
origin as:
\begin{equation}
  \dydx{\ell_1}{x_r[\nn]} = \frac{x_r[\nn]}{\sqrt{x_r[\nn]^2 + x_i[\nn]^2}}
\end{equation}
is not defined when $x_r=x_i=0$. A similar problem was mentioned in \autoref{sec:ch3:mag}
where we wanted to pass gradients through the magnitude operation of a
ScatterNet. Since we do not explicitly care if weights are zero or near zero, we choose
to handle this by setting the gradient at the origin to be 0. It is unlikely our
weights will ever be zero with this method, but if they are, we cover the case of dividing 
by zero. 

\section{ReLU Applied to the Real and Imaginary Parts Independently}\label{sec:appE:complex_relu}
If we define a nonlinearity to be $y = \sigma(x)$ where:
\begin{equation}
  y = y_r + jy_i = \max(0, x_r) + j\max(0, x_i)
\end{equation}
then the passthrough gradients are:
\begin{align}
  \dydx{L}{x_r} &= \Delta y_r \indic(x_r > 0)\\
  \dydx{L}{x_i} &= \Delta y_i \indic(x_i > 0)
\end{align}

\section{Soft Shrinkage}\label{sec:appE:soft_shrink}
Let $z = x + jy$ and $w = u+jv = \mathcal{S}(z, t)$ where we define the soft
shrinkage on a complex number $z = re^{j\theta}$ by a real threshold $t$ as:
\begin{equation}
  \mathcal{S}(z, t) = \left\{ \begin{array}{ll}
      0 & r < t \\
      (r-t)e^{j\theta} & r \geq t\\
\end{array}
\right. 
\end{equation}
This can alternatively be written as:
\begin{align}
  \mathcal{S}(z, t) &= \frac{\max(r-t, 0)}{r} z \\
                    &= gz \label{eq:appE:soft}
\end{align}
To find the pass through gradients $\dydx{L}{x},\ \dydx{L}{y}$ and update
equations $\dydx{L}{t}$ we need to find all the real and imaginary partial
derivatives. We can apply the product rule once we find
$\dydx{g}{x},\dydx{g}{y},\dydx{g}{t}$:
\begin{align}
  \dydx{g}{x} &= \left\{ \begin{array}{ll}
      0 & r < t \\
      \frac{r \dydx{r}{x} - (r-t)\dydx{r}{x}}{r^2}  & r \geq t
  \end{array} \right.  \\
  &= \frac{xt\indic(g > 0)}{r^3} \label{eq:appE:dgdx} \\
  \dydx{g}{y}&= \frac{yt\indic(g > 0)}{r^3} \label{eq:appE:dgdy} \\\\
  \dydx{g}{t}&= \frac{-\indic(g > 0)}{r} \label{eq:appE:dgdt}
\end{align}
Then from the definition of $w = u+jv$ we have $u=gx$ and $v=gy$, giving us:

\begin{minipage}{.48\linewidth}
\begin{align}
  \dydx{u}{x} &= g + \frac{x^2t\indic(g > 0)}{r^3} \\
  \dydx{u}{y} &= \frac{xyt\indic(g > 0)}{r^3} \\
  \dydx{u}{t} &= \frac{-x\indic(g > 0)}{r}
\end{align}
\vspace{5pt}
\end{minipage}
\begin{minipage}{.48\linewidth}
\begin{align}
  \dydx{v}{x} &= \frac{xyt\indic(g > 0)}{r^3} \\
  \dydx{v}{y} &=g + \frac{y^2t\indic(g > 0)}{r^3} \\
  \dydx{v}{t} &= \frac{-y\indic(g > 0)}{r}
\end{align}
\vspace{5pt}
\end{minipage}
Putting it all together, our update and passthrough gradients are:
\begin{align}
  \dydx{L}{x} &= \frac{xt\indic(g>0)}{r^3}(x\Delta u + y\Delta v) + g\Delta u \label{eq:appE:softx}\\
  \dydx{L}{y} &= \frac{yt\indic(g>0)}{r^3}(x\Delta u + y\Delta v) + g\Delta v \label{eq:appE:softy}\\
  \dydx{L}{t} &= \frac{-\indic(g > 0)}{r} (x\Delta u + y\Delta v) \label{eq:appE:softt}
\end{align}
These equations are for point-wise application of soft-thresholding. When
applied to an entire image, then we sum $\dydx{L}{t}$ over all locations.

\section{Batch Normalization and ReLU Applied to the Complex Magnitude}\label{sec:appE:bnrelu}
Again let $z= x+jy = re^{j\theta}$ and $w=u + jv = \F{ReLU}(\F{BN}(r))e^{j\theta}$. 
In \eqref{eq:ch6:bnrelu_soft} we showed that this nonlinearity is equivalent to
soft-thresholding with threshold $t= \mu_r - \frac{\sigma_r\beta}{\gamma}$ and
multiplied by a learned gain $\gamma$ divided by the tracked standard deviation of 
$r$: $\frac{\gamma}{\sigma_r}$. 

Let us call the action of this nonlinearity $\mathcal{B}$, defined by:
\begin{align}
  \mathcal{B}(z, \gamma, \beta) &= \left\{ \begin{array}{ll}
      0 & r < t \\
      \frac{\gamma}{\sigma_r}(r-t)e^{j\theta} & r \geq t\\
  \end{array} 
\right. \\
&= \frac{\gamma}{\sigma_r} \frac{\max(r-t, 0)}{r} z \\
&= g'z
\end{align}
Where $g'$ is now the $g$ from \eqref{eq:appE:soft} scaled by $\frac{\gamma}{\sigma_r}$.
It is clear from our new definition of $g'$ 
that the equations \eqref{eq:appE:softx} - \eqref{eq:appE:softt} are also scaled
by $\frac{\gamma}{\sigma_r}$. 
This immediately gives us the passthrough gradients. For the update equations,
we must find some additional information:
\begin{align}
  \dydx{t}{\beta} &= -\frac{\sigma_r}{\gamma} \\
  \dydx{t}{\gamma} &= \frac{\sigma_r \beta}{\gamma^2} \\
  \dydx{g'}{\beta} &= \frac{-\gamma \indic(g' > 0)}{\sigma_r r} \dydx{t}{\beta} = \frac{\indic(g' > 0)}{r}  \\
  \dydx{g'}{\gamma} &= \frac{\indic(g' > 0)}{\sigma_r r}\left(r - t -\dydx{t}{\gamma}\right) = 
      \frac{\indic(g' > 0)}{\sigma_r r}\left(r - t - \frac{\sigma_r \beta}{\gamma^2}
      \right)  
\end{align}
Therefore, combining these with \eqref{eq:appE:softt} we get:
\begin{align}
  \dydx{L}{\beta} &= \frac{\indic(g' > 0)}{r} (x\Delta u + y\Delta v) \\
  \dydx{L}{\gamma} &= \frac{\indic(g' > 0)}{\sigma_r r}\left(r - t - \frac{\sigma_r \beta}{\gamma^2}\right)(x\Delta u + y\Delta v) 
\end{align}

% \qquad
% \begin{aligned}[c]
  % \dydx{v}{x} &= \frac{xyt\indic(g > 0)}{r^3} = \dydx{u}{y} \\
  % \dydx{v}{y} &= \frac{y^2t\indic(g > 0)}{r^3} \\
  % \dydx{v}{t} &= \frac{-y\indic(g > 0)}{r}
% \end{aligned}
% \end{equation}
  % As a first pass, I think I shouldn't concern myself too much with analytic
  % functions and having the Cauchy-Riemann equations met. Instead, I will focus
  % on implementing the CNN with a real and imaginary component to the filters,
  % and have these stored as independent variables.\\\\
  % Unfortunately, most current neural network tools only work with real numbers,
  % so we must write out the equations for the forward and backwards passes, and
  % ensure ourselves that we can achieve the equivalent of a complex valued
  % filter.

% Consider a complex number $z=x+iy$, and the complex mapping 
% \begin{equation}
% w=f(z) = u(x,y) + iv(x,y)
% \end{equation}
% where $u$ and $v$ are called `conjugate functions'. Let us examine the
% properties of $f(z)$ and its gradient. 

% The definition of gradient for complex numbers is:
% \begin{equation} 
  % \lim_{\Delta z\to 0} \frac{f(z + \Delta z) - f(z)}{\Delta z}
% \end{equation}

% A necessary condition for $f(z,\conj{z})$ to be an analytic function is
% $\dydx{f}{\conj{z}}=0$. I.e.\ f must be purely a function of $z$, and not
% $\conj{z}$.

% A geometric interpretation of complex gradient is shown in
% \autoref{fig:app6:complex_grad}.  As $\Delta z$ shrinks to 0, what does $\Delta w$
% converge to? E.g.\ consider the gradient of approach $m=\frac{dy}{dx}=\tan
% \theta$, then the derivative is
% \begin{equation}
  % \gamma = \alpha + i\beta = D(x,y) + P(x,y)e^{-2i\theta}
% \end{equation}
% where
% \begin{align}
  % D(x,y) & =  \half(u_x + v_y + i(v_x - u_y)) \\
  % P(x,y) & =  \half(u_x - v_y + i(v_x + u_y))
% \end{align}
% $P(x,y)=\frac{dw}{d\conj{z}}$ needs to be 0 for the function to be analytic.
% This is where we get the Cauchy-Riemann equations:
% \begin{align}
  % \dydx{u}{x}&= \dydx{v}{y} \\
  % \dydx{u}{y}&= -\dydx{v}{x}
% \end{align}
% The function $f(z)$ is analytic (or regular or holomorphic) if the derivative 
% $f'(z)$ exists at all points z in a region $R$. If $R$ is the entire z-plane,
% then f is entire. 

% \begin{figure}[!h]
	% \centering
  % \subfloat{%
	\begin{tikzpicture}[>=latex,scale=0.6]
	\draw[style=help lines, color=gray!30, dashed] (-4.9,-4.9) grid[step=1cm] (4.9,4.9);
	\draw[->,ultra thick] (-5,0)--(5,0) node[right]{$x$};
	\draw[->,ultra thick] (0,-5)--(0,5) node[above]{$y$};

	\coordinate (z) at (20:4); % Define some complex numbers
	\coordinate (dz) at (100:1); % with r*exp(i \theta) form

	\draw[->,thick,blue] (0,0) -- (z) node[near end,below=-0.05] {$z$};
	\draw[->,thick,red] (z)  -- ++(dz) node[near end,right] {$\Delta z$};
	\draw[->,thick,green] (0,0) -- ($(z) + (dz)$) node[near end,above left] {$z+\Delta z$};
	\node[above,font=\large] at (current bounding box.north) {z-plane};
	\end{tikzpicture}
}
\subfloat{%
	\begin{tikzpicture}[>=latex,scale=0.6]
	\draw[style=help lines, color=gray!30, dashed] (-4.9,-4.9) grid[step=1cm] (4.9,4.9);
	\draw[->,ultra thick] (-5,0)--(5,0) node[right]{$x$};
	\draw[->,ultra thick] (0,-5)--(0,5) node[above]{$y$};

	\coordinate (w) at (-30:4); % Define some complex numbers
	\coordinate (dw) at (20:1); % with r*exp(i \theta) form

	\draw[->,thick,blue] (0,0) -- (w) node[near end,below] {$w$};
	\draw[->,thick,red] (w) -- ++(dw) node[near end,below] {$\Delta w$};
	\draw[->,thick,green] (0,0) -- ($(w) + (dw)$) node[near end, above=0.1] {$w+\Delta w$};
	\node[above,font=\large] at (current bounding box.north) {w-plane};
	\end{tikzpicture}
}

  % \mycaption{Geometric interpretation of complex gradient}{The gradient is
  % defined as $f'(z) = \lim_{\Delta z \to 0} \frac{\Delta w}{\Delta z}$. It must
  % approach the same value independent of the direction $\Delta z$ approaches
  % zero. This turns out to be a very strong and somewhat restrictive property.}
  % \label{fig:app6:complex_grad}
% \end{figure}

% \section{Grad Operator}
% Recall, the gradient is a multi-variable  generalization of the derivative. The
% gradient is a vector valued function. In the case of complex numbers, it can be
% represented as a complex number too. E.g.\ consider $W(z) = F(x,y)$ (note that
% in general it may be simple to find F given G, but they are different
% functions).

% I.e. 
% $$\nabla F = \dydx{F}{x} + i \dydx{F}{y}$$

% Consider the case when F is purely real, then 
% $F(x,y) = F(\frac{z+\conj{z}}{2},\frac{z-\conj{z}}{2i}) = G(z,\conj{z})$
% Then
% $$\nabla F = \dydx{F}{x} + i \dydx{F}{y} = 2\dydx{G}{\conj{z}}$$

% If F is complex, let $F(x,y) = P(x,y) + iQ(x,y) = G(z,\conj{z})$, then
% $$\nabla F = \left(\dydx{}{x} + i \dydx{}{y}\right)(P+iQ) 
           % = \left(\dydx{P}{x} - \dydx{Q}{x}\right) + 
                % i\left(\dydx{P}{y} + \dydx{Q}{x}\right) 
           % = 2\dydx{G}{\conj{z}}$$
% It is clear to see how the purely real case is a subset of this (set Q=0 and
% all its partials will be 0 too).

% If G is an analytic function, then $\dydx{G}{\conj{z}} = 0$ and so the gradient
% is 0, and the Cauchy-Riemann equations hold $\dydx{P}{x}=\dydx{Q}{y}$ and 
% $\dydx{P}{y}=-\dydx{Q}{x}$

% \section{Hilbert Pairs of General Functions}
% How does this affect me? I don't think I'll be able to use analytic
% non-linearities, however I may be able to have analytic filters, like those of
% the $\DTCWT$.

% The Hilbert pair of the cosine is the sine function, but what about in general?
% If $x=\delta(t)$, its Hilbert pair $jy = \frac{-j}{\pi t}$. Like the dirac
% delta function, this also has a flat spectrum, and the figure for it is shown
% below.

  % \begin{center}
    % \begin{tikzpicture}[scale=0.6]
  \draw[->] (-4,0) -- (4,0) node[right] {$x$};
  \draw[->] (0,-4) -- (0,4) node[above] {$y$};
   \draw[domain=-4:-0.1,smooth,variable=\x,blue] 
     plot ({\x},{1/(pi*\x)});
   \draw[domain=0.1:4,smooth,variable=\x,blue] 
     plot ({\x},{1/(pi*\x)});
\end{tikzpicture}

  % \end{center}

% That means if we wanted to get the Hilbert pair of a sampled signal, then we
% would have to add shifts and scales of $y$, which unfortunately has infinite
% support. We would also have to lowpass it, as we do for the sampled version
% (so their frequency spectrums are the same).

% \section{Usefulness of Complex Numbers}
% Nick made a good point in our recent meeting that when trying to use the
% complex plane, we must know/understand what it is we want to gain from the
% added representation. For the case of the \DTCWT, he converted the non-analytic
% sinusoids of the wavelet transform into an analytic signal.

% I.e.\ let us ignore the previous notation of $x=\real{z}, y=\imag{z}$ and redefine
% them to indicate the horizontal and vertical directions in an image.

% For a real wavelet transform, all of the cosine terms are:
% \begin{equation}
  % \cos \omega_1 x = \half\left(e^{j\omega_1 x} + e^{-j\omega_1 x} \right)
% \end{equation}

% If we consider $z_x = e^{j\omega_1 x}$, then this is clearly a function of both
% $z_x$ and its conjugate (as are all real valued functions). I.e. $\cos \omega_1
% x = F(z_x,\conj{z_x})$. Nick replaced this with the analytic equivalent of this
% function by adding in the Hilbert pair term.

% \begin{equation}
  % \cos \omega_1 x +j \sin \omega_1 x = e^{j\omega_1 x} = F(z_x)
% \end{equation}

% From our above definitions of analytic functions, it is clear to see that this
% is now no longer a function of the conjugate term $\conj{z_x}$, so it is analytic. The
% benefit for Nick was that now he could separably multiply the x and the
% y sinusoids to get:
% \begin{equation}
  % e^{j\omega_1 x} \times e^{j\omega_2 y} =F(z_x)F(z_y)= e^{j(\omega_1 x + \omega_2 y)} = F(z_x+z_y)
% \end{equation}

% \section{Working with Complex weights in CNNs}\label{ch6:app:complex_weights}
  % As a first pass, I think I shouldn't concern myself too much with analytic
  % functions and having the Cauchy-Riemann equations met. Instead, I will focus
  % on implementing the CNN with a real and imaginary component to the filters,
  % and have these stored as independent variables.\\\\
  % Unfortunately, most current neural network tools only work with real numbers,
  % so we must write out the equations for the forward and backwards passes, and
  % ensure ourselves that we can achieve the equivalent of a complex valued
  % filter.
