\chapter{Invertible Transforms and Optimization} \label{appC:invertible}
\def \path {freqlearn/}
\def \imgpath {freqlearn/images}

This Appendix proves that reparameterization of filters with an invertible
transform can affect linear methods like SGD. This section is referenced from 
\autoref{sec:ch6:related} in the main thesis.

We initially looked at this problem after seeing the claim in
\cite{rippel_spectral_2015} that \emph{any} invertible transform of the parameter
space would not change the update equations for linear methods like SGD. In
their work, they examine reparameterizing convolutional filters in the DFT
space. The filters are taken into the pixel domain with the inverse DFT before
regular convolution was applied. We wondered if this also applied to redundant
representations like the $\DTCWT$.

We prove in this section that this statement only holds for tight frames.
To prove this we follow some notation and theory from 
\cite{kovacevic_introduction_2008}.

\section{Background}
Consider a pair of dual frames $\{\Phi, \tilde{\Phi}\}$ where $\tilde{\Phi}$ is
the \emph{analysis} operator and $\Phi$ is the \emph{synthesis} operator. 
In $\reals[n], \complexes[n]$ $\tilde{\Phi}$ is an $n\x m$ matrix describing the
frame change (with $m \geq n$), with the \emph{dual} frame vectors as its columns.
Similarly, $\Phi$ is an $n\x m$ matrix containing the frame vectors as its
columns. The analysis and synthesis operations respectively are:
\begin{align}
  X &= \tilde{\Phi}^* x \label{eq:appC:analysis}\\
  x &= \Phi X \label{eq:appC:synthesis}
\end{align}
Where $\tilde{\Phi}^*$ is the Hermitian transpose of $\tilde{\Phi}$. As
$\tilde{\Phi}$ is the dual of $\Phi$, $\Phi \tilde{\Phi}^* = I_n$.

\section{Proof}
Consider a single filter parameterized in the pixel and frame space. In one
system, the original filter parameters are updated. In a second system, the
frame representation of them are updated. We want to track the evolution of the
two filters and compare them when the same data are present to them, they have
the same $\ell_2$ regularization rate $\lambda$ and the same learning rate $\eta$.

Let us assume that the weights at time $t$ are $\vec{w}_t$ and the frame-parameterized
filter is:
\begin{equation}
  \hat{\vec{w}}_t = \tilde{\Phi}^* \vec{w}_t \label{eq:appC:initial_condition}
 \end{equation}
 It follows from \eqref{eq:appC:synthesis} and our definition of
 $\hat{\vec{w}}_t$ that:
\begin{equation}
  \dydx{L}{\hat{\vec{w}}_t} =\dydx{\vec{w}_t}{\hat{\vec{w}}_t} \dydx{L}{\vec{w}_t} =  {\Phi}^* \dydx{L}{\vec{w}_t} \label{eq:appC:synthesis_grad}
\end{equation}

After presenting both systems with the same minibatch of samples $\mathcal{D}$
and calculating the gradient $\dydx{L}{\vec{w}_t}$ we update both parameters:
\begin{align}
  \vec{w}_{t+1} & =  \vec{w}_t - \eta \left( \dydx{L}{\vec{w}_t} + \lambda \vec{w}_t \right) \\
                &= (1-\eta\lambda)\vec{w}_t - \eta \dydx{L}{\vec{w}_t} \label{eq:appC:update} \\
  \hat{\vec{w}}_{t+1} & = (1-\eta\lambda)\hat{\vec{w}}_t - \eta \dydx{L}{\hat{\vec{w}}_t} 
\end{align}
If we left multiply \eqref{eq:appC:update} by the analysis operator we get:
% We can then compare the effect the new parameters would have on the next
% minibatch by calculating $\Phi \hat{\vec{w}}_{t+1}$. This gives us:
\begin{align}
  \tilde{\Phi}^* \vec{w}_{t+1} &= \tilde{\Phi}^* \left( (1-\eta\lambda){\vec{w}}_t - \eta\ \dydx{L}{{\vec{w}}_t}\right) \\       
                               & =  (1-\eta\lambda)\hat{\vec{w}}_t - \eta \tilde{\Phi}^* \dydx{L}{{\vec{w}}_t} 
\end{align}
In general, this does not reduce further. However, if $\tilde{\Phi} =
{\Phi}$ as is the case with tight frames \cite{kovacevic_introduction_2008},
then we can use \eqref{eq:appC:synthesis_grad} and this last line simplifies to:
\begin{align}
  \tilde{\Phi}^* \vec{w}_{t+1} &= (1-\eta\lambda)\hat{\vec{w}_t} - \eta \dydx{L}{\hat{\vec{w}}_t} \\
                               &= \hat{\vec{w}}_{t+1}
\end{align}
Which shows that they remain related at time $t+1$ given they were
related at time $t$.

This proves the simpler case for SGD, but the same result holds when
momentum terms are added due to the linearity of the update equations. This does
not hold for the Adam \cite{kingma_adam:_2014} or Adagrad
\cite{duchi_adaptive_2011} optimizers, which automatically rescale the learning
rates for each parameter based on estimates of the parameter's variance.

We mention in \autoref{sec:ch2:dtcwt_tight} that when the $\DTCWT$ uses orthogonal
wavelet transforms, as is the case with the q-shift filters \cite{kingsbury_dual-tree_2000}, then it
forms a tight frame. If the biorthogonal filters are used (as is often the case
for the first scale of the transform), it does not form a tight frame.
