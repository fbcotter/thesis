\section{Experiments}\label{sec:experiments}
In this section we examine the effectiveness of our invariant layer by testing
its performance on the well known datasets (listed in the order of increasing
difficulty):
\begin{itemize}
  \item MNIST: 10 classes, 6000 images per class, $28\x 28$ pixels per image.
  \item CIFAR-10: 10 classes, 5000 images per class, $32\x 32$ pixels per image.
  \item CIFAR-100: 100 classes, 500 images per class, $32\x 32$ pixels per image. 
  \item Tiny ImageNet\cite{li_tiny_nodate}: 200 classes, 500 images per class, 
    $64 \x 64$ pixels per image. 
\end{itemize}
Our experiment code is available at
\url{https://github.com/fbcotter/invariant_convolution}.

\subsection{Layer Introduction with MNIST}\label{sec:ch5:mnist}
To begin our experiments on the proposed locally invariant layer, we must look at the
previously mentioned hyperparameters such as how to initialize the mixing matrix $A$, 
how fast we should update the parameters, and whether or not we should use
momentum. On top of this, how well does the layer scale with the number of
output channels $C_{l+1}$?

To achieve this, we build a very simple four layer network on MNIST. The first
two layers are our proposed invariant layer, the third layer is a random
projection layer that unravels the invariant layer output and projects
down to the number of classes (10). This layer is not learned, so does not add
any power to the classifier. The fourth and final layer is a small learned
fully connected layer, that takes the 10 outputs from the projection layer and
outputs 10 new values. This is to facilitate swapping of the values from the
previous layer. 

On this network we search over a range of learning rates, momenta, weight decay
and the variance of the initialized weights in the mixing matrix. We do a grid
search over values, but use Hyperband \cite{li_hyperband:_2016} to schedule early 
stopping of poor runs. We use Tune \cite{liaw2018tune} to do a parallel distributed
search over the parameter space. For reference, we also train a system that uses
standard convolutional blocks for the first two layers. 

Finally, we use fANOVA \cite{hutter_efficient_2014} analysis to find the
importance of these hyperparameters to our layer, comparing to a standard layer.

\textbf{Not sure how to report results.} The takeaway message I want to give
here is that I found out that the most important hyperparameters were the
learning rate and the momentum. We need the learning rate to be larger than for
the conv case, about 0.5 is a good start. Momentum should be slightly smaller,
with the best value at 0.25. The initialization scheme we found to be the best
for the weights was a uniform method based on \cite{glorot_understanding_2010}:

\begin{equation}
  A_{ij} \drawnfrom U\left[ -\sqrt{\frac{6}{C_l + C_{l+1}}}, \sqrt{\frac{6}{C_l + C_{l+1}}}\
  \right]
\end{equation}


\subsection{Layer Comparison with CIFAR and Tiny ImageNet}\label{sec:conv_exp}
To compare our proposed locally invariant layer (inv) to a regular convolutional
layer (conv), we build a simple yet powerful reference network based on VGG.
It has six convolutional layers for CIFAR and eight layers for Tiny ImageNet as shown in
\autoref{tab:arch}. The initial number of channels $C$ we use is 64. Despite
this simple design, this reference architecture achieves competitive performance
for the three datasets.

This network is optimized with stochastic gradient descent with momentum. The
initial learning rate is $0.5$, momentum is $0.85$, batch size $N=128$ and
weight decay is $10^{-4}$. For CIFAR-10/CIFAR-100 we scale the learning rate by
a factor of 0.2 after 60, 80 and 100 epochs, training for 120 epochs in total.
For Tiny ImageNet, the rate change is at 18, 30 and 40 epochs (training for 45 in total).

\begin{table}[ht]
  \centering
  \caption{Base Architecture used for Convolution Layer comparison tests.
  \textsuperscript{\textdagger}indicates only used in Tiny ImageNet
  experiments.}
  \label{tab:arch}
  \begin{tabular}{c|c}
    \textbf{Name} & \textbf{Output Size}  \\\hline
    convA & $C\x H\x W$ \\
    convB & $C\x H\x W$ \\
    convC & $2C\x H/2\x W/2$\\
    convD & $2C\x H/2\x W/2$\\
    convE & $4C\x H/4\x W/4$\\
    convF & $4C\x H/4\x W/4$\\
    convG\textsuperscript{\textdagger} & $8C\x H/8\x W/8$\\
    convH\textsuperscript{\textdagger} & $8C\x H/8\x W/8$\\
    fc & num classes \\
    % conv1\_1 & $C\x H\x W$ & conv+bn+ReLU\\
    % conv1\_2 & $C\x H\x W$ & conv+bn+ReLU+maxpool \\
    % conv2\_1 & $2C\x H/2\x W/2$& conv+bn+ReLU \\
    % conv2\_2 & $2C\x H/2\x W/2$& conv+bn+ReLU+maxpool \\
    % conv3\_1 & $4C\x H/4\x W/4$& conv+bn+ReLU \\
    % conv3\_2 & $4C\x H/4\x W/4$& conv+bn+ReLU+maxpool\textsuperscript{\textdagger} \\
    % conv4\_1\textsuperscript{\textdagger} & $8C\x H/8\x W/8$& conv+bn+ReLU \\
    % conv4\_2\textsuperscript{\textdagger} & $8C\x H/8\x W/8$& conv+bn+ReLU \\
    % fc1 & num classes & average pool + fc \\
    % conv1\_1 & conv $C\x 3\x 3\x 3$, bn+ReLU\\
    % conv1\_2 & conv $C\x C\x 3\x 3$, bn+ReLU, maxpool $2\x 2$ \\
    % conv2\_1 & conv $2C\x C\x 3\x 3$, bn+ReLU \\
    % conv2\_2 & conv $2C\x 2C\x 3\x 3$, bn+ReLU, maxpool $2\x 2$ \\
    % conv3\_1 & conv $4C\x 2C\x 3\x 3$, bn+ReLU \\
    % conv3\_2 & conv $4C\x 4C\x 3\x 3$, bn+ReLU maxpool\textsuperscript{\textdagger} $2\x 2$\\
    % conv4\_1\textsuperscript{\textdagger} & conv $8C\x 4C\x 3\x 3$, bn+ReLU \\
    % conv4\_2\textsuperscript{\textdagger} & conv $8C\x 8C\x 3\x 3$, bn+ReLU \\
    % fc1 & average pool $8\x 8$, fc $4C\x \F{num classes}$
  \end{tabular}
\end{table}



We perform an ablation study where we progressively swap out convolutional
layers for invariant layers keeping the input and output activation sizes the
same. As there are 6 layers (or 8 for Tiny ImageNet), there are too many
permutations to list the results for swapping out all layers for our locally
invariant layer, so we restrict our results to swapping 1 or 2 layers. 
\autoref{tab:conv_results} reports the top-1 classification accuracies for
CIFAR-10, CIFAR-100 and Tiny ImageNet. In addition to testing on the full
datasets we report results for a reduced training set size. In the table, `invX'
means that the `convX' layer from \autoref{tab:arch} is replaced with an 
invariant layer.

Interestingly, we see improvements when one or two invariant layers are used near the
start of a system, but not for the first layer. 
\begin{table}
  \renewcommand{\arraystretch}{1.2}
  \centering
  \caption{Results for testing VGG like architecture with convolutional and
  invariant layers on several datasets. An architecture with `invX' means the
  equivalent convolutional layer `convX' from \autoref{tab:arch} was swapped for
  our proposed layer.}
  % \begin{tabular}{@{}lllcllcll@{}}
  \begin{tabular}{@{}lcclcclcc@{}}
    \toprule
    & \multicolumn{2}{c}{CIFAR-10} & \phantom{abc} & \multicolumn{2}{c}{CIFAR-100} & \phantom{abc} & \multicolumn{2}{c}{Tiny ImgNet} \\ \cline{2-9}
    \phantom{abc} & 10k   &  50k  &&  10k  &  50k && 20k & 100k \\ \midrule
    reference arch & 84.4  & 91.9  && 53.2  & 70.3 && 37.4 & 59.1 \\ \hline
    invA & 82.8 & 91.3 && 48.4 & 69.5 && 33.2 & 57.7 \\
    invB & 84.8 & 91.8  && 54.8  & 70.7 && 36.9& 59.5 \\
    invC & 85.3 & 92.3 && 54.2 & 71.2 && 37.1& 59.8\\
    invD& 83.8 & 91.2 && 51.2 & 70.1 && 37.6 & 59.3 \\
    invE& 83.3 & 91.6 && 50.3 & 70.0 && 37.8 & 59.4\\
    invF& 82.1 & 90.5 && 47.6& 68.9 && 34.0 & 57.8\\ \hline
    invA, invB& 83.1  & 90.5 && 49.8 & 68.4 && 35.0 & 57.9\\
    invB, invC& 83.8  & 91.2 && 50.6 & 69.1 && 34.6 & 57.5\\
    invC, invD& 85.1 & 92.1 && 54.3 & 70.1 && \textbf{37.9} & 59.0\\
    invD, invE& 80.2 & 89.1 && 49.0 & 67.3 && 33.9 & 57.5\\
    invA, invC& 82.8 & 90.7 && 49.5 & 69.6 && 34.0 & 56.9\\
    invB, invD& \textbf{85.4} & \textbf{92.7} && \textbf{54.6} & \textbf{71.3} && \textbf{37.9} & 59.8\\
    invC, invE& 84.8 & 91.8 && 53.5 & 70.9 && 37.6 & \textbf{60.2}\\ \bottomrule
  \end{tabular}\label{tab:conv_results}
\end{table}



\subsection{Network Comparison}\label{sec:scat_exp}
In the previous section, we examined how the locally invariant layer performs when
directly swapped in for a convolutional layer in an architecture designed
for convolutional layers. In this section, we look at how
it performs in a Hybrid ScatterNet-like \cite{oyallon_hybrid_2017,oyallon_scaling_2017},
network.

To build the second order ScatterNet, we stack two invariant layers on top of each
other. For 3 input channels, the output of these layers has $3(1 +
6 + 6 +36) = 147$ channels at $1/16$ the spatial input size. We then use 4
convolutional layers, similar to convC to convF in \autoref{tab:arch} with
$C=96$. In addition, we use dropout after these later convolutional layers with drop probability $p=0.3$.

We compare a ScatterNet with no learning in between scattering orders
(ScatNet A) to one with our proposal for a learned mixing matrix $A$ (ScatNet B). Finally,
we also test the hypothesis seen from \autoref{sec:conv_exp} about putting conv
layers before an inv layer, and test a version with a small convolutional layer
before ScatNets A and B, taking the input from 3 to 16 channels, and call these ScatNet
architectures ScatNet C and D respectively.

See \autoref{tab:hybrid_scat} for results from these experiments. It is clear from
the improvements that the mixing layer helps the Scattering front end.
Interestingly, ScatNet C and D further improve on the A and B versions
(albeit with a larger parameter and multiply cost than the mixing operation). This reaffirms that there
may be benefit to add learning before as well as inside the ScatterNet.

For comparison, we have also listed the performance of other architectures as
reported by their authors in order of increasing complexity. Our proposed ScatNet D achieves
comparable performance with the the All Conv, VGG16 and FitNet architectures.
The Deep\cite{he_identity_2016} and Wide\cite{zagoruyko_wide_2016}
ResNets perform best, but with very many more
multiplies, parameters and layers.
%and
%being quick to train --- dropping the spatial size to $8\x 8$ on CIFAR after
%only two layers is unique to the ScatterNet design. 

ScatNets A to D with 6 layers like convC to convG from \autoref{tab:arch} after
the scattering, achieve $58.1, 59.6, 60.8$ and $62.1\%$ top-1 accuracy on Tiny ImageNet. As
these have more parameters and multiplies from the extra layers we exclude them
from \autoref{tab:hybrid_scat}.
\begin{table}
  \centering
  \caption{Hybrid ScatterNet top-1 classification accuracies on CIFAR-10 and CIFAR-100. 
  $N_l$ is the number of learned convolutional layers, \#param is the number of
  parameters, and \#mults is the number of multiplies per $32\x 32\x 3$ image. An asterisk indicates
  that the value was estimated from the architecture description.}\label{tab:hybrid_scat}
  \hspace{-10pt}
  \begin{tabular}{c|ccc|ccc}
    & $N_l$ &\#param&\#mults&CIFAR-10&CIFAR-100\\ \hline 
    ScatNet A & 4 & 2.6M & 165M & 89.4 & 67.0 \\ 
    ScatNet B & 6 & 2.7M & 167M & 91.1  & 70.7 \\ %                
    ScatNet C & 5 & 3.7M & 251M & 91.6 & 70.8  \\ %           
    ScatNet D & 7 & 4.3M & 294M & 93.0 & 73.5  \\\hline %           
    All Conv\cite{springenberg_striving_2014-3} & 8 & 1.4M & 281M\textsuperscript{*} & 92.8 & 66.3 \\ %    & - \\
    VGG16\cite{liu_very_2015} & 16 & 138M\textsuperscript{*} & 313M\textsuperscript{*}  & 91.6 & -  \\ 
    FitNet\cite{romero_fitnets:_2014} & 19 & 2.5M & 382M & 91.6 & 65.0 \\ %  & - \\
    % ResNet-110\cite{he_deep_2016} & 109 & 1.7M & 255M\textsuperscript{*}& 93.6 & 74.8 \\ %              &  -\\
    ResNet-1001\cite{he_identity_2016} & 1000 & 10.2M & 4453M\textsuperscript{*}& 95.1 & 77.3 \\ %              &  -\\
    WRN-28-10\cite{zagoruyko_wide_2016} & 28 & 36.5M & 5900M\textsuperscript{*} & 96.1 & 81.2 \\ %    & - \\
  \end{tabular}
\end{table}


