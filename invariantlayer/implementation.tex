
\section{Implementation}\label{sec:implementation}
Like \cite{singh_dual-tree_2017,singh_multi-resolution_2016} we use the \DTCWT
\cite{selesnick_dual-tree_2005} for our wavelet filters $\psi_{j, \theta}$ due
to their fast implementation with separable convolutions which we will discuss
more in \autoref{sec:computation}.  A side effect of this choice is that the
number of orientations of wavelets is restricted to $K=6$.

The output of the \DTCWT is decimated by a factor of $2^j$ in each direction for
each scale $j$.  In all our experiments we set $J=1$ for each invariant layer,
meaning we can mix the lowpass and bandpass coefficients at the same resolution.
\autoref{fig:block_diagram} shows how this is done. Note that setting $J=1$ for
a single layer does not restrict us from having $J>1$ for the entire system, as
if we have a second layer with $J=1$ after the first, including downsampling
($\downarrow$), we would have:
%
\begin{equation}
  \left(\left(\left(x \conv \phi_1\right) \downarrow 2\right) \conv \psi_{1, \theta}\right) \downarrow 2 = \left(x \conv \psi_{2, \theta}\right) \downarrow 4
\end{equation}

\subsection{Memory Cost}\label{sec:memory}
A standard convolutional layer with $C_l$ input channels, $C_{l+1}$ output channels
and kernel size $L$ has $L^2C_{l}C_{l+1}$ parameters. 

The number of learnable parameters in each of our proposed invariant layers with
$J=1$ and $K=6$ orientations is:
%
\begin{equation}
  \text{\#params} = (JK+1)C_{l}C_{l+1} = 7C_{l}C_{l+1}
\end{equation} 
%
The spatial support of the wavelet filters is typically $5\x 5$ pixels or more,
and we have reduced $\text{\#params}$ to less than $3\x3$ per filter, while
producing filters that are significantly larger than this.

\subsection{Computational Cost}\label{sec:computation}
A standard convolutional layer with kernel size $L$ needs $L^2C_{l+1}$
multiplies per input pixel (of which there are $C_{l}\x H\x W$).

As mentioned in \autoref{sec:memory}, we use the \DTCWT for our complex, shift
invariant wavelet decomposition. We use the open source Pytorch implementation
of the \DTCWT \cite{cotter_pytorch_2018} as it can run on GPUs and
has support for backpropagating gradients.

There is an overhead in doing the wavelet decomposition for each input channel. A
regular discrete wavelet transform (DWT) with filters of length $L$ will have
$2L\left(1-2^{-2J}\right)$ multiplies for a $J$ scale decomposition. A \DTCWT
has 4 DWTs for a 2-D input, so its cost is $8L\left(1-2^{-2J}\right)$, with
$L=6$ a common size for the filters. It is important to note that unlike the
filtering operation, this does not scale with $C_{l+1}$, the end result being that as
$C_{l+1}$ grows, the cost of $C_l$ forward transforms is outweighed by that of the mixing
process.

Because we are using a decimated wavelet decomposition, the sample rate decreases after each
wavelet layer. The benefit of this is that the mixing process then only works on
one quarter the spatial size after one first scale and one sixteenth the spatial
after the second scale. Restricting ourselves to $J=1$ as we mentioned in
\autoref{sec:implementation}, the computational cost is then:

\begin{equation}
  % \frac{7}{4}C_{l+1} + 48 \label{eq:comp}
  \underbrace{ \frac{7}{4}C_{l+1} }_{\textrm{mixing}} +
  \underbrace{\vphantom{\frac{7}{4}} 36}_{\textrm{\DTCWT}} \quad
  \textrm{multiplies per input pixel}\label{eq:comp}
\end{equation}
In most CNNs, $C_{l+1}$ is several dozen if not several
hundred, which makes \autoref{eq:comp} significantly smaller than
$L^2C_{l+1}=9C_{l+1}$ multiplies for $3\x 3$ convolutions.


\begin{tikzpicture}[%
  path image/.style={
    path picture={
      \node at (path picture bounding box.center) {
        \includegraphics[height=2.0cm]{#1}
      };
    }
  }, 
  path pic/.style={
    path picture={
      \node at (path picture bounding box.center) {
        \includegraphics[height=1.2cm]{#1}
      };
    }
  }, 
  path pic2/.style={
    path picture={
      \node at (path picture bounding box.center) {
        \includegraphics[height=0.8cm]{#1}
      };
    }
  }, 
  scale=0.6]

  \tikzcuboid{
  shiftx=-1.5cm,
  shifty=-2.5cm,
  scale=0.5,
  anglex=0, 
  angley=90, 
  anglez=230,
  dimx=3, 
  dimy=3, 
  dimz=6,
  densityx=1, 
  densityy=1, 
  densityz=1,
  shade=false,
  emphedge=true,
  shadeopacity=0,
  emphstyle/.style={rounded corners=0.2pt,line width=0.3mm},
  front/.style={draw=blue!50!white,fill=blue!50!white},%
  right/.style={draw=blue!50!white,fill=blue!50!white},%
  top/.style={draw=blue!50!white,fill=blue!50!white},%
  drawxdims=true,
  dimxval=W,
  drawydims=true,
  dimyval=H,
  drawzdims=true,
  dimzval=C_l,
  }
  \draw (0, .3, 0) node {\large{$x^{(l)}$}};
  \draw [path image=\imgpath/waveys.png, draw=black] (1.5,1.5,0) rectangle (6,0,0);
  \draw (3.75, 1.9, 0) node {\large{$\psi_{j, \theta}$}};
  \draw (1.5,0,0) -- (3,-1.7,0);
  \draw (6,0,0) -- (3.3,-1.7,0);
  \draw [path pic2=\imgpath/lowpass.png, draw=black] (2.5,-2.5,10) rectangle (3.5,-1.5,10);
  \draw (3, -1.1, 10) node {\large{$\phi_{j}$}};
  \draw (3.5,-1.5,10) -- (3.5,-1.7,8);

  % \draw (2.4, -1.5, 0) node {\Large{$\conv$}};
  \draw (2.5, -1.5, 3) node {\Large{$\conv$}};
  \draw [->, fill=gray!30,ultra thick] (4, -1.5, 0) -- (5, -1.5, 0);
  \draw [->, fill=gray!30,ultra thick] (4.5, -1.5, 9) -- (5.85, -1.5, 11);

  \tikzcuboid{
  shiftx=3cm,
  shifty=-1.7cm,
  shiftz=0,
  scale=0.3,
  dimx=1, dimy=1, dimz=4,
  densityx=2, densityy=2, densityz=2,
  drawxdims=false,
  drawydims=false,
  drawzdims=true,
  dimzval=12,
  front/.style={draw=yellow!70!white,fill=yellow!70!white},%
  right/.style={draw=yellow!70!white,fill=yellow!70!white},%
  top/.style={draw=yellow!70!white,fill=yellow!70!white},%
  }
  \tikzcuboid{
  shiftz=8/0.3,
  scale=0.3,
  dimx=1, dimy=1, dimz=1,
  densityx=2, densityy=2, densityz=2,
  drawxdims=false,
  drawydims=false,
  drawzdims=false,
  }


  \tikzcuboid{
  shiftx=6cm,
  shifty=-2.0cm,
  shiftz=0.8cm,
  scale=0.5,
  dimx=2, dimy=2, dimz=4,
  densityx=4, densityy=4, densityz=2,
  drawzdims=true,
  dimzval=C_l,
  front/.style={draw=blue!50!white,fill=blue!50!white},%
  right/.style={draw=blue!50!white,fill=blue!50!white},%
  top/.style={draw=blue!50!white,fill=blue!50!white},%
  }

  \tikzcuboid{
  shiftx=6cm,
  shifty=-2cm,
  shiftz=0cm,
  scale=0.5,
  dimx=2, dimy=2, dimz=24,
  densityx=2, densityy=2, densityz=2,
  drawxdims=true,
  dimxval=\frac{W}{2},
  drawydims=true,
  dimyval=\frac{H}{2},
  drawzdims=true,
  dimzval=12C_l,
  front/.style={draw=blue!50!white,fill=blue!50!white},%
  right/.style={draw=blue!50!white,fill=blue!50!white},%
  top/.style={draw=blue!50!white,fill=blue!50!white},%
  }

  % \draw [->, fill=gray!30,ultra thick] (8.5, -1.5, 0) -- (10.5, -1.5, 0) node[midway, above] (mag) {\large{$\lvert\cdot\rvert$}};
  \draw [->, fill=gray!30,ultra thick] (8.5, -1.5, 0) -- (10.5, -1.5, 0) node[midway, above] (mag) {\Large{ $\lvert \cdot \rvert$} };
  \draw [path pic=\imgpath/mag.png, draw=white] (9.25,0,0) rectangle (11.75,2,0);
  \draw (9.25,0,0) -- (mag.north);
  \draw (11.75,0,0) -- (mag.north);
  \draw[->, fill=gray!30, ultra thick] (8.5, -2, 10) -- (10, -3, 3);

  \tikzcuboid{
  shiftx=11.5cm,
  shifty=-2cm,
  scale=0.5,
  dimx=2, dimy=2, dimz=12,
  densityx=2, densityy=2, densityz=2,
  dimzval=6C_l,
  drawxdims=false,
  drawydims=false,
  front/.style={draw=blue!50!white,fill=blue!50!white},%
  right/.style={draw=blue!50!white,fill=blue!50!white},%
  top/.style={draw=blue!50!white,fill=blue!50!white},%
  }
  \tikzcuboid{
  shiftx=11.5cm,
  shifty=-2cm,
  shiftz=8,
  scale=0.5,
  drawxdims=true,
  drawydims=true,
  dimx=2, dimy=2, dimz=4,
  densityx=2, densityy=2, densityz=2,
  dimzval=C_l,
  front/.style={draw=blue!50!white,fill=blue!50!white},%
  right/.style={draw=blue!50!white,fill=blue!50!white},%
  top/.style={draw=blue!50!white,fill=blue!50!white},%
  }
  \draw (13.2, 0.8, 0) node {\large{$z^{(l+1)}$}};
  \draw (20.5, 0.3, 0) node {\large{$y^{(l+1)}$}};
  \draw (24.5, 0.3, 0) node {\large{$x^{(l+1)}$}};
  \draw (14, -1.5, 0) node {\Large{$\conv$}};

  \tikzcuboid{
  shiftx=15.2cm,
  shifty=-1.0cm,
  shiftz=0,
  scale=0.5,
  dimx=0.4, dimy=0.4, dimz=14,
  densityx=5, densityy=5, densityz=2,
  drawxdims=false,
  drawydims=false,
  drawzdims=false,
  front/.style={draw=red!50!white,fill=red!50!white},%
  right/.style={draw=red!50!white,fill=red!50!white},%
  top/.style={draw=red!50!white,fill=red!50!white},%
  }
  \tikzcuboid{
  shifty=-1.65cm,
  }
  \tikzcuboid{
  shifty=-3.0cm,
  scale=0.5,
  drawxdims=true,
  dimxval=1,
  drawydims=true,
  dimyval=1,
  drawzdims=true,
  dimzval=7C_l,
  }
  \draw (15.5, -1.8, 0) node {$\vdots$};
  \draw [<->] (15.7, -0.8, -3) -- (15.7, -3, -3) node[near start, right] {$C_{l+1}$};
  \draw [->, fill=gray!30,ultra thick] (17.5, -1.5, 0) -- (18.5, -1.5, 0);

  \tikzcuboid{
  shiftx=19.5cm,
  shifty=-2.25cm,
  scale=0.5,
  dimx=2, dimy=2, dimz=6,
  densityx=4, densityy=4, densityz=2,
  drawzdims=false,
  drawxdims=false,
  drawydims=false,
  front/.style={draw=blue!50!white,fill=blue!50!white},%
  right/.style={draw=blue!50!white,fill=blue!50!white},%
  top/.style={draw=blue!50!white,fill=blue!50!white},%
  }
  \draw [->, fill=gray!30,ultra thick] (21.5, -1.5, 0) -- (22.5, -1.5, 0)
    node[midway, above] {$\sigma$};

  \tikzcuboid{
  shiftx=23.5cm,
  shifty=-2.25cm,
  scale=0.5,
  dimx=2, dimy=2, dimz=6,
  densityx=4, densityy=4, densityz=2,
  drawzdims=true,
  dimzval=C_{l+1},
  drawxdims=true,
  dimxval=\frac{W}{2},
  drawydims=true,
  dimyval=\frac{H}{2},
  front/.style={draw=blue!50!white,fill=blue!50!white},%
  right/.style={draw=blue!50!white,fill=blue!50!white},%
  top/.style={draw=blue!50!white,fill=blue!50!white},%
  }

\end{tikzpicture}


