\begin{table}[t]
  \renewcommand{\arraystretch}{1.2}
  \centering
  \mycaption{Hybrid ScatterNet models}{Hybrid ScatterNet architectures used
  for experiments on CIFAR-10 and CIFAR-100. ScatNet A is a regular ScatterNet,
  and ScatNet B is our proposed learnable ScatterNet. Both ScatNet A and
  ScatNet B the same back end, the architecture shown in
  \subref{tab:ch5:net_ref}. $N_c$ is the number of output classes; in our
  experiments, we set the channel multiplier to be $C=96$.}
  \label{tab:ch5:cifar_arch2}
  \subfloat[ScatNet A Front End]{%
    \label{tab:ch5:net_scatA}
    \begin{tabular}{l l}
      \toprule
      Layer & Act. Size \\\midrule
      \begin{tabular}{@{}l@{}}
        scatA, no $w$ \\
        scatB, no $w$ \\
      \end{tabular} &
      \begin{tabular}{@{}l@{}}
        $3\x 32\x 32$ \\ $21 \x 16\x 16$ \\ $147\x 8 \x 8$ \\
      \end{tabular}\\
      \bottomrule
    \end{tabular}
  }
  \subfloat[ScatNet B Front End]{%
    \label{tab:ch5:net_scatB}
    \begin{tabular}{l l}
      \toprule
      Layer & Act. Size \\\midrule
      \begin{tabular}{@{}l@{}}
        invA, $A \in \reals[21\x 21]$ \\
        invB, $A \in \reals[147\x 147]$ \\
      \end{tabular} &
      \begin{tabular}{@{}l@{}}
        $3\x 32\x 32$ \\ $21 \x 16\x 16$ \\ $147\x 8 \x 8$ \\
      \end{tabular}\\
      \bottomrule
    \end{tabular}
  }\\
  \subfloat[BackEnd]{%
    \label{tab:ch5:net_ref}
    \begin{tabular}{l l}
      \toprule
      \begin{tabular}{@{}l@{}}
        convC, $w\in \reals[2C\x 147\x 3\x 3]$ \\
        convD, $w\in \reals[2C\x 2C\x 3\x 3]$ \\
        convE, $w\in \reals[4C\x 2C\x 3\x 3]$ \\
        convF, $w\in \reals[4C\x 4C\x 3\x 3]$ \\
        avg pool $8\x 8$ \\
        fc1, $4C\x N_c$
      \end{tabular} &
      \begin{tabular}{@{}l@{}}
        $147\x 8\x 8$ \\$2C\x 8\x 8$ \\$2C\x 8\x 8$ \\$4C\x 8\x 8$ \\$4C\x 8\x 8$ \\
        $4C$ \\ $N_c$ \\
      \end{tabular}\\
      \bottomrule
    \end{tabular}
    } \\
\end{table}
\begin{table}[t]
  \renewcommand{\arraystretch}{1.2}
  \centering
  \mycaption{Hybrid ScatterNet models with convolutional layer
  first}{The two ScatNet models are similar to the
  learnable ScatterNet from \autoref{tab:ch5:net_scatB} but with a small
  convolutional layer (`conv0') before it. ScatNet C ensures the same $147\x 8\x
  8$ output size as the models in \autoref{tab:ch5:cifar_arch2} but ScatNet D
  has a larger output size, allowing for the natural growth of a second-order
  ScatterNet model from $C$ input channels to $49C$ output channels.}
  % \subfloat[Reference 2]{%
    % \label{tab:ch5:net_ref2}
    % \begin{tabular}{l l}
      % \toprule
      % Layer & Act. Size \\\midrule
      % \begin{tabular}{@{}l@{}}
        % conv0, $w\in \reals[16\x 3\x 3\x 3]$ \\
        % convA, $w\in \reals[50\x 16\x 3\x 3]$ \\
        % pool1, max pooling $2\x 2$ \\
        % convB, $w\in \reals[147\x 50\x 3\x 3]$\\
        % pool2, max pooling $2\x 2$ \\
      % \end{tabular} &
      % \begin{tabular}{@{}l@{}}
        % $3\x 32\x 32$ \\ $16\x 32\x 32$ \\ $50 \x 32\x 32$ \\ $50 \x 16\x 16$ \\ $147\x 16\x 16$ \\
        % $147\x 8 \x 8$ \\
      % \end{tabular}\\
      % \bottomrule
    % \end{tabular}
  % }\\
  \label{tab:ch5:net_scatCD}
  \subfloat[ScatNet C]{%
    \label{tab:ch5:net_scatC}
    \begin{tabular}{l l}
      \toprule
      Layer & Act. Size \\\midrule
      \begin{tabular}{@{}l@{}}
        conv0, $w\in \reals[16\x 3\x 3\x 3]$ \\
        invA, $A \in \reals[50\x 112]$ \\
        invB, $A \in \reals[147\x 350]$ \\
      \end{tabular} &
      \begin{tabular}{@{}l@{}}
        $3\x 32\x 32$ \\ $16\x 32\x 32$ \\ $50 \x 16\x 16$ \\ $147\x 8 \x 8$ \\
      \end{tabular}\\
      \bottomrule
    \end{tabular}
  }
  \subfloat[ScatNet D]{%
    \label{tab:ch5:net_scatD}
    \begin{tabular}{l l}
      \toprule
      Layer & Act. Size \\\midrule
      \begin{tabular}{@{}l@{}}
        conv0, $w\in \reals[16\x 3\x 3\x 3]$ \\
        invA, $A \in \reals[112\x 112]$ \\
        invB, $A \in \reals[784\x 784]$ \\
      \end{tabular} &
      \begin{tabular}{@{}l@{}}
        $3\x 32\x 32$ \\ $16\x 32\x 32$ \\ $112 \x 16\x 16$ \\ $784\x 8 \x 8$ \\
      \end{tabular}\\
      \bottomrule
    \end{tabular}
  }
\end{table}
