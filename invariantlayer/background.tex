\section{Related Work}\label{sec:ch5:related}

There have been several similar works that look into designing new convolutional
layers by separating them into two stages --- a first stage that performs a
non-standard filtering process, and a second stage that combines the first stage
into single activations. The inception layer 
\cite{szegedy_rethinking_2015} by \citeauthor*{szegedy_rethinking_2015} does this by filtering with different
kernel sizes in the first stage, and then combining with a $1\x 1$ convolution
in the second stage. \citeauthor*{ioannou_training_2015} also do something similar by making
a first stage with horizontal and vertical filters, and then combining in the
second stage again with a $1\x 1$ convolution\cite{ioannou_training_2015}. But perhaps the most similar
works are those that use a first stage with fixed filters, combining them in a
learned way in the second stage. Of particular note are:
\begin{itemize}
\item 
\citetitle{juefei-xu_local_2016} \cite{juefei-xu_local_2016}. This paper builds a
first stage with a small $3\x 3$ kernel filled with zeros, and randomly insert
$\pm 1$ in several locations, keeping a set sparsity level. This builds a very
crude spatial differentiator in random directions. The output of the first stage
is then passed through a sigmoid nonlinearity before being mixed with a $1\x 1$
convolution. The imposed structure on the first stage was found to be a good
regularizer and prevented overfitting, and the combination of the mixing in the
second layer allowed for a powerful and expressive layer, with performance near
that of a regular CNN layer.

\item
``DCFNet: Deep Neural Network with Decomposed Convolutional Filters"
\cite{qiu_dcfnet:_2018}. This paper decomposes convolutional filters as linear
combinations of Fourier Bessel and random bases.  The first stage projects the
inputs onto the chosen basis, and the second stage learns how to mix these
projections with a $1\x 1$ convolution. Unlike \cite{juefei-xu_local_2016} this
layer is purely linear. The supposed advantage being that the
basis can be truncated to save parameters and make the input less susceptible to
high frequency variations. The work found that this layer had marginal benefits
over regular CNN layers in classification, but had improved stability to noisy
inputs. 

\end{itemize}

\section{Recap of Useful Terms}\label{sec:ch5:background}

\subsection{Convolutional Layers}\label{sec:ch5:conv}

Let the output of a CNN at layer $l$ be 
$ \cnnlact{x}{l}{c}{\xy}, \quad c\in \{0, \ldots C_l-1\}, \xy \in \reals[2]$
where $c$ indexes the channel dimension, and $\xy$ is a vector of coordinates
for the spatial position. Of course, $\xy$ is typically sampled on a grid, but
we keep it continuous to more easily differentiate between the spatial and
channel dimensions. Recall from \eqref{eq:ch2:conv4} and \eqref{eq:ch2:conv4a} that
a convolutional layer in a standard CNN is defined by the two operations:
%
\begin{eqnarray} 
  \cnnlact{y}{l+1}{f}{\xy} &=& \sum_{c=0}^{C_l - 1}  x^{(l)}(c,\xy) \conv h^{(l)}_{f}(c, \xy)
    \label{eq:ch5:conv}\\
    \cnnlact{x}{l+1}{f}{\xy} & = & \sigma \left( \cnnlact{y}{l+1}{f}{\xy} \right) \label{eq:ch5:nonlin}
\end{eqnarray}

where $\cnnfilt{l}{f}{c}{\xy}$ is the $f$th filter of the $l$th layer with $C_l$
different point spread functions, and $f \in \{0, \ldots, C_{l+1}-1 \}$. $\sigma$ is a nonlinearity 
such as the ReLU, possibly combined with scaling such as batch normalization. The convolution
is done independently for each $c$ in the $C_l$ channels and the resulting outputs are
summed together to give one activation map. 

\subsection{Wavelet Transforms}\label{sec:ch5:wavelets}
Recall from \eqref{eq:ch2:wave2} that:
\begin{equation}
  \mathcal{W}x(c, \xy) = \left\{x(c, \xy) \ast \phitd_J(\xy),\ x(c, \xy) \ast \psitd_{\lambda}(\xy) \right\}_{\lambda} \label{eq:ch5:wave2}
\end{equation}
Where $\psitd_\lambda$ is a mother wavelet dilated by $2^j,\ 1 \leq j \leq J$ and rotated by
$\theta = \frac{\pi + 2k\pi}{12},\ 0\leq k < 6$:
%
\begin{equation}
  \psi_{j, \theta}(\xy) = 2^{-j}\psi \left(2^{-j} r_{-\theta} \xy\right)
\end{equation}
Define the set of all possible $\lambda$s as $\Lambda$ whose size is $|\Lambda | = JK$.
%
\subsection{Scattering Transforms}\label{sec:ch5:scatter}
As the real and imaginary parts of complex wavelets are in quadrature with
each other, taking the modulus of the resulting transformed coefficients removes
the high frequency oscillations of the output signal while preserving the energy
of the coefficients over the frequency band covered by $\psi_\lambda$. This is
crucial to ensure that the scattering energy is concentrated towards
zero-frequency as the scattering order increases, allowing sub-sampling.
% We experimentally show in \autoref{sec:??} that it plays a vital role in the
%classification process. 
We define the wavelet modulus propagator to be:
%
\begin{equation}
  \label{eq:ch5:wave_mod}
\widetilde{\mathcal{W}}x(c, \xy) = \left\{ x(c, \xy) \conv \phi_{J}(\xy),\ |x(c, \xy) \conv \psi_\lambda (\xy) | \right\}_{\lambda \in \Lambda} 
\end{equation}
The modulus terms are called $U[\lambda] x = \lvert x \conv \psi_\lambda \rvert$, and the scattering terms
are $S[\lambda] x = U[\lambda]x \conv \phi_J (\xy)$.
