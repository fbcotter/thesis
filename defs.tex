\newcommand{\bmu}[1]{\mathbf{#1}}           % bold math upright
\newcommand{\xy}{\bmu{u}}
\newcommand{\nn}{\bmu{n}}
\newcommand{\zz}{\bmu{z}}
\newcommand{\ww}{\bm{\omega}}
\newcommand{\psitd}{\psi}
\newcommand{\psiod}{\uppsi}
\newcommand{\phitd}{\phi}
\newcommand{\phiod}{\upphi}


% Define some nicely typesetted words
\newcommand{\CWT}{\mathbb{C}\mathrm{WT}} % Nice display of CWT 
\newcommand{\DTCWT}{\mathrm{DT}\CWT}
\newcommand{\cifar}{CIFAR-10}
\newcommand{\x}{\times}                     % I don't like having to write out \times so often
\newcommand{\degs}{^{\circ}}
\newcommand{\conv}{\ast}
\newcommand{\wrt}{w.r.t.}
\def\definedas{\triangleq}
\newcommand{\mycaption}[2]{\caption[#1]{\textbf{#1.} #2}} % Use \macpation{bold font}{rest}
\newcommand{\mycaptionnodot}[2]{\caption[#1]{\textbf{#1} #2}} % Use \macpation{bold font}{rest}

\DeclareMathOperator*{\argmin}{arg\,min}
\DeclareMathOperator*{\argmax}{arg\,max}
% Generic command to make upright words in mathmode 
\newcommand{\F}[1]{\mathrm{#1}}

% And some particularly useful operators
\newcommand{\sgn}{\F{sgn}}
\newcommand{\tr}{\F{trace}}
\newcommand{\diag}{\F{diag}}

% Declare the floor and ceiling operators
\DeclarePairedDelimiter\ceil{\lceil}{\rceil}
\DeclarePairedDelimiter\floor{\lfloor}{\rfloor}

% Define a new command for the euclidean norm of an expression
\newcommand{\norm}[1]{\left\lVert #1 \right\rVert}
\newcommand{\lnorm}[2]{{\left\lVert #1 \right\rVert}_{#2}}
\newcommand{\energy}[1]{\left\lVert #1 \right\rVert^2}
\newcommand{\loss}{L}
\newcommand{\ltwo}{$\ell_2$}
\newcommand{\logloss}{\mathcal{L}}
\newcommand{\dydx}[2]{\frac{\partial #1}{\partial #2}}
\newcommand{\dLdx}[1]{\frac{\partial \mathcal{L}}{\partial #1}}
\newcommand{\conj}[1]{\bar{#1}}
\newcommand{\half}{\frac{1}{2}}
% \newcommand{\expected}[2][]{\ensuremath{{\mathbb{E}_{#1}\left[#2\right]}}\xspace}
% \newcommand{\drawnfrom}{\ensuremath{\sim}\xspace}
\newcommand{\expected}[2][]{\mathbb{E}_{#1}\left[#2\right]}
\newcommand{\drawnfrom}{\sim}
\newcommand{\real}[1]{\F{Re}\left(#1\right)}
\newcommand{\imag}[1]{\F{Im}\left(#1\right)}
\newcommand{\DFT}[1]{\F{DFT}\left\{#1\right\}}
\newcommand{\IFT}[1]{\F{DFT}^{-1}\left\{#1\right\}}

% Some vector spaces. These can often be used outside of equations so we will
% add in the ensure math mode. These all have optional arguments which are the
% dimensionality of the space. I.e. if you want to say something belongs to the
% 2 dimensional space of reals, we can call \reals[2]. The Galois spaces need
% to be given the order of their space, so a 3-dimensional binary space would
% be \galois[3]{2}, or simply \binaries[3]
% \newcommand{\reals}[1][]{\ensuremath{{\mathbb{R}}^{#1}}\xspace}
\newcommand{\reals}[1][]{\mathbb{R}^{#1}}
\newcommand{\complexes}[1][]{\mathbb{C}^{#1}}
\newcommand{\integers}[1][]{\mathbb{Z}^{#1}}
\newcommand{\galois}[2][]{{\mathbb{F}_{#2}}^{#1}}
% \newcommand{\complexes}[1][]{\ensuremath{{\mathbb{C}}^{#1}}\xspace}
% \newcommand{\integers}[1][]{\ensuremath{{\mathbb{Z}}^{#1}}\xspace}
% \newcommand{\galois}[2][]{\ensuremath{{\mathbb{F}_{#2}}^{#1}}\xspace}
\newcommand{\binaries}[1][]{\galois[#1]{2}\xspace}

\newcommand{\cnnfilt}[4]{h^{(#1)}_{#2}\left(#3, #4 \right)}
\newcommand{\cnndfilt}[4]{h^{(#1)}_{#2}\left[#3, #4 \right]}
\newcommand{\cnnact}[3]{#1\left(#2, #3\right)}
\newcommand{\cnnlact}[4]{#1^{(#2)}\left(#3, #4\right)}
\newcommand{\cnndlact}[4]{#1^{(#2)}\left[#3, #4\right]}
\renewcommand{\vec}[1]{\mathbf{#1}}
\newcommand{\Mallat}{Mallat et.\ al.}
