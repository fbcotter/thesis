\chapter{Front Ends}

This chapter is about having a wavelet transform/scatternet front end to a deep
learning system.

Previous work including Oyallon and Singh have explored ways in which
ScatterNets can be improved on for image analysis tasks. This chapter explores
some alternate methods we have.

\section{Fast GPU Implementation}

\section{Something}
Where to modify the scatternet? We could look at the difference in training cost
when we drop the second order coefficients.

Perhaps just doing the wavelet transform and complex magnitude is enough?

Nick wanted me to move the magnitude before mixing the channels. The argument
being that taking the magnitude means the output becomes shift invariant. I.e.,
we only need to see that there is energy in a wavelet band, and then we can
accommodate shifts.

The question is though, how much do we need the phase? Also can we impose some
priors on the distribution of subbands for CNN activations? We certainly can for
the image statistics.

\section{Relu Properties}
What is the effect of the ReLU on the activations? Show that it sparsifies
things. This wasn't really the case for resnet, and the sparsity stayed at around 50\% the whole way
through.

\section{Activation Statistics}
I want to take the DTCWT of activations throughout a neural network and look at
the pdf over the subband energy.

\section{Wavelet CNNs}
I wonder if it's worth looking at the work of \cite{fujieda_wavelet_2018} and doing something
similar to that 

\section{Properties of a Scatternet}
