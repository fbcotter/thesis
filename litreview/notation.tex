\section{Notation}
We define standard notation to help the reader better understand figures and
equations. Many of the terms we define here relate to concepts that have not been
introduced yet, so may be unclear until later.

\begin{itemize}
  \item \textbf{Pixel coordinates}\\
    When referencing spatial coordinates in an image, the preferred index
    is $\bmu{u}$ for a 2D vector of coordinates, or $[u_1,u_2]$ if we wish to
    specify the coordinates explicitly. $u_1$ indexes rows from top to bottom
    of an image, and $u_2$ indexes columns from left to right. We typically use
    $H\x W$ for the size of the image, (but this is less strict). I.e.,\ 
    $u_1 \in \{0, 1, \ldots H-1\}$ and $u_2 \in {0, 1, \ldots W-1}$. 

  \item \textbf{Convolutional networks}\\
    Image convolutional neural networks often work with 4-dimensional arrays. In particular,
    mini-batches of images with multiple channels. When we need to, we index over the 
    minibatch with the variable $n$ and over the channel dimension with $c$. For example, we can
    index an activation $\bmu{x}$ by $\bmu{x}[n, c, u_1, u_2]$.

    To distinguish between features, filters, weights and biases of different
    levels in a deep network, we may add a layer subscript, or $l$ for the
    general case, i.e.,\ $\bmu{z}_l[n, c, \bmu{u}]$ indexes the feature map at the $l$-th
    layer of a deep network. 
    
  \item \textbf{Fourier transforms}\\
    When referring to the Fourier transform of a function, $f$, we typically
    adopt the overbar notation: i.e., $\mathcal{F}\{f\} = \bar{f}$. 

  \item \textbf{Wavelet Filter Banks}\\
    Using standard notation, we define the scaling function as $\phi$ and the wavelet function 
    as $\psi$. For a filter bank implementation of a wavelet transform, we use $h$ for analysis and
    $g$ for synthesis filters.    

    In a multiscale environment, $j$ indexes scale from $\{1,2, \ldots, J\}$. For 2D complex
    wavelets, $\theta$ indexes the orientation at which the wavelet has the largest response, i.e.,\
    $\psi_{\theta, j}$ refers to the wavelet at orientation $\theta$ at the $j$-th scale.

\end{itemize}
