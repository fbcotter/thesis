\pagestyle{empty}

\def\layersep{2.5cm}

\begin{tikzpicture}[]
    % \tikzstyle{every pin edge}=[<-,shorten <=1pt]
    \tikzstyle{every pin edge}=[signal]
    % \tikzstyle{neuron}=[circle,draw=black,minimum size=17pt,inner sep=0pt]
    % \tikzstyle{input neuron}=[neuron];
    % \tikzstyle{output neuron}=[neuron, fill=red!50];
    % \tikzstyle{hidden neuron}=[neuron];
    % \tikzstyle{annot} = [text width=4em, text centered]

    % Draw the input layer nodes
    \foreach \name / \y in {1,...,5}
    % This is the same as writing \foreach \name / \y in {1/1,2/2,3/3,4/4}
        \node[inputnode,pin={[pin edge={latex-}]left:Input \y}] %left:Input \y] 
            (I-\name) at (0,1-\y) {$x_\y$};
    \node[inputnode] (I-0) at (0,-5) {1};

    % Draw the output layer node
    \node[outputnode,pin={[pin edge={latex}]right:Output}] (O) at (\layersep, -2.5) {y};

    % Connect every node in the input layer with every node in the
    % hidden layer.
    % \foreach \source in {1,...,5}
        % \draw(I-\source) edge (O) node[above, midway] {$w_\source$};
    \foreach \source in {1,...,5}
        \draw[signal] (I-\source) -- (O) node[above,midway] {$w_\source$};
    \draw[signal] (I-0) -- (O) node[above,midway] {$b$};

    % Annotate the layers
    % \node[annot,above of=I-1, node distance=1cm] (il) {Input Layer};
    % \node[annot,right of=hl] {Output layer};
\end{tikzpicture}
% End of code
