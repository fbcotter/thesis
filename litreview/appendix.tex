% \chapter{Shift Invariance of the $\DTCWT$}
% \begin{figure}
  % \includegraphics[width=\textwidth]{\imgpath/dtcwt.png}
  % \mycaption{Full 1-D $\DTCWT$}{}
% \end{figure}
% \begin{figure}
  % \centering
  % \begin{tikzpicture}
    \matrix (m1) [minimum height=4mm, column sep=6mm, align=center]
	{
	%--------------------------------------------------------------------
		\node[coordinate]                  (m00) {};    &
		\node[coordinate]                  (m01) {};          &
		\node[dspsquare]                   (m02) {$A_a(z)$};          &
		\node[circle,draw,inner sep=1pt]   (m03) {\downsamplertext{M}}; &
		\node[dspnodeopen,dsp/label=above] (m04) {$X_a(z)$};          &
		\node[circle,draw,inner sep=1pt]   (m07) {\upsamplertext{M}}; &
		\node[dspsquare]                   (m08) {$C(z)$};          &
		\node[coordinate]                  (m09) {};          &
		\node[coordinate]                  (m0X) {};          \\
		%--------------------------------------------------------------------
		\node[dspnodefull]                 (m10) {$X(z)$};          &
		\node[coordinate]                  (m11) {};          &
		\node[coordinate]                  (m12) {};    &
		\node[coordinate]                  (m13) {};          &
		\node[coordinate]                  (m14) {};    &
		\node[coordinate]                  (m17) {};          &
		\node[coordinate]                  (m18) {};    &
		\node[dspadder]                    (m19) {};          &
		\node[]     (m1X) {};          \\
		%--------------------------------------------------------------------
		\node[coordinate]                  (m20) {};    &
		\node[coordinate]                  (m21) {};          &
		\node[dspsquare]                   (m22) {$B(z)$};          &
		\node[circle,draw,inner sep=1pt]   (m23) {\downsamplertext{M}}; &
		\node[dspnodeopen,dsp/label=below] (m24) {$X_b(z)$};          &
		\node[circle,draw,inner sep=1pt]   (m27) {\upsamplertext{M}}; &
		\node[dspsquare]                   (m28) {$D(z)$};          &
		\node[coordinate]                  (m29) {};          &
		\node[coordinate]                  (m2X) {};          \\
		%--------------------------------------------------------------------
	};
	\draw[dspline] (m10) -- (m11);
	\draw[dspline] (m11) -- (m01);
	\draw[dspline] (m11) -- (m21);
	\foreach \i in {0,2} {
    	\draw[dspconn] (m\i1) -- (m\i2);
    	\draw[dspconn] (m\i2) -- (m\i3);
    	\draw[dspline] (m\i3) -- (m\i4);
    	\draw[dspconn] (m\i4) -- (m\i7);
    	\draw[dspconn] (m\i7) -- (m\i8);
    	\draw[dspline] (m\i8) -- (m\i9);
	}
  \draw[dspconn] (m09) -- node[right] {$Y_a'(z)$} (m19);
  \draw[dspconn] (m29) -- node[right] {$Y_b'(z)$} (m19);
	\draw[dspconn] (m19) -- (m1X);
	
\end{tikzpicture}

  % \mycaption{Block Diagram of 1-D $\DTCWT$}{Note the top and bottom paths are
  % through the wavelet or scaling functions from just level m ($M=2^m$). Figure
  % based on Figure~4 in \cite{kingsbury_complex_2001}.}
  % \label{fig:ch2:dtcwt_two_tree}
% \end{figure}
% Firstly, let us look at what would happen if we retained only one of the subbands.
% Note that we have to keep the same band from each tree. For any pair of
% coefficients on the tree, this would look like \autoref{fig:ch2:dtcwt_two_tree}.
% E.g.\ if we kept $x_{001a}$ and $x_{001b}$ then $M=8$ and $A(z) =
% H_{0a}(z)H_{00a}(z^2)H_{001a}(z^4)$ is the transfer fucntion from $x$ to
% $x_{001a}$. The transfer functions for $B(z)$, $C(z)$ and $D(z)$ are obtained
% similarly. It is well known that:
% \begin{align}
  % U(z) \downarrow M &\rightarrow  U(z^M) \\
  % U(z) \uparrow M &\rightarrow  \frac{1}{M}\sum_{k=0}^{M-1}U(W^kz^{1/M})
% \end{align}
% Where $W=e^{j2\pi/M}$. So downsampling followed by upsampling becomes:
% \begin{equation}
% U(z) \downarrow M \uparrow M \rightarrow \frac{1}{M}\sum_{k=0}^{M-1}U(W^kz)
% \end{equation}
% This means that
% %
% \begin{equation}
  % Y(z) = Y_{a}(z) + Y_{b}(z) = \frac{1}{M} \sum_{k=0}^{M-1} X(W^k z) [A(W^kz)C(z) + B(W^kz)D(z)]
% \end{equation}
% %
% The aliasing terms for which are everywhere where $k \neq 0$ (as $X(W^kz)$ is
% $X(z)$ shifted by $\frac{2k\pi}{M}$). I.e. to avoid aliasing in this reduced
% tree, we want to make sure that $A(W^kz)C(z) + B(W^kz)D(z) = 0$ for all $k \neq
% 0$.

% The figure below (Fig 5 from \cite{kingsbury_complex_2001}) shows what $A(W^kz)$
% and $C(z)$ look like for both the lowpass case (left) and the highpass case
% (right). Note that these responses will be similar for $B(W^kz)$ and $D(z)$. For
% large values of k, there is almost no overlap (i.e. 
% $A(W^kz)C(z) \approx B(W^kz)D(z) \approx 0$, 
% but for small values of k (particularly $k = \pm 1$),
% the transition bands have significant overlap with the central response. It is
% here that we need to use the flexibility of having 2 trees to ensure that
% $A(W^kz)C(z)$ and $B(W^kz)D(z)$ cancel out.
% \begin{figure}
  % \centering
  % \includegraphics[width=\textwidth]{\imgpath/overlaps.png}
% \end{figure}
% To do this, let us consider the lowpass filters first. If we let:
% \begin{align}
  % B(z) &= z^{\pm M/2}A(z) \\
  % D(z) &= z^{\mp M/2}C(z)
% \end{align}
% Then
% \begin{align}
  % A(W^kz)C(z) + B(W^kz)D(z) &= A(W^kz)C(z) + (W^kz)^{\pm M/2}A(W^kz) z^{\mp M/2}C(z) \\
  % =& A(W^kz)C(z) + e^{\frac{jk2\pi}{M} \times (\pm \frac{M}{2})} z^{\pm M/2} z^{\mp M/2} A(W^kz)C(z) \\
  % =& A(W^kz)C(z) + (-1)^k A(W^kz)C(z)
% \end{align}
% which cancel when k is odd.

% Now consider the bandpass case. For shifts of $k=1$ the right half of the left
% peak overlaps with the left half of the right peak. For a shift of $k=2$, the
% left half of the left peak overlaps with the right half of the right peak.
% Similarly for $k=-1$ and $k=-2$. For $|k| > 2$, there is no overlap. The fact
% that we have overlaps at both even and odd shifts of $k$ means that we can't use
% the same clever trick from before. However, what we do note is that the overlap
% is always caused by opposite peaks (i.e. the left with the right peak, and never
% the left with itself, or the right with itself). The solution then is to have
% $B$ and $D$ have upper and lower passbands of opposite polarity, whil $A$ and
% $C$ should have passbands of the same polarity.

% Consider two prototype complex filters $P(z)$ and $Q(z)$ each with single
% passbands going from $f_s/2M \rightarrow f_s/M$ (or $\frac{\pi}{M} \rightarrow
% \frac{2*\pi}{M}$) - they must be complex to have support in only one half of the
% frequency plane. Now say $P^*(z) = \sum_{r}p_r^*z^{-r}$ is the z-transform of
% the conjugate of $p_r$, which has support only in the negative half of the
% frequency plane. Then we can get the required filters by:
% \begin{align}
% A(z) &= 2\Re [P(z)] = P(z) + P^*(z) \\
% B(z) &= 2\Im [P(z)] = -j[P(z) - P^*(z)] \\
% C(z) &= 2\Re [Q(z)] = Q(z) + Q^*(z) \\
% D(z) &= -2\Im [Q(z)] = j[Q(z) - Q^*(z)]
% \end{align}
% Then:
% \begin{align}
  % A(W^kz)C(z) + B(W^kz)D(z) = &  \left[P(W^kz) + P^*(W^kz)\right]\left[Q(z) + Q^*(z)\right] + \nonumber \\
                            % & (-j*j)\left[P(W^kz) - P^*(W^kz)\right]\left[Q(z) - Q^*(z)\right] \\
                          % = & P(W^kz)Q(z)[1+1] + P^*(W^kz)Q(z)[1-1] + \nonumber \\ 
                            % & P(W^kz)Q^*(z)[1-1] + P^*(W^kz)Q^*(z)[1+1] \\
                          % = & 2P(W^kz)Q(z) + 2P^*(W^kz)Q^*(z)
% \end{align}

% So now we only need to ensure that $P(W^kz)$ overlaps as little as possible with
% $Q(z)$. This is somewhat more manageable, the diagram below shows the problem.

% \begin{figure}
  % \includegraphics[width=\textwidth]{\imgpath/overlaps_complex.png}
% \end{figure}

