\section{Neural Networks}\label{sec:cnns}
\subsection{The Neuron and Single Layer Neural Networks}
Single hidden layer neural networks are universal function approximators

\subsection{Backpropagation}
  With a deep network like most CNNs, calculating $\dydx{L}{w}$ may not seem
  particularly obvious if $w$ is a weight in one of the lower layers.  Say we
  have a deep network, with $L$ layers. We need to define a rule for updating
  the weights in all $L$ layers of the network, however, only the weights $w_L$
  are connected to the loss function, $\loss$. We assume for whatever function
  the last layer is that we can write down the derivative of the output with
  respect to the weights $\dydx{z_{L}}{w_{L}}$. We can then write down the
  weight-update gradient, $\dydx{\loss}{w_L}$ with application of the chain rule:
  \begin{equation}
    \dydx{\loss}{w_L} = \dydx{\loss}{z_L} \dydx{z_L}{w_L} + \underbrace{\lambda
    w}_{\text{from the reg.\ loss}}
  \end{equation}
  $\dydx{\loss}{z_L}$ can be done simply from the equation of the loss function
  used. Typically this is parameterless.

  Since all of the layers in a CNN are well-defined and
  differentiable\footnote{The ReLU is not differentiable at its corner, but
  backpropagation can still be done easily by simply looking at the sign of the
  input.} we assume that we can also write down what $\dydx{z_L}{z_{L-1}}$ is.
  Repeating this process for the next layer down, we have:
  \begin{equation}
    \dydx{\loss}{w_{L-1}}
    = \dydx{\loss}{z_L}\dydx{z_L}{z_{L-1}}\dydx{z_{L-1}}{w_{L-1}}
  \end{equation}
  We can generalize this easily like so:
  \begin{equation}
    \dydx{\loss}{w_{l}}
    = \dydx{\loss}{z_L} \underbrace{\prod_{i=L}^{l+1}
    \dydx{z_i}{z_{i-1}}}_{\text{product to $l$'s output}} 
    \dydx{z_{l}}{w_{l}}
  \end{equation}

  So far we have defined the loss function for a given data point $(x^{(i)},
  y^{(i)})$. Typically, we want our network to be able to generalize to the
  true real world joint distribution $P(x,y)$, minimizing the expected risk ($R_E$) of
  loss:
  \begin{equation}
    R_E(f(x,w)) = \int \loss(y,f(x,w)) dP(x,y)
  \end{equation}
  Instead, we are limited to the training set, so we must settle for the
  empirical risk ($R_{EMP}$):
  \begin{equation}
    R_{EMP}(f(x,w)) = \frac{1}{N} \sum_{i=1}^{N} \loss( y^{(i)},f(x^{(i)}, w))
    \label{eq:empirical_risk}
  \end{equation}

\subsection{Gradient descent vs Stochastic Gradient Descent vs Mini-Batches}
  We can minimize \autoref{eq:empirical_risk} with \emph{gradient descent}
  \citep{rumelhart_parallel_1986}. Updates can be made on a generic network
  parameter $w$ with:
  \begin{equation}
    w_{t+1} = w_{t} - \eta \dydx{E_n}{w}
  \end{equation}
  where $\eta$ is caleld the learning rate. Calculating the gradient
  $\dydx{E_n}{w}$ is done by averaging the individual gradients
  $\dydx{\loss}{w}$ over the entire training dataset. This can be very slow,
  particularly for large training sets.
  
  Instead, we can learn far more quickly by using a single estimate for the
  weight update equation, i.e., 
  \begin{equation}
    w_{t+1} = w_{t} - \eta \dydx{\loss}{w}
  \end{equation}
  This is called \emph{stochastic gradient descent}. Each weight update now
  uses a noisy estimate of the true gradient $\dydx{E_n}{w}$. Carefully
  choosing the learning rate $\eta$ update scheme can ensure that the network
  converges to a local minimum, but the process may not be smooth (the
  empirical risk may fluctuate, which could be interpreted as the network
  diverging).

  An often used trade off between these two schemes is called \emph{mini-batch
  gradient descent}. Here, the variance of the estimate of $\dydx{E_n}{w}$ is
  reduced by averaging out the point estimate $\dydx{L}{w}$ over a mini-batch
  of samples, size $N_B$. Typically $1 << N_B << N$, with $N_B$ usually being
  around 128. This number gives a clue to another benefit that has seen the use
  of mini-batches become standard --- they can make use of parallel processing.
  Instead of having to wait until the gradient from the previous data point was
  calculated and the network weights are updated, a network can now process
  $N_B$ samples in parallel, calculating gradients for each point with the same
  weights, average all these gradients in one quick step, update the weights,
  and continue on with the next $N_B$ samples. The update equation is now:
  \begin{equation}
    w_{t+1} = w_{t} - \eta \sum_{n=1}^{N_B} \dydx{\loss}{w}
  \end{equation}

\subsubsection{Loss Functions}
  For a given sample $q=(x, y)$, the loss
  function is used to measure the cost of predicting $\hat{y}$ when the
  true label was $y$. We define a loss function $\loss (y, \hat{y})$. 
  A CNN is a deterministic
  function of its weights and inputs, $f(x,w)$ so this can be written as $\loss
  (y, f(x,w))$, or simply $\loss(y, x, w)$.
  
  It is important to remember that we can choose
  to penalize errors however we please, although for CNNs, the vast majority of
  networks use the same loss function --- the softmax loss. Some networks have
  experimented with using the hinge loss function (or `SVM loss'), stating they
  could achieve improved results \citep{gu_recent_2015,tang_deep_2013}. 
  \begin{enumerate}

  \item \emph{Softmax Loss:} The more common of the two, the softmax turns
    predictions into non-negative, unit summing values, giving the sense of
    outputting a probability distribution. The softmax function is applied to
    the $C$ outputs of the network (one for each class):
    \begin{equation}
      p_j = \frac{e^{(f(x, w))(j)}}{\sum\limits_{k=1}^{C}e^{(f(x,w))(k)}}
    \end{equation}
    where we have indexed the $c$-th element of the output vector $f(x,w)$ with
      $(f(x,w))(c)$. The softmax \emph{loss} is then defined as:
    \begin{equation}
      \loss (y_i, x, w)
      = \sum_{j=1}^{C} \mathbbm{1}\{y_i=j\} \log p_j
    \end{equation}
    Where $\mathbbm{1}$ is the indicator function, and $y_i$ is the true label
    for input $i$.

  \item \emph{Hinge Loss:} The same loss function from Support Vector
    Machines (SVMs) can be used to train a large margin classifier in a CNN:
    \begin{equation}
      \loss(y,x,w) = \sum_{l=1}^{C} 
        {\left[ \max(0, 1 - \delta(y_i,l) w^{T}x_{i}) \right]}^p \label{eq:nn_hinge_loss}
    \end{equation}
    Using a hinge loss like this introduces extra parameters, which would
    typically replace the final fully connected layer. The $p$ parameter in
    \autoref{eq:nn_hinge_loss} can be used to choose $\ell^1$ Hinge-Loss, or
    $\ell^2$ Squared Hinge-Loss. 

  \end{enumerate}
  

\subsubsection{Regularization}
  Weight regularization, such as an $\ell^2$ penalty is often given to the
  learned parameters of a system. This applies to the parameters of the fully
  connected, as well as the convolutional layers in a CNN\@. These are added to
  the loss function. Often the two loss components are differentiated between by
  their monikers - `data loss' and `regularization loss'. The above 
  equation then becomes:
  \begin{equation}
    \mathcal{L} = \mathcal{L}_{data} + \underbrace{\frac{1}{2}\lambda_{fc}
    \sum_{i=1}^{L_{fc}} \sum_{j} \sum_{k} {(w_{i}[j,k])}^2}_{\text{fully connected
    loss}} +
    \underbrace{\frac{1}{2}\lambda_{c} \sum_{i=1}^{L_c} \sum_{u_1} \sum_{u_2} \sum_{d}
    \sum_{n} \left(f_i[u_1,u_2,d,n]\right)^2}_{\text{convolutional loss}}
  \end{equation}

  Where $\lambda_{fc}$ and $\lambda_{c}$ control the regularization parameters
  for the network. These are often also called `weight decay' parameters.

  The choice to split the $\lambda$'s between fully connected and convolutional
  layers was relatively arbitrary. More advanced networks can make $\lambda$
  a function of the layer. 


\subsection{Learning}

\subsection{Optimization}

\subsection{Extending to multiple layers and different block types}
\begin{figure}
  \centering
  \begin{tikzpicture}[every node/.style={node font=\large}]
\definecolor{mycolor}{RGB}{252,247,189};
\pgfmathsetmacro{\EDGE}{.5}
\node [fill=mycolor, draw, minimum width=3cm, minimum height=2cm] (block) at (1,1) {$y=f(x,w)$};
\coordinate[right=\EDGE of block.south west] (xin);
\coordinate[left=\EDGE of block.south east] (dxin);
\coordinate[below=\EDGE of block.north west] (win);
\coordinate[above=\EDGE of block.south west] (dwin);
\coordinate[right=\EDGE of block.north west] (yin);
\coordinate[left=\EDGE of block.north east] (dyin);
\draw[<-] (xin) -- +(0,-1) node[below] {$x$};
\draw[->] (dxin) -- ++(0,-1) node[below] {$\dydx{\logloss}{x}$};
\draw[<-] (win) -- ++(-1,0) node[left] {$w$};
\draw[->] (dwin) -- ++(-1,0) node[left] {$\dydx{\logloss}{w}$};
\draw[->] (yin) -- ++(0,1) node[above] {$y$};
\draw[<-] (dyin) -- ++(0,1) node[above] {$\dydx{\logloss}{y}$};
\end{tikzpicture}

\end{figure}

\subsubsection{Convolutional Layers}
  The image/layer of features is convolved by a set of filters.
  The filters are typically small, ranging from $3\x 3$ in ResNet and VGG
  to $11\x 11$ in AlexNet. We have quoted only spatial size
  here, as the filters in a CNN are always \emph{fully connected in depth} ---
  i.e.,\ they will match the number of channels their input has.

  For an input $\bmu{x} \in \mathbb{R}^{H\x W\x D}$, and filters 
  $\bmu{f} \in \mathbb{R}^{H'\x W'  \x D \x D''}$ ($D''$ is the 
  number of filters), our output $\bmu{z} \in
  \mathbb{R}^{H''\x W'' \x D''}$ will be given by:
  \begin{equation}
    z[u_1, u_2, d''] = b[d''] + \sum_{i=-\frac{H'}{2}}^{\frac{H'}{2}-1}
                       \sum_{j=-\frac{W'}{2}}^{\frac{W'}{2}-1}  \sum_{k=0}^{D-1}  
                        f[i, j, k, d''] x[u_1-i, u_2-j, k]
  \end{equation}

\subsubsection{ReLUs}
  Activation functions, neurons, or non-linearities, are the core of a neural networks
  expressibility. Historically, they were sigmoid or tanh functions, but these
  have been replaced recently by the Rectified Linear Unit (ReLU), which has
  equation $g(x) = \max(0,x)$. A ReLU
  non-linearity has two main advantages over its smoother predecessors~\citep{%
  glorot_deep_2011, nair_rectified_2010}.
  \begin{enumerate}
  \item It is less sensitive to initial conditions as the gradients that
    backpropagate through it will be large even if $x$ is large. A common
    observation of sigmoid and tanh non-linearities was that their learning would
    be slow for quite some time until the neurons came out of saturation, and then
    their accuracy would increase rapidly before levelling out again at
    a minimum~\citep{glorot_understanding_2010}. The ReLU, on the other hand, has
    constant gradient.
  \item It promotes sparsity in outputs, by setting them to a hard 0. Studies
    on brain energy expenditure suggest that neurons encode information in
    a sparse manner. \citet{lennie_cost_2003} estimates the percentage of
    neurons active at the same time to be between 1 and 4\%. Sigmoid and tanh
    functions will typically have \emph{all} neurons firing, while 
    the ReLU can allow neurons to fully turn off.
  \end{enumerate}

  \begin{figure}
    \centering
      % \section{Wavelet Based Nonlinearities and Multiple Gain Layers}
So far we have only considered doing the mixing operations in the wavelet
domain. This is a good starting point, and it is nice to see that we can do
apply the gain layer that
preserve some of the nice properties of the $\DTCWT$. Let us call the layer
considered so far a \textbf{first order gain layer}. Recall this is where a
$\DTCWT$ is done, followed by a single linear convolution and then straight away
returning to the pixel domain with the inverse $\DTCWT$. While it is good to see
that it is possible to do and even achieves some benefit over a convolutional
layer, the proposed layer \eqref{eq:ch6:end2end} can be implemented in the pixel domain
as a single convolution.

It would be particularly interesting if we could find a sensible nonlinearity to
do in the wavelet domain. This would mean it would be no longer possibile to do the
gain layer in the pixel domain. Further, we could then do multiple mixing
stages in the wavelet domain before returning to the pixel domain.

But what sensible nonlinearity to use? Two particular options are good initial
candidates:
\begin{enumerate}
  \item The ReLU: this is a mainstay of most modern neural networks and has
    proved invaluable in the pixel domain. Perhaps its sparsifying properties
    will work well on wavelet coefficients too. 
  \item Soft thresholding: a technique commonly applied to wavelet
    coefficients for denoising and compression. Many proponents of compressed
    sensing and dictionary learning even like to compare soft thresholding to a
    two sided ReLU \cite{papyan_theoretical_2018, papyan_convolutional_2016}.
\end{enumerate}

In this section we will look at each, see if they add to the first order gain
layer, and see if they open the possibility of having multiple layers in the
wavelet domain. 

\subsection{ReLUs in the Wavelet Domain}
A ReLU could be applied to the real lowpass coefficients with ease, but it does
not generalize so easily to complex coefficients. One option could be to apply
it independently to the real and imaginary coefficients, effectively only
selecting one quadrant of the complex plane.

One potential problem with this is
that applying a ReLU independently to the real and imaginary components


      \caption[Differences in non-linearities]
              {Differences in non-linearities. Green --- the \emph{sigmoid} function, 
               Blue --- the \emph{tanh} function, and Red --- the \emph{ReLU}. The ReLU
               solves the problem of small gradients outside of the activation
               region (around $x=0$) as well as promoting sparsity.}\label{fig:nonlinearities}
  \end{figure}


\subsubsection{Pooling}
  Typically following a convolutional layer (but not strictly), activations are subsampled with
  max pooling. Pooling adds some invariance to shifts smaller than the pooling
  size at the cost of information loss. For this reason, small pooling is
  typically done often $2\x 2$ or $3\x 3$, and the invariance to larger shifts
  comes after multiple pooling (and convolutional) layers.
  
  While initial designs of max pooling would do it in non-overlapping regions, 
  AlexNet used $3\x 3$ pooling with stride 2 in their breakthrough design,
  quoting that it gave them an increase in accuracy of roughly $0.5\%$ and
  helped prevent their network from `overfitting'. More recent networks will
  typically employ either this or the original $2\x 2$ pooling with stride 2,
  see \autoref{fig:maxpool}. A review of pooling methods in
  \citep{mishkin_systematic_2016} found them both to perform equally well.
  
  \begin{figure}
    \centering
    % \subfloat[]{%
        % \input{tikz/maxpool.tex}\label{fig:maxpool_tight}
    % }
%    \newline
    \centering
    % \subfloat[]{%
        % \input{tikz/maxpool_overlap.tex}\label{fig:maxpool_overlap}
    % }
    \caption[Tight vs.\ overlapping pooling]
            {\subref{fig:maxpool_tight} Tight $2\x 2$ pooling with stride 2, vs
            \subref{fig:maxpool_overlap} overlapping $3\x 3$ pooling with
            stride 2. Overlapping pooling has the possibility of having one
            large activation copied to two positions in the reduced size
            feature map, which places more emphasis on the odd columns.}
    \label{fig:maxpool}
  \end{figure}

\subsubsection{Batch Normalization}
      Batch normalization proposed only very recently in
      \citep{ioffe_batch_2015} is a conceptually simpler technique. Despite
      that, it has become quite popular and has been found to be very useful.
      At its core, it is doing what standard normalization is doing, but also
      introduces two learnable parameters --- scale ($\gamma$) and offset
      ($\beta$).
      \autoref{eq:normalization} becomes:
      \begin{equation}
        \tilde{z}(u_1,u_2,d) = \gamma\frac{z-E[z]}{\sqrt{Var[z]}} + \beta 
				\label{eq:batch_normalization}
      \end{equation}
      These added parameters make it a \emph{renormalization} scheme, as instead of
      centring the data around zero with unit variance, it can be centred
      around an arbitrary value with arbitrary variance. S etting
      $\gamma = \sqrt{Var[z]}$ and $\beta = E[z]$, we would get the identity
      transform. Alternatively, setting $\gamma = 1$ and $\beta = 0$ (the
      initial conditions for these learnable parameters), we get standard
      normalization.
      
      The parameters $\gamma$ and $\beta$ are learned through backpropagation.
      As data are usually processed in batches, the gradients for $\gamma$ and
      $\beta$ are calculated per sample, and then averaged over the whole
      batch.

      From \autoref{eq:batch_normalization}, let us briefly use the hat
      notation to represent the standard normalized input:
      $\hat{z} = (z-E[z])/\sqrt{Var[z]}$, then:
      \begin{eqnarray}
        \tilde{z}^{(i)} & = & \gamma \hat{z}^{(i)} + \beta \nonumber\\
        \frac{\partial \mathcal{L}}{\partial \gamma}& =&
        \frac{1}{N}\sum_{i=1}^{N} \frac{\partial
        \mathcal{L}^{(i)}}{\partial \tilde{z}^{(i)}} \cdot \hat{z}^{(i)} \\
        \frac{\partial \mathcal{L}}{\partial \beta}& =&
        \frac{1}{N}\sum_{i=1}^{N} \frac{\partial
        \mathcal{L}^{(i)}}{\partial \tilde{z}^{(i)}} 
      \end{eqnarray}

      Batch normalization layers are typically placed \emph{between} convolutional layers
      and non-linearities. I.e.,\ if $Wu+b$ is the output of a convolutional
      layer, and $z=g(Wu+b)$ is the output of the non-linearity, then with the
      batch normalization step, we have:
      \begin{eqnarray}
        z &=& g(BN(Wu+b)) \nonumber\\
          &=& g(BN(Wu))
      \end{eqnarray}
      Where the bias term was ignored in the convolutional layer, as it can be
      fully merged with the `offset' parameter $\beta$.

      This has particular benefit of removing the sensitivity of our network to
      our initial weight scale, as for scalar $a$,
      \begin{equation}
        BN(Wu) = BN((aW)u)
      \end{equation}
      It is also particularly useful for backpropagation, as an increase in
      weights leads to \emph{smaller} gradients \citep{ioffe_batch_2015}, making
      the network far more resilient to the problems of vanishing and exploding
      gradients:
      \begin{eqnarray}
        \frac{\partial BN((aW)u)}{\partial u} & = & \frac{\partial
        BN(Wu)}{\partial u} \nonumber\\
        \frac{\partial BN((aW)u)}{\partial (aW)} & = & \frac{1}{a} \cdot \frac{\partial
        BN(Wu)}{\partial W} 
      \end{eqnarray}


\section{Relevant Architectures}

\subsection{LeNet}
\subsection{AlexNet}
\subsection{VGG}
\subsection{Residual Networks}
  The current state of the art design introduced a clever novel feature called
  a residual unit\citep{he_deep_2015,he_identity_2016}. The inspiration for their design came from the difficulties
  experienced in training deeper networks. Often, adding an extra layer would
  \emph{decrease} network performance. This is counter-intuitive as the deeper
  layers could simply learn the identity mapping, and achieve the same
  performance.

  To promote the chance of learning the identity mapping, they define
  a residual unit, shown in \autoref{fig:residual_unit}. If a desired mapping
  is denoted $\mathcal{H}(x)$, instead of trying to learn this, they instead
  learn $\mathcal{F}(x) = \mathcal{H}(x) - x$. 
  \begin{figure}
    \centering
    % \includegraphics[width=0.5\textwidth]{images/residual_unit.png}
    \caption[The residual unit from ResNet]
          {A residual unit. The identity mapping is always present, and the
            network learns the difference from the identity mapping, $\mathcal{F}(x)$.
            Taken from \citep{he_deep_2015}.}
      \label{fig:residual_unit}
  \end{figure}


\subsection{old}
  Convolutional Neural Networks (CNNs) were initially introduced by \citet{lecun_backpropagation_1989} in
  \citep{lecun_backpropagation_1989}. Due to the difficulty of training and
  initializing them, they failed to be popular for more than two decades.  This
  changed in 2012, when advancements in pre-training with unsupervised networks
  \citep{bengio_greedy_2007}, the use of an improved non-linearity --- the Rectified Linear
  Unit, or ReLU, new regularization methods\citep{hinton_improving_2012}, and
  access to more powerful computers in graphics cards, or GPUs, allowed
  Krizhevsky, Sutskever and Hinton  to develop
  AlexNet\citep{krizhevsky_imagenet_2012}. This network nearly halved the
  previous state of the art's error rate.  Since then, interest in them has
  expanded very rapidly, and they have been successfully applied to object
  detection~\citep{ren_object_2015} and human pose estimation
  \citep{tompson_efficient_2015}. It would take a considerable amount of effort
  to document the details of all the current enhancements and tricks many
  researches are using to squeeze extra accuracy, so for the purposes of this
  report we restrict ourselves to their generic design, with some time spent
  describing some of the more promising enhancements. 
  
  We would like to make note of some of the key architectures
  in the history of CNNs, which we, unfortunately, do not have space to describe:
  \begin{itemize}
    \item Yann LeCun's LeNet-5~\citep{lecun_gradient-based_1998}, the state of the art
      design for postal digit recognition on the MNIST dataset.
    \item Google's GoogLeNet~\citep{szegedy_going_2015} achieved $6.67\%$ top-5
      error on ILSVRC2014, introducing the new `inception' architecture, which
      uses combinations of $1\x 1$, $3\x 3$ and $5\x 5$ convolutions.
    \item Oxford's VGG~\citep{simonyan_very_2014} --- $6.8\%$ and runner up in
      ILSVRC2014. The VGG design is very similar to AlexNet but was roughly
      twice as deep. More convolutional layers were used, but with smaller
      support --- only $3\x 3$. These were often stacked directly on top of
      each other without a non-linearity in between, to give the effective
      support of a $5\x 5$ filter.
    \item Microsoft Research's ResNet~\citep{he_deep_2015} achieved $4.5\%$ top-5 
      error and was the winner of ILSVRC2015. This network we will talk briefly
      about, as it introduced a very nice novel layer --- the residual layer.
  \end{itemize}

  Despite the many variations of CNN architectures currently being used, most
  follow roughly the same recipe (shown in \autoref{fig:cnn_generic}):
  \begin{figure}
    \centering
      % \includegraphics[width=\textwidth]{images/cnns.png}
      \caption[Standard CNN architecture]
              {Standard CNN architecture. Taken
              from~\citep{lecun_gradient-based_1998}}\label{fig:cnn_generic}
  \end{figure}

\subsection{Fully Connected Layers}\label{sec:cnn_fullyconnected}
  The convolution, pooling, and activation layers all
  conceptually form part of the \emph{feature extraction} stage of a CNN. One
  or more fully connected layers are usually placed after these layers to form
  the \emph{classifier}. One of the most elegant and indeed most powerful
  features of CNNs is this seamless connection between the \emph{feature
  extraction} and \emph{classifier} sub-networks, allowing the backpropagation
  of gradients through all layers of the entire network.

  The fully connected layers in a CNN are the same as those in a classical
  Neural Network (NN), in that they compute a dot product between their input
  vector and a weight vector:
  \begin{equation}
    z_i = \sum_{j} W_{ij}x_j
  \end{equation}
  The final output of the Fully Connected layer typically has the same number
  of outputs as the number of classes $C$ in the classification problem.

