\section{Conclusion}
This chapter presents a way to investigate what the higher orders of a ScatterNet
are responding to - the DeScatterNet described in \autoref{sec:ch4:descatternet}.
Using this, we have shown that the second-order of a ScatterNet
responds strongly to patterns that are very different to those that highly activate the
second layer of a CNN\@. As well as being dissimilar to CNNs, visual inspection of the
ScatterNet's patterns reveal that they may be less useful for discriminative
tasks, and we believe this may be causing the current gaps in state-of-the-art
performance between the two.

Additionally, we performed occlusion tests to heuristically measure the
importance of the individual scattering channels when a ScatterNet is used as a
front-end to a CNN. The results of this test
reaffirmed the suspicions raised from the visualizations. In particular, many of
the second-order Scattering coefficients may not be very useful in a deep
classifier. Those that were more useful were typically when the second-order wavelet
had the same orientation as the first. We also noted that diagonal
orientations appear less important than horizontal and vertical ones, even at
coarser scales. This appears to be an artefact of the CIFAR and Tiny ImageNet
datasets, as rotating the images by $30\degs$ made diagonal edges more
important.

Finally, we demonstrated the possible shapes attainable when we filter
across orientations with complex mixing coefficients. We believe that this
mixing is a key step in the development of improved ScatterNets and wavelets in deep
learning systems.

