\chapter{Introduction}\label{ch:intro}

\section{Gradients of Complex Activations}
\subsection{Cauchy-Riemann Equations}
None of these will be solved. However, we can still find the partial derivatives \wrt the real and
imaginary parts.

\subsection{Complex Magntitude}
Care must be taken when calculating gradients for the complex magnitude, as the
gradient is undefined at the origin. We take the common approach of setting the
gradient at the corner point to be 0. 

Let us call the real and imaginary inputs to a magnitude block $x$ and $y$,
and we define the real output $r$ as:

$$ r = |x + jy| = \sqrt{x^2 + y^2} $$

Then the partial derivatives \wrt the real and imaginary inputs are:
\begin{eqnarray*}
  \dydx{r}{x} & = & \frac{x}{\sqrt{x^2 + y^2}} = \frac{x}{r} \\
  \dydx{r}{y} & = & \frac{y}{\sqrt{x^2 + y^2}} = \frac{y}{r} \\
\end{eqnarray*}

Except for the singularity at the origin, these partial derivatives are restricted to be in the
range $[-1, 1]$. The complex magnitude is convex in $x$ and $y$ as:

$$\nabla^2 r(x,y) = \frac{1}{r^3} 
\begin{bmatrix}
  y^2 & -xy \\
  -xy & x^2
\end{bmatrix}
 = \frac{1}{r^3} \begin{bmatrix} y \\ -x \end{bmatrix} 
 \begin{bmatrix} y & -x \end{bmatrix} \geq 0 
$$
These partial derivatives are very variable around 0. \textbf{Show a plot of this}. We can smooth it out by
adding a smoothing term:

$$ r_s = \sqrt{x^2 + y^2 + b} - \sqrt{b} $$

This keeps the magnitude zero for small $x,y$ but does slightly shrink larger values. The gain we
get however is a new smoother gradient surface \textbf{Plot this}.
