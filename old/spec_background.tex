\chapter{Wavelets and Scatternets}\label{ch:spec_background}
Scatternets get their own chapter as they have been a very large influence on
our work, as well as being quite distinct from the previous discussions on
learned methods. They were first introduced by Bruna and Mallat in
\cite{bruna_invariant_2013}. The particularly nice feature of scatternets is
they are mathematically (fairly) well understood, and give a basis for
justifying the use of convolutions to build features \emph{cite here}. 

The overall architecture involves cascading complex wavelet transforms, with
a modulus non-linearity in between layers. We first set the properties of
the transform that drive our decision choices (in particular, what
\emph{invariances} we hope to achieve), then introduce the wavelet
transform that form the core of the scattering transform, and examine the
effect of the modulus non-linearities that allow us to cascade multiple wavelet
transforms and how they aide getting second order features. And finally explore the
results obtained with the simple classifiers that Mallat uses.

\section{Invariances}
Mallat chooses to have invariance to shifts, scale and (later on) rotation. 

\section{Wavelet Transform}
Should this be a whole section to a chapter? There are so many things I want to
talk about with these. In particular in the case of the \DTCWT, the Hilbert
symmetry property, the simple frequency space work of the wavelet transform,
a comment on how to efficiently do wavelet transforms - \DTCWT, shift
invariance properties of hilbert pairs. What the wavelet bases look like. 

\section{Scattering Transform}
Good for stationary processes. Taking the average is a poor operator. It's
neither sufficient nor invariant.

