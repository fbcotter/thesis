\section{Fast Calculation of the DWT and IDWT}\label{sec:ch3:dwt}
\begin{figure}
  \centering
  \begin{tikzpicture}
    \matrix (m1) [minimum height=4mm, column sep=6mm, align=center]
	{
	%--------------------------------------------------------------------
		\node[coordinate]                  (m00) {};    &
		\node[coordinate]                  (m01) {};          &
		\node[dspsquare]                   (m02) {$A(z)$};          &
		\node[circle,draw,inner sep=1pt]   (m03) {\downsamplertext{M}}; &
		\node[dspnodeopen,dsp/label=above] (m04) {$X_a(z)$};          &
		\node[circle,draw,inner sep=1pt]   (m07) {\upsamplertext{M}}; &
		\node[dspsquare]                   (m08) {$C(z)$};          &
		\node[coordinate]                  (m09) {};          &
		\node[coordinate]                  (m0X) {};          \\
		%--------------------------------------------------------------------
		\node[dspnodefull]                 (m10) {$X(z)$};          &
		\node[coordinate]                  (m11) {};          &
		\node[coordinate]                  (m12) {};    &
		\node[coordinate]                  (m13) {};          &
		\node[coordinate]                  (m14) {};    &
		\node[coordinate]                  (m17) {};          &
		\node[coordinate]                  (m18) {};    &
		\node[dspadder]                    (m19) {};          &
		\node[]     (m1X) {};          \\
		%--------------------------------------------------------------------
		\node[coordinate]                  (m20) {};    &
		\node[coordinate]                  (m21) {};          &
		\node[dspsquare]                   (m22) {$B(z)$};          &
		\node[circle,draw,inner sep=1pt]   (m23) {\downsamplertext{M}}; &
		\node[dspnodeopen,dsp/label=below] (m24) {$X_b(z)$};          &
		\node[circle,draw,inner sep=1pt]   (m27) {\upsamplertext{M}}; &
		\node[dspsquare]                   (m28) {$D(z)$};          &
		\node[coordinate]                  (m29) {};          &
		\node[coordinate]                  (m2X) {};          \\
		%--------------------------------------------------------------------
	};
	\draw[dspline] (m10) -- (m11);
	\draw[dspline] (m11) -- (m01);
	\draw[dspline] (m11) -- (m21);
	\foreach \i in {0,2} {
    	\draw[dspconn] (m\i1) -- (m\i2);
    	\draw[dspconn] (m\i2) -- (m\i3);
    	\draw[dspline] (m\i3) -- (m\i4);
    	\draw[dspconn] (m\i4) -- (m\i7);
    	\draw[dspconn] (m\i7) -- (m\i8);
    	\draw[dspline] (m\i8) -- (m\i9);
	}
  \draw[dspconn] (m09) -- node[right] {$Y_a'(z)$} (m19);
  \draw[dspconn] (m29) -- node[right] {$Y_b'(z)$} (m19);
	\draw[dspconn] (m19) -- (m1X);
	
\end{tikzpicture}


  \mycaption{Block Diagram of 2-D DWT}{The components of a filter bank DWT in
  two dimensions.}
  \label{fig:ch3:dwt}
\end{figure}

To have a fast implementation of the Scattering Transform, we need a fast
implementation of the $\DTCWT$. For a fast implementation of the $\DTCWT$ we
need a fast implementation of the DWT. Later in our work will we also explore
the DWT as a basis for learning, so having an implementation that is fast and
can pass gradients through will prove beneficial in and of itself.

Writing a DWT in PyTorch lowlevel calls is not theoretically difficult to do.
There are only a few things to be wary of. Firstly, a `convolution' in most deep
learning packages is in fact a correlation. This does not make any difference
when learning but when using preset filters, as we want to do, it means that we
must take care to reverse the filters beforehand. Secondly, the automatic 
differentiation will naturally save activations after every step in the DWT
(e.g. after row filtering, downsampling and column filtering). This is for the
calculation of the backwards pass. We do not need to save these intermediate
activations and we can save a lot of memory by overwriting the automatic
differentiation logic and defining our own backwards pass.

\subsection{Primitives}
We start with the commonly known property that for a convolutional block, the
gradient with respect to the input is the gradient with respect to the output
convolved with the time reverse of the filter. More formally, if 
$Y(z) = H(z) X(z)$:
%
\begin{equation}\label{eq:ch3:backprop}
  \Delta X(z) = H(z^{-1}) \Delta Y(z)
\end{equation}
%
where $H(z^{-1})$ is the $Z$-transform of the time/space reverse of $H(z)$,
$\Delta Y(z) \triangleq \dydx{L}{Y}(z)$ is the gradient of the loss with respect
to the output, and $\Delta X(z) \triangleq \dydx{L}{X}(z)$ is the gradient of
the loss with respect to the input. 

Additionally, if we decimate by a factor of two on the forwards pass, the
equivalent backwards pass is interpolating by a factor of two (this is easy to
convince yourself with pen and paper).

\autoref{fig:ch3:dwt} shows the block diagram for performing the forward pass of
a DWT. Like matlab, PyTorch has an efficient function for doing convolution
followed by downsampling. Similarly, there is an efficient function to do
upsampling followed by convolution (for the inverse transform). Using the above
two properties for the backwards pass of convolution and sample rate
changes, we quickly see that the backwards pass of a wavelet transform is simply
the inverse wavelet transform with the time reverse of the analysis filters used
as the synthesis filters. For orthogonal wavelet transforms, the synthesis
filters are the time reverse of the analysis filters so backpropagation can
simply be done by calling the inverse wavelet transform on the wavelet
coefficient gradients.

\subsection{A brief description of autograd}

\subsection{The Forward and Backward Algorithms}
For clarity and repeatability, we give pseudocode for all the core operations
developed in \emph{PyTorch Wavelets}. By the end of this thesis, it should be
clear how every attempted method has been implemented.

Let us start by giving generic names to the above mentioned primitives. In
PyTorch, convolution followed by downsampling is done with a call to
\texttt{torch.nn.functional.conv2d} with the stride parameter set to 2. There are
similar such functions for other deep learning packages; let us refer to them
all as \texttt{conv2d_down}. As mentioned earlier, in all such packages, this
function's name is misleading as it in fact does correlation. As such we need to
be careful to reverse the filters before calling it (\texttt{flip}). Convolution
followed by upsampling can be done with a call to
\texttt{torch.nn.functional.conv_transpose2d} with the stride parameter set to
2; let us refer to this notional function as \texttt{conv2d_up}. Confusingly,
this function does in fact do true convolution, so we do not need to reverse any
filters. 

These functions in turn call the cuDNN lowlevel fucntions which can only support
zero padding. If another padding type is desired, it must be done beforehand
with a padding function \texttt{pad}.

\subsubsection{The Input}
In all the work in the following chapters, we would like to work on four
dimensional arrays. The first dimension represents a minibatch of $N$ images;
the second is the number of channels $C$ each image has. For a colour image,
$C=3$, but this often grows deeper in the network. Finally, the last two
dimensions are the spatial dimensions, of size $H\x W$. 

\subsubsection{1-D Filter Banks}
Let us assume that the analysis ($h_0,\ h_1$) and synthesis ($g_0,\ g_1$)
filters are already in the form needed to do column filtering. The necessary
steps to do the 1-D analysis and synthesis steps are described in
\autoref{alg:ch3:fb1d}. We do not need to define backpropagation functions for the
\texttt{afb1d} and \texttt{sfb1d} functions as they are each others backwards
step.

\begin{algorithm}[tb]
\caption{1-D analysis and synthesis stages of a DWT}\label{alg:ch3:fb1d}
\begin{algorithmic}[1]
\Function{afb1d}{$x,\ h_0,\ h_1,\ mode,\ axis$}
  % \mbox{\\$h_0,\ h_1$ are 1-D lowpass and highpass filters}
  \State $ h_0,\ h_1 \gets \F{flip}(h_0),\ \F{flip}(h_1) $\Comment{flip the filters for \texttt{conv2d_down}}
  \If{axis == -1} 
    \State $h_0,\ h_1 \gets h_0^t,\ h_1^t$  \Comment{row filtering}
  \EndIf
  \State $p \gets \lfloor (\F{len}(x) + \F{len}(h_0) - 1) / 2 \rfloor$ \Comment{calculate output size}
  \State $b \gets \lfloor p/2 \rfloor$ \Comment{calculate pad size before}
  \State $a \gets \lceil p/2 \rceil$ \Comment{calculate pad size after}
  \State $x \gets \F{pad}(x,\ b,\ a,\ mode)$ \Comment{pre pad the signal with selected mode} 
  \State $lo \gets \mathtt{conv2d\_down} (x, h_0)$
  \State $hi \gets \mathtt{conv2d\_down} (x, h_1)$
  \State \textbf{return} $lo,\ hi$
\EndFunction
\end{algorithmic}\vspace{10pt}
\begin{algorithmic}[1]
\Function{sfb1d}{$lo,\ hi,\ g_0,\ g_1,\ mode,\ axis$}
  % \mbox{\\$g_0,\ g_1$ are 1-D lowpass and highpass filters.}
  \If{axis == -1} 
    \State $g_0,\ g_1 \gets g_0^t,\ g_1^t$  \Comment{row filtering}
  \EndIf
  \State $p \gets \F{len}(g_0) - 2$ \Comment{calculate output size}
  \State $lo \gets \F{pad}(lo,\ p,\ p,\ ``zero")$ \Comment{pre pad the signal with zeros} 
  \State $hi \gets \F{pad}(hi,\ p,\ p,\ ``zero")$ \Comment{pre pad the signal with zeros} 
  \State $x \gets \mathtt{conv2d\_up}(lo,\ g_0) + \mathtt{conv2d\_up}(hi,\ g_1)$
  \State \textbf{return} $x$
\EndFunction
\end{algorithmic}
\end{algorithm}

\subsubsection{2-D Transforms and their gradients}
Having built the 1-D filter banks, we can easily generalize this to 2-D.
Furthermore we can now define the backwards steps of both the forward DWT
and the inverse DWT using these filter banks. We show how to do do this in 
\autoref{alg:ch3:dwt}. The inverse transform logic is moved to the appendix
\autoref{alg:ch3:idwt}. An interesting result is the similarity between the two 
transforms' forward and backward stages. Further, note that the only things that
need to be saved are the filters, as seen in
Algorithm~\algref{alg:ch3:dwt}{line:ch3:dwt_save}. These are typically only a
few floats, giving us a large saving over relying on autograd.

A multiscale DWT (and IDWT) can easily be made by calling \autoref{alg:ch3:dwt}
(\autoref{alg:ch3:idwt}) multiple times on the lowpass output (reconstructed
image). Again, no intermediate activations need be saved, giving this
implementation almost no memory overhead.

\begin{algorithm}[tb]
\caption{2-D DWT and its gradient}\label{alg:ch3:dwt}
\begin{algorithmic}[1]
\Function{DWT}{$x,\ h_0,\ h_1,\ mode$}
  \State \textbf{save} $h_0,\ h_1,\ mode$ \Comment{For the backwards pass} \label{line:ch3:dwt_save}
  \State $lo,\ hi \gets \F{afb1d}(x,\ h_0,\ h_1,\ mode,\ axis=-2)$ \Comment{column filter}
  \State $ll,\ lh \gets \F{afb1d}(lo,\ h_0,\ h_1,\ mode,\ axis=-1)$ \Comment{row filter}
  \State $hl,\ hh \gets \F{afb1d}(hi,\ h_0,\ h_1,\ mode,\ axis=-1)$ \Comment{row filter}
  \State \textbf{return} $ll,\ lh,\ hl,\ hh$
\EndFunction
\end{algorithmic}\vspace{10pt}
\begin{algorithmic}[1]
\Function{DWT\_backprop}{$\Delta ll,\ \Delta lh,\ \Delta hl,\ \Delta hh$}
  \State \textbf{load} $h_0,\ h_1,\ mode$
  \State $ h_0,\ h_1 \gets \F{flip}(h_0),\ \F{flip}(h_1) $\Comment{flip the filters as in \eqref{eq:ch3:backprop}}
  % \mbox{\\$g_0,\ g_1$ are 1-D lowpass and highpass filters.}
  \State $\Delta lo \gets \F{sfb1d}(\Delta ll,\ \Delta lh,\ h_0,\ h_1,\ mode,\ axis=-2) $
  \State $\Delta hi \gets \F{sfb1d}(\Delta hl,\ \Delta hh,\ h_0,\ h_1,\ mode,\ axis=-2) $
  \State $\Delta x \gets \F{sfb1d}(\Delta lo,\ \Delta hi,\ h_0,\ h_1,\ mode,\ axis=-1) $
  \State \textbf{return} $\Delta x$
\EndFunction
\end{algorithmic}
\end{algorithm}


\section{Fast Calculation of the $\DTCWT$}\label{sec:ch3:dtcwt}

\section{Changing the ScatterNet Core}\label{sec:ch3:scat}
Now that we have a forward and backward pass for the $\DTCWT$, the final missing
piece is the magnitude operation. Again, it is not difficult to calculate the
gradients given the direct form, but we must be careful about their size. If $z
= x + jy$, then:

\begin{equation}
  r = |z| =  \sqrt{x^2 + y^2}
\end{equation}

This has two partial derivatives, $\dydx{z}{x},\ dydx{z}{y}$:
\begin{eqnarray}
  \dydx{r}{x} & = & \frac{x}{\sqrt{x^2 + y^2}} = \frac{x}{r}\\
  \dydx{r}{y} & = & \frac{y}{\sqrt{x^2 + y^2}} = \frac{y}{r} 
\end{eqnarray}

Except for the singularity at the origin, these partial derivatives are restricted to be in the
range $[-1, 1]$. Also note that the complex magnitude is convex in $x$ and $y$ as:

\begin{equation}
\nabla^2 r(x,y) = \frac{1}{r^3} 
\begin{bmatrix}
  y^2 & -xy \\
  -xy & x^2
\end{bmatrix}
 = \frac{1}{r^3} \begin{bmatrix} y \\ -x \end{bmatrix} 
 \begin{bmatrix} y & -x \end{bmatrix} \geq 0 
\end{equation}
Given an input gradient, $\Delta r$, the passthrough gradient is:
\begin{eqnarray}
  \Delta z & = & \Delta r \dydx{r}{x} + j\Delta r \dydx{r}{y} \\
           &= & \Delta r \frac{x}{r} + j\Delta r \frac{y}{r} \\
           &=& \Delta r e^{j\theta}
\end{eqnarray}
where $\theta = \arctan{\frac{y}{x}}$. This has a nice interpretation to it as
well, as the backwards pass is simply reinserting the discarded phase
information. The pseudo-code for this operation is shown in
\autoref{alg:ch3:mag}.

These partial derivatives are variable around 0, in particular the second
derivative goes to infinity at the origin \textbf{Show a plot of this}. This is not a
feature commonly seen with other nonlinearities such as the tanh, sigmoid and
ReLU, however it is not necessarily a bad thing. The bounded nature of the first
derivative means we will not have any problems so long as our optimizer does not
use higher order derivatives; this is commonly the case. Nonetheless, we
propose to slightly smooth the magnitude operator:

\begin{equation}\label{eq:ch3:magbias}
 r_s = \sqrt{x^2 + y^2 + b^2} - b
\end{equation}

This keeps the magnitude near zero for small $x,y$ but does slightly shrink larger
values. The gain we get however is a new smoother gradient surface \textbf{Plot
this}. We can then increase/decrease the size of $b$ as a hyperparameter in
optimization. The partial derivatives now become:
\begin{eqnarray}
  \dydx{r_s}{x} & = & \frac{x}{\sqrt{x^2 + y^2+b^2}} = \frac{x}{r_s}\\
  \dydx{r_s}{y} & = & \frac{y}{\sqrt{x^2 + y^2+b^2}} = \frac{y}{r_s} 
\end{eqnarray}
There is a memory cost associated with this, as we will now need to save both
$\dydx{r_s}{x}$ and $\dydx{r_s}{y}$ as opposed to saving only the phase.
\autoref{alg:ch3:mag_smooth} has the pseudo-code for this.

\begin{algorithm}[tb]
\caption{Magnitude forward and backward steps}\label{alg:ch3:mag}
\begin{algorithmic}[1]
\Function{mag}{$x,\ y$}
  \State $r \gets \sqrt{x^2 + y^2}$
  \State $\theta \gets \arctan2(y,\ x)$ \Comment{$\arctan2$ handles $x=0$}
  \State \textbf{save} $\theta$ 
  \State \textbf{return} $r$
\EndFunction
\end{algorithmic}\vspace{10pt}
\begin{algorithmic}[1]
\Function{mag\_backprop}{$\Delta r$}
  \State \textbf{load} $\theta$
  \State $\Delta x \gets \Delta r \cos{\theta}$ \Comment{Reinsert phase}
  \State $\Delta y \gets \Delta r \sin{\theta}$ \Comment{Reinsert phase}
  \State \textbf{return} $\Delta x,\ \Delta y$
\EndFunction
\end{algorithmic}
\end{algorithm}

Now that we have the $\DTCWT$ and the magnitude operation, it is straightforward
to get a $\DTCWT$ scattering layer, shown in \autoref{alg:ch3:dtcwt_scat}. To
get a multilayer scatternet, we can call the same function again on $Z$, which
would give $S_0,\ S_1$ and $U_2$ and so on for higher orders. 

Note that for ease in handling the different sample rates of the lowpass and the
bandpass, we have averaged the lowpass over a $2\x 2$ window. This slightly
affects the higher order coefficients, as the true $\DTCWT$ needs the doubly
sampled lowpass for the second scale. We noticed little difference in
performance from doing the true $\DTCWT$ and the decimated one.

\begin{algorithm}[tb]
\caption{$\DTCWT$ ScatterNet Layer}\label{alg:ch3:dtcwt_scat}
\begin{algorithmic}[1]
\Function{dtcwt\_scat}{$x$}
  \State $yl,\ yh \gets \DTCWT(x)$
  \State $S_0 \gets \F{avg\_pool}(yl, 2)$ \Comment{the lowpass is double the sample
  size of the bandpass}
  \State $U_1 \gets \F{mag}(yh.real,\ yh.imag)$ 
  \State $Z \gets \F{concatenate}(S_0,\ U_1)$ \Comment{stack 1 lowpass with 6
  magnitudes}
  \State \textbf{return} $Z$
\EndFunction
\end{algorithmic}
\end{algorithm}
