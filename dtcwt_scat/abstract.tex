The drive of this thesis is in exploring if wavelet theory, in
particular the $\DTCWT$, has any place in deep learning and if it does,
quantifying how beneficial it can be. The introduction of more powerful GPUs and
fast and popular deep learning frameworks such as PyTorch, Tensorflow and Caffe
in the past few years has helped the field of deep learning grow very rapidly.
Never before has it been so possible and so accessible to test new designs and
ideas for a machine learning algorithm than today. Despite this rapid growth,
there has been little interest in building wavelet analysis software in modern
frameworks. 

This poses a challenge and an opportunity. To pave the way for more detailed 
research (both by myself in the rest of this thesis, and by other researchers
who want to explore wavelets applied to deep learning), we must have the right
foundation and tools to facilitate research.

A good example of this is the current implementation of the ScatterNet. While
ScatterNets have been the most promising start in using wavelets in a deep
learning system, they are orders of magnitude slower and significantly more
difficult to run than a standard convolutional network. 

Additionally, any researchers wanting to explore the DWT in a deep learning
system have to rewrite the filter bank implementation themselves, ensuring they
correctly handle boundary conditions and ensure correct filter tap alignment to
achieve perfect reconstruction.

This chapter describes how we have built a fast ScatterNet implementation in 
PyTorch with the $\DTCWT$ as its core. At the core of that is an efficient 
implementation of the DWT\@. The result is an open source library that provides
all three, available on GitHub as \emph{PyTorch Wavelets} \cite{pytorch_wavelets}.

In parallel with our efforts, the original authors of the ScatterNet have
improved their implementation, also building it on PyTorch. My proposed
$\DTCWT$ ScatterNet is $15-35$x faster than their improved implementation,
depending on the padding style and wavelet length, while using less memory. 

