\section{Conclusion}
In this chapter we have proposed changing the backend for Scattering transforms
from a Morlet wavelet transform to the spatially separable $\DTCWT$. This was
originally inspired by the need to speed up the slow Matlab scattering package,
as well as to provide GPU accelerated code that could do wavelet transforms as part
of a deep learning package.

We have derived the forward and backpropagation functions necessary to do 
fast and memory efficient DWTs, $\DTCWT$s, and Scattering based on the $\DTCWT$, 
and have made this code publically available at \cite{cotter_pytorch_2018}. We
hope that this will reduce some of the barriers we faced in using wavelets and
Scattering in deep learning.

In parallel with our efforts, the original ScatterNet authors rewrote their
package to speed up Scattering. In theory, a spatially separable wavelet
transform acting on $N$ pixels has order $\mathcal{O}(N)$ whereas an FFT based
implementation has order $\mathcal{O}(N \log N)$. We have shown experimentally
that on modern GPUs, the difference is far larger than this, with the $\DTCWT$
backend an order of magnitude faster than Fourier-based Morlet implementation
\cite{andreux_kymatio:_2018}.

Additionally, we have experimentally verified that using a different complex
wavelet backend does not have a negative impact on the performance of the
ScatterNet as a frontend to Hybrid networks. 
