In this chapter, we move away from the ScatterNet ideas from the previous 
chapters and instead look at using the wavelet domain as a new space in which to
learn. With ScatterNets, complex wavelets are used to scatter the energy into
different channels (corresponding to the different wavelet subbands), before the
complex modulus demodulates the signal to low frequencies. These channels can
then be mixed before scattering again (as we saw in the learnable scatternet),
but the progressive stages all result in a steady demodulation of signal energy
towards zero frequency. 

In this chapter we introduce the \emph{wavelet gain layer}
which starts in a similar fashion to the ScatterNet -- by taking the $\DTCWT$ of
a multi-channel input. Next, instead of taking a complex modulus, we learn a 
complex gain for each subband in each input channel. A single value here can 
amplify or attenuate all the energy in one part of the frequency plane. Then, 
while still in the wavelet domain, we mix the different input channels by subband (e.g.
all the $15\degs$ wavelet coefficients at a given scale are mixed together, but
the $75\degs$ and $45\degs$ coefficients are not). We can then return to the
pixel domain with the inverse wavelet transform. The shift-invariant properties
of the $\DTCWT$ allows gains and phases to be changed at will without
introducing sampling artifacts.

We also briefly explore the possibility of doing nonlinearities in the wavelet
domain. The goal being to ultimately connect multiple wavelet gain layers
together with nonlinearities before returning to the pixel domain. 

The proposed wavelet gain layer can be used in conjunction with regular
convolutional layers, with a network moving into the wavelet or pixel space and
learning filters in one that would be difficult to learn in the other.

Our experiments so far have shown some promise. We are able to learn complex
wavelet gains and have found that the ReLU works well as a wavelet nonlinearity.
However, replacing a convolutional layer with a gain layer degrades performance
by a small amount.

\section{Chapter Layout}
