\chapter{Complex Convolution and Gradients}\label{app:ch6:complex_backprop}
Consider a complex number $z=x+iy$, and the complex mapping 
\begin{equation}
w=f(z) = u(x,y) + iv(x,y)
\end{equation}
where $u$ and $v$ are called `conjugate functions'. Let us examine the
properties of $f(z)$ and its gradient. 

The definition of gradient for complex numbers is:
\begin{equation} 
  \lim_{\Delta z\to 0} \frac{f(z + \Delta z) - f(z)}{\Delta z}
\end{equation}

A necessary condition for $f(z,\conj{z})$ to be an analytic function is
$\dydx{f}{\conj{z}}=0$. I.e.\ f must be purely a function of $z$, and not
$\conj{z}$.

A geometric interpretation of complex gradient is shown in
\autoref{fig:app6:complex_grad}.  As $\Delta z$ shrinks to 0, what does $\Delta w$
converge to? E.g.\ consider the gradient of approach $m=\frac{dy}{dx}=\tan
\theta$, then the derivative is
\begin{equation}
  \gamma = \alpha + i\beta = D(x,y) + P(x,y)e^{-2i\theta}
\end{equation}
where
\begin{align}
  D(x,y) & =  \half(u_x + v_y + i(v_x - u_y)) \\
  P(x,y) & =  \half(u_x - v_y + i(v_x + u_y))
\end{align}
$P(x,y)=\frac{dw}{d\conj{z}}$ needs to be 0 for the function to be analytic.
This is where we get the Cauchy-Riemann equations:
\begin{align}
  \dydx{u}{x}&= \dydx{v}{y} \\
  \dydx{u}{y}&= -\dydx{v}{x}
\end{align}
The function $f(z)$ is analytic (or regular or holomorphic) if the derivative 
$f'(z)$ exists at all points z in a region $R$. If $R$ is the entire z-plane,
then f is entire. 

\begin{figure}[!h]
	\centering
  \subfloat{%
	\begin{tikzpicture}[>=latex,scale=0.6]
	\draw[style=help lines, color=gray!30, dashed] (-4.9,-4.9) grid[step=1cm] (4.9,4.9);
	\draw[->,ultra thick] (-5,0)--(5,0) node[right]{$x$};
	\draw[->,ultra thick] (0,-5)--(0,5) node[above]{$y$};

	\coordinate (z) at (20:4); % Define some complex numbers
	\coordinate (dz) at (100:1); % with r*exp(i \theta) form

	\draw[->,thick,blue] (0,0) -- (z) node[near end,below=-0.05] {$z$};
	\draw[->,thick,red] (z)  -- ++(dz) node[near end,right] {$\Delta z$};
	\draw[->,thick,green] (0,0) -- ($(z) + (dz)$) node[near end,above left] {$z+\Delta z$};
	\node[above,font=\large] at (current bounding box.north) {z-plane};
	\end{tikzpicture}
}
\subfloat{%
	\begin{tikzpicture}[>=latex,scale=0.6]
	\draw[style=help lines, color=gray!30, dashed] (-4.9,-4.9) grid[step=1cm] (4.9,4.9);
	\draw[->,ultra thick] (-5,0)--(5,0) node[right]{$x$};
	\draw[->,ultra thick] (0,-5)--(0,5) node[above]{$y$};

	\coordinate (w) at (-30:4); % Define some complex numbers
	\coordinate (dw) at (20:1); % with r*exp(i \theta) form

	\draw[->,thick,blue] (0,0) -- (w) node[near end,below] {$w$};
	\draw[->,thick,red] (w) -- ++(dw) node[near end,below] {$\Delta w$};
	\draw[->,thick,green] (0,0) -- ($(w) + (dw)$) node[near end, above=0.1] {$w+\Delta w$};
	\node[above,font=\large] at (current bounding box.north) {w-plane};
	\end{tikzpicture}
}

  \mycaption{Geometric interpretation of complex gradient}{The gradient is
  defined as $f'(z) = \lim_{\Delta z \to 0} \frac{\Delta w}{\Delta z}$. It must
  approach the same value independent of the direction $\Delta z$ approaches
  zero. This turns out to be a very strong and somewhat restrictive property.}
  \label{fig:app6:complex_grad}
\end{figure}

\section{Grad Operator}
Recall, the gradient is a multi-variable  generalization of the derivative. The
gradient is a vector valued function. In the case of complex numbers, it can be
represented as a complex number too. E.g.\ consider $W(z) = F(x,y)$ (note that
in general it may be simple to find F given G, but they are different
functions).

I.e. 
$$\nabla F = \dydx{F}{x} + i \dydx{F}{y}$$

Consider the case when F is purely real, then 
$F(x,y) = F(\frac{z+\conj{z}}{2},\frac{z-\conj{z}}{2i}) = G(z,\conj{z})$
Then
$$\nabla F = \dydx{F}{x} + i \dydx{F}{y} = 2\dydx{G}{\conj{z}}$$

If F is complex, let $F(x,y) = P(x,y) + iQ(x,y) = G(z,\conj{z})$, then
$$\nabla F = \left(\dydx{}{x} + i \dydx{}{y}\right)(P+iQ) 
           = \left(\dydx{P}{x} - \dydx{Q}{x}\right) + 
                i\left(\dydx{P}{y} + \dydx{Q}{x}\right) 
           = 2\dydx{G}{\conj{z}}$$
It is clear to see how the purely real case is a subset of this (set Q=0 and
all its partials will be 0 too).

If G is an analytic function, then $\dydx{G}{\conj{z}} = 0$ and so the gradient
is 0, and the Cauchy-Riemann equations hold $\dydx{P}{x}=\dydx{Q}{y}$ and 
$\dydx{P}{y}=-\dydx{Q}{x}$

% \section{Hilbert Pairs of General Functions}
% How does this affect me? I don't think I'll be able to use analytic
% non-linearities, however I may be able to have analytic filters, like those of
% the $\DTCWT$.

% The Hilbert pair of the cosine is the sine function, but what about in general?
% If $x=\delta(t)$, its Hilbert pair $jy = \frac{-j}{\pi t}$. Like the dirac
% delta function, this also has a flat spectrum, and the figure for it is shown
% below.

  % \begin{center}
    % \begin{tikzpicture}[scale=0.6]
  \draw[->] (-4,0) -- (4,0) node[right] {$x$};
  \draw[->] (0,-4) -- (0,4) node[above] {$y$};
   \draw[domain=-4:-0.1,smooth,variable=\x,blue] 
     plot ({\x},{1/(pi*\x)});
   \draw[domain=0.1:4,smooth,variable=\x,blue] 
     plot ({\x},{1/(pi*\x)});
\end{tikzpicture}

  % \end{center}

% That means if we wanted to get the Hilbert pair of a sampled signal, then we
% would have to add shifts and scales of $y$, which unfortunately has infinite
% support. We would also have to lowpass it, as we do for the sampled version
% (so their frequency spectrums are the same).

% \section{Usefulness of Complex Numbers}
% Nick made a good point in our recent meeting that when trying to use the
% complex plane, we must know/understand what it is we want to gain from the
% added representation. For the case of the \DTCWT, he converted the non-analytic
% sinusoids of the wavelet transform into an analytic signal.

% I.e.\ let us ignore the previous notation of $x=\real{z}, y=\imag{z}$ and redefine
% them to indicate the horizontal and vertical directions in an image.

% For a real wavelet transform, all of the cosine terms are:
% \begin{equation}
  % \cos \omega_1 x = \half\left(e^{j\omega_1 x} + e^{-j\omega_1 x} \right)
% \end{equation}

% If we consider $z_x = e^{j\omega_1 x}$, then this is clearly a function of both
% $z_x$ and its conjugate (as are all real valued functions). I.e. $\cos \omega_1
% x = F(z_x,\conj{z_x})$. Nick replaced this with the analytic equivalent of this
% function by adding in the Hilbert pair term.

% \begin{equation}
  % \cos \omega_1 x +j \sin \omega_1 x = e^{j\omega_1 x} = F(z_x)
% \end{equation}

% From our above definitions of analytic functions, it is clear to see that this
% is now no longer a function of the conjugate term $\conj{z_x}$, so it is analytic. The
% benefit for Nick was that now he could separably multiply the x and the
% y sinusoids to get:
% \begin{equation}
  % e^{j\omega_1 x} \times e^{j\omega_2 y} =F(z_x)F(z_y)= e^{j(\omega_1 x + \omega_2 y)} = F(z_x+z_y)
% \end{equation}

% \section{Working with Complex weights in CNNs}\label{ch6:app:complex_weights}
  % As a first pass, I think I shouldn't concern myself too much with analytic
  % functions and having the Cauchy-Riemann equations met. Instead, I will focus
  % on implementing the CNN with a real and imaginary component to the filters,
  % and have these stored as independent variables.\\\\
  % Unfortunately, most current neural network tools only work with real numbers,
  % so we must write out the equations for the forward and backwards passes, and
  % ensure ourselves that we can achieve the equivalent of a complex valued
  % filter.

\section{Forward pass}
\subsection{Convolution}
  %% Define the variables we will use in this section
\newcommand{\SigIn}{z}
\newcommand{\SigOut}{w}
\newcommand{\Filter}{f}
\newcommand{\SigInB}{\bmu{\SigIn}}
\newcommand{\SigOutB}{\bmu{\SigOut}}
\newcommand{\FilterB}{\bmu{\Filter}}
  %% Now the body
  
In the above example $\FilterB$ has a spatial support of only $1\x 1$. We still
were able to get a somewhat complex shape by shifting the relative phases of
the complex coefficients, but we are inherently limited (as we can only
rotate the coefficient by at most $2\pi$). So in general, we want to be able
to consider the family of filters $\FilterB \in \complexes[m_1 \x m_2 \x C]$. For
ease, let us consider only square filters of spatial support $m$, so
$\FilterB \in \complexes[m\x m\x C]$. Note that we have restricted the third
dimension of our filter to be $C=12$ in this case. This means that convolution is
only in the spatial domain, rather than across channels. Ultimately we would
like to be able to handle the more general case of allowing the filter to
rotate through channels, but we will tackle the simpler problem 
first\footnote{Recall from \autoref{fig:ch4:newshapes}, the benefit of allowing
a filter to rotate through the channel dimension was we could easily obtain
$30\degs$ shifts of the sensitive shape.}

Let us represent the complex input with $\SigIn$, which is of shape
$\complexes[n_1\x n_2\x C]$. We call $\SigOut$ the result we get from convolving
$\SigInB$ with $\FilterB$, so $\SigOutB \in \complexes[n_1 + m - 1, n_2 + m -1, 1]$. With
appropriate zero or symmetric padding, we can make $\SigOut$ have the same
spatial shape as $\SigInB$. Now, consider the full complex convolution to get
$\SigOut$:
\begin{equation}
 \SigOut[l_1,l_2] = \sum_{c=0}^{C-1} \sum_{k_1,k_2} \Filter[k_1,k_2,c] \SigIn[l_1-k_1,l_2-k_2,c]
\end{equation}
Let us define 
\begin{align}
  \SigIn & =  \SigIn_R +j\SigIn_I \\
  \SigOut & =  \SigOut_R +j\SigOut_I \\
  \Filter & =  \Filter_R + j\Filter_I
\end{align}
where all of these belong to the real space of the same dimension as their
parent. Then
\begin{align}
  \SigOut[l_1, l_2]  =&  \SigOut_R +j\SigOut_I \nonumber\\
  = & \sum_{c=0}^{C-1} \sum_{k_1,k_2} \Filter[k_1,k_2,c] \SigIn[l_1-k_1,l_2-k_2,c] \nonumber\\
  =& \sum_{c=0}^{C-1} \sum_{k_1,k_2} (\Filter_R[k_1,k_2,c] +j\Filter_I[k_1,k_2,c]) 
        (\SigIn_R[l_1-k_1,l_2-k_2,c]+ j\SigIn_I[l_1-k_1,l_2-k_2,c]) \nonumber \\
        =&  \sum_{c=0}^{C-1} \sum_{k_1,k_2} (
          \SigIn_R[l_1-k_1,l_2-k_2,c] \Filter_R[k_1,k_2,c] - 
          \SigIn_I[l_1-k_1,l_2-k_2,c] \Filter_I[k_1,k_2,c]) \nonumber \\
          &+ j\sum_{c=0}^{C-1} \sum_{k_1,k_2} (\SigIn_R[l_1-k_1,l_2-k_2,c] \Filter_I[k_1,k_2,c] + 
          \SigIn_I[l_1-k_1,l_2-k_2,c] \Filter_R[k_1,k_2,c]) \nonumber\\
          =& \left((\SigIn_R \ast \Filter_R) - (\SigIn_I \ast \Filter_I)\right)[l_1,l_2] + 
          \left((\SigIn_R \ast \Filter_I) + (\SigIn_I \ast \Filter_R)\right)[l_1,l_2]
\end{align}
Unsurprisingly, complex convolution is then the sum and difference of 4 real convolutions.

% \section{Backwards Pass}  
  % %% Define the variables we will use in this section
  % \renewcommand{\dataloss}{\logloss_{\text{data}}}
  % \renewcommand{\SigIn}{z}
  % \renewcommand{\SigOut}{w}
  % \renewcommand{\Filter}{f}
  
% Now we need to calculate what the complex gradients will be, and importantly,
% how to implement them using only real representations.

% A typical loss function is:
% \begin{eqnarray}
  % \logloss &=& \dataloss + \logloss_{\text{reg.}} \\
    % &=& \frac{1}{N}\sum_{n=0}^{N-1} \sum_{k} y_{nk}\log \pi_{nk} + \lambda
    % \sum_l \theta_l^2
% \end{eqnarray}
% The derivative of the regularizer loss w.r.t.\ the parameters is trivial to
% find, so let us only focus on the data loss for now.

\subsection{Regularization}
We must be careful with regularizing complex weights. We want to set some
of the weights to 0, and let the remaining ones evolve to whatever phase they
please. To do this, either use the L-2 norm on the real and imaginary parts
independently, or be careful about using the L-1 norm. This is because we
really want to be penalising the magnitude of the complex weights, $r$ and: 
\begin{equation}
  \lnorm{r}{2}^2 = \lnorm{\sqrt{x^2+y^2}}{2}^2 = \sum x^2 + y^2 = \sum x^2 + \sum y^2 = \lnorm{x}{2}^2 + \lnorm{y}{2}^2
\end{equation}
But for $\ell_1$ regularization:
\begin{equation}
  \lnorm{r}{2} = \lnorm{\sqrt{x^2+y^2}}{1} = \sum \sqrt{x^2 + y^2} \neq \lnorm{x}{1} + \lnorm{y}{1}
\end{equation}

\section{Soft Shrinkage}\label{app:ch6:soft_shrink}
Let $z = x + jy$ and $w = u+jv = \mathcal{S}(z, t)$ where we define the soft
shrinkage on a complex number $z = re^{j\theta}$ by a real threshold $t$ as:
\begin{equation}
  \mathcal{S}(z, t) = \left\{ \begin{array}{ll}
      0 & r < t \\
      (r-t)e^{j\theta} & r \geq t\\
\end{array}
\right. 
\end{equation}
This can alternatively be written as:
\begin{align}
  \mathcal{S}(z, t) &= \frac{\max(r-t, 0)}{r} z \\
                    &= gz
\end{align}
To find the pass through gradients $\dydx{L}{x}, \dydx{L}{y}$ we need to find
all the real and imaginary partial derivatives.
\begin{align}
  \dydx{g}{x} &= \left\{ \begin{array}{ll}
      0 & r < t \\
      \frac{r \dydx{r}{x} - (r-t)\dydx{r}{x}}{r^2}  & r \geq t
  \end{array} \right.  \\
  &= \frac{xt\indic(g > 0)}{r^3} \\
  \dydx{g}{y}&= \frac{yt\indic(g > 0)}{r^3} \\
  \dydx{g}{t}&= \frac{-\indic(g > 0)}{r} 
\end{align}
Then from the definition of $w = u+jv$ we have $u=gx$ and $v=gy$, giving us:

\begin{minipage}{.48\linewidth}
\begin{align}
  \dydx{u}{x} &= g + \frac{x^2t\indic(g > 0)}{r^3} \\
  \dydx{u}{y} &= \frac{xyt\indic(g > 0)}{r^3} \\
  \dydx{u}{t} &= \frac{-x\indic(g > 0)}{r}
\end{align}
\end{minipage}
\begin{minipage}{.48\linewidth}
\begin{align}
  \dydx{v}{x} &= \frac{xyt\indic(g > 0)}{r^3} \\
  \dydx{v}{y} &=g + \frac{y^2t\indic(g > 0)}{r^3} \\
  \dydx{v}{t} &= \frac{-y\indic(g > 0)}{r}
\end{align}
\end{minipage}
\vspace{10pt}
And putting it all together we have:
\begin{align}
  \dydx{L}{x} &= \dydx{L}{u}\dydx{u}{x} + \dydx{L}{v}\dydx{v}{x} \\
  \dydx{L}{y} &= \dydx{L}{u}\dydx{u}{y} + \dydx{L}{v}\dydx{v}{y} \\
  \dydx{L}{t} &= \dydx{L}{u}\dydx{u}{t} + \dydx{L}{v}\dydx{v}{t} 
\end{align}
These equations are for point-wise application of soft-thresholding. When
applied to an entire image, then we sum $\dydx{L}{t}$ over all locations.
% \qquad
% \begin{aligned}[c]
  % \dydx{v}{x} &= \frac{xyt\indic(g > 0)}{r^3} = \dydx{u}{y} \\
  % \dydx{v}{y} &= \frac{y^2t\indic(g > 0)}{r^3} \\
  % \dydx{v}{t} &= \frac{-y\indic(g > 0)}{r}
% \end{aligned}
% \end{equation}
