\section{Gain Layer Experiments}\label{sec:ch6:gainlayer_experiments}
To explore the effectiveness of our gain layer, we do a similar ablation study
to \autoref{sec:ch5:conv_exp} on CIFAR-10, CIFAR-100 and Tiny ImageNet. We use
the same reference network so that we can also compare the wavelet gain layer to the
invariant layer (see \autoref{tab:ch5:cifar_tiny_arch}). We also use the same
naming technique, calling a network `gainX' means that the `convX' layer was
replaced with a wavelet gain layer, but otherwise keeping the rest of the
architecture the same. 

In this section, we are only exploring the wavelet gain layer and not any
wavelet nonlinearities. This equates to the path shown in \autoref{fig:ch6:fwd_chain}b. 
I.e\. we are taking wavelet transforms of inputs, applying the gain layer and
taking inverse wavelet transforms to do a ReLU in the pixel space. 
% In cases like
% `gainA, gainB' from \autoref{tab:}, we go in and out of the wavelet layer twice.

We train all our networks for with stochastic gradient descent with momentum.
The initial learning rate is $0.5$, momentum is $0.85$, batch size $N=128$ and
weight decay is $10^{-4}$. For CIFAR-10/CIFAR-100 we scale the learning rate by
a factor of 0.2 after 60, 80 and 100 epochs, training for 120 epochs in total.
For Tiny ImageNet, the rate change is at 18, 30 and 40 epochs (training for 45 in total).

\begin{table}
  \renewcommand{\arraystretch}{1.2}
  \centering
  \mycaption{Results for testing VGG like architecture with convolutional and
  wavelet gain layers on several datasets}{An architecture with `gainX' means the
  equivalent convolutional layer `convX' from \autoref{tab:ch5:arch} was swapped for
  our proposed layer. The top row is the reference architecture using all
  convolutional layers. The DWT architecture takes a single scale DWT of the
  activations at the given layer, and mixes the coefficients to get new wavelet
  coefficients. The lowpass mixing coefficients for the DWT architecture have spatial
  size $3\x 3$ and the bandpass coefficients have spatial size $1\x 1$. The
  $\DTCWT$ architecture takes a single scale $\DTCWT$ and mixes the lowpass and 
  complex bandpass coefficients. Both the lowpass and bandpass coefficients have
  spatial size $1\x 1$. Numbers are averages over 3 runs.} 
  % \begin{tabular}{@{}lllcllcll@{}}
  \begin{tabular}{@{}lcclcclcc@{}}
    \toprule
    & \multicolumn{2}{c}{CIFAR-10} & \phantom{abc} & \multicolumn{2}{c}{CIFAR-100} & \phantom{abc} & \multicolumn{2}{c}{Tiny ImgNet} \\ \cline{2-3}\cline{5-6}\cline{8-9}
    \phantom{abc} & DWT & $\DTCWT$ &&  DWT & $\DTCWT$ && DWT & $\DTCWT$ \\ \midrule
    reference & \multicolumn{2}{c}{92.6} && \multicolumn{2}{c}{72.0} && \multicolumn{2}{c}{59.3} \\ \midrule
    gainA        & & && 72.3 & 72.8 && & \\
    gainB        & & && 72.5 & 73.4 && & \\
    gainC        & & && 72.1 & 72.8 && & \\
    gainD        & & && 72.9 & 73.1 && & \\
    gainE        & & && 72.8 & 72.7 && & \\
    gainF        & & && 73.6 & 72.7 && & \\
    gainA, gainB & & && 70.6 & 72.3 && & \\
    gainB, gainC & & && 71.7 & 73.1 && & \\
    gainC, gainD & & && 71.0 & 72.4 && & \\
    gainD, gainE & & && 72.4 & 71.9 && & \\
    gainA, gainC & & && 71.8 & 73.6 && & \\
    gainB, gainD & & && 71.7 & 72.9 && & \\
    gainC, gainE & & && 71.7 & 72.2 && & \\ \midrule
    gainA-F      & & &&      &      && & \\ \bottomrule
    
  \end{tabular}\label{tab:ch6:ablation_results}
\end{table}



\autoref{tab:ch6:ablation_results} lists the results from these experiments.
These show a promising start to this work. We can get an improvement in accuracy
by over a percent when we use the gain layer \textbf{Finish me when results are
done}.

Unlike the scatternet inspired invariant layer from the previous chapter, the
gain layer does naturally downsample the output, so we are able to stack more of
them? Maybe.

