\section{Introduction}\label{sec:ch6:intro}
Using wavelet based methods with deep learning is nascent but not novel.
Wavelets have been applied to texture classification \cite{fujieda_wavelet_2017,
sifre_combined_2012}, super-resolution \cite{guo_deep_2017} and for adding
detail back into dense pixel-wise segmentation tasks \cite{ma_detailed_2018}.
One exciting piece of work built on wavelets is the Scattering Transform
\cite{mallat_group_2012}, which has been used as a feature extractor for
learning, firstly with simple classifiers \cite{bruna_invariant_2013,
singh_scatternet_2017}, and later as a front end to hybrid deep learning
tasks\cite{oyallon_scaling_2017, singh_scatternet_2018}. Despite their power and
simplicity, scattering features are fixed and are visibly different to regular
CNN features \cite{cotter_visualizing_2017} - their nice invariance properties
come at the cost of flexibility, as there is no ability to learn in between
scattering layers. 

For this reason, we have been investigating a slightly different approach, more
similar to the Fourier based work in \cite{rippel_spectral_2015} in which Rippel
et.\ al.\ investigate parameterization of filters in the Fourier domain. In the
forward pass, they take the inverse DFT of their filter, and then apply normal
pixel-wise convolution. We wish to extend this by not only parameterizing
filters in the wavelet domain, but by performing the convolution there as well
(i.e., also taking the activations into the wavelet domain). After processing is
done, we can return to the pixel domain. Doing these forward and inverse
transforms has two significant advantages: 
\begin{enumerate*}[label=\roman*)]
  \item the layers can easily replace standard convolutional layers if they
    accept and return the same format;
  \item we can learn both in the wavelet and pixel space.
\end{enumerate*}

As neural network training involves presenting thousands of training samples, we
want our layer to be fast. To achieve this we would ideally choose to use
a critically sampled filter bank implementation. The fast 2-D Discrete Wavelet
Transform (DWT) is a possible option, but it has two drawbacks: it has poor
directional selectivity and any alteration of wavelet coefficients will cause
the aliasing cancelling properties of the reconstructed signal to disappear.
Instead we choose to use the Dual-Tree Complex Wavelet Transform ($DTCWT$)
\cite{selesnick_dual-tree_2005} as at the expense of limited redundancy (4:1),
it enables us to have better directional selectivity, and allows us to modify
the wavelet coefficients and still have minimal aliasing terms when we
reconstruct \cite{kingsbury_complex_2001}.

% This work is a step
\autoref{sec:ch6:method} of the paper describes the implementation details of
our design, and \autoref{sec:ch6:results} describes the experiments and results
we have done so far.

