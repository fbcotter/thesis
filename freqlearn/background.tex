\section{Background and Notation}
As we now want to consider the DWT and the $\DTCWT$ which are both implemented
as filter bank systems, we deviate slightly from the notation in the previous
chapter (which was inspired by sampling a continuous wavelet transform). 

Firstly, instead of talking about the continuous spatial variable $\xy$, we now
consider the discrete spatial variable $\nn = [n_1, n_2]$. We switch to square 
brackets to make this clearer. With the new discrete notation, the output of a CNN at layer $l$ is:
%
\begin{equation}
  \cnndlact{x}{l}{c}{\nn}, \quad c\in \{0, \ldots C_l-1\}, \nn \in \integers[2]
\end{equation}
%
where $c$ indexes the channel dimension. 
We also make use of the 2-D $Z$-transform to simplify our analysis:
%
\begin{equation}
  X(\zz) = \sum_{n_1}\sum_{n_2} x[n_1, n_2]z_1^{-n_1}z_2^{-n_2} =
  \sum_{\nn}x[c, \nn]\zz^{-\nn}
\end{equation}
%
As we are working with three dimensional arrays (two spatial and one channel) but are
only doing convolution in two, we introduce a slightly modified 2-D $Z$-transform
which includes the channel index:
%
\begin{equation}
  X(c, \zz) = \sum_{n_1}\sum_{n_2} x[c, n_1, n_2]z_1^{-n_1}z_2^{-n_2} =
  \sum_{\nn}x[c, \nn]\zz^{-\nn} \label{eq:ch6:ztransform}
\end{equation}

Recall that a typical convolutional
layer in a standard CNN gets the next layer's output in a two-step process:
%
\begin{eqnarray} 
  \cnndlact{y}{l+1}{f}{\nn} &=& \sum_{c=0}^{C_l - 1} \cnndlact{x}{l}{c}{\nn} \conv \cnndfilt{l}{f}{c}{\nn}
    \label{eq:ch6:conv}\\
    \cnndlact{x}{l+1}{f}{\xy} & = & \sigma \left( \cnndlact{y}{l+1}{f}{\xy} \right) \label{eq:ch6:nonlin}
\end{eqnarray}
%
In shorthand, we can reduce the action of the convolutional layer in \eqref{eq:ch6:conv} to $\mathcal{H}$, saying:
\begin{equation}
  y^{(l+1)} = \mathcal{H}x^{(l)}
\end{equation}

With the new $Z$-transform notation introduced in \eqref{eq:ch6:ztransform}, we
can rewrite \eqref{eq:ch6:conv} as:

\begin{equation}
  \cnnlact{Y}{l+1}{f}{\zz} = \sum_{c=0}^{C_l - 1} \cnnlact{X}{l}{c}{\zz}
  H_f^{(l)}(c, \zz)
\end{equation}
%
Note that we cannot rewrite \eqref{eq:ch6:nonlin} with $Z$-transforms as it is a nonlinear
operation.

Also recall that with multirate systems, upsampling by $M$ takes $X(z)$ to
$X(z^M)$ and downsampling by $M$ takes $X(z)$ to $\frac{1}{M}\sum_{k=0}^{M-1} X(W_M^k
z^{1/k})$ where $W_M^k = e^{\frac{j2\pi k}{M}}$. We will drop the $M$ subscript
below unless it is unclear of the sample rate change, simply using $W^k$.

\subsection{DWT Notation}
In 2-D, a $J$ scale DWT gives $3J+1$ coefficients, 3 sets of bandpass
coefficients for each scale, representing the horizontal, vertical and diagonal
regions of the frequency plane, and 1 set of lowpass coefficients at the final
scale. Let us refer to the DWT wavelet coefficients with the typewriter font
$\mathtt{u}$ and write the $J$ scale DWT as:

\begin{equation}
  \F{DWT}_J(x) = \mathtt{u}_{lp}, \{\mathtt{u}_{j,k} \}_{1\leq j\leq J, 1\leq k\leq 3}
  \label{eq:ch6:dwt_coeffs}
\end{equation}

If $x$ is a batch of 2-D images with multiple channels, the DWT is done
indpendently on each channel in each minibatch sample. I.e.\ if the input is a
minibatch of $N$ samples with $C$ channels of images of spatial size $H\x W$,
then $x \in \reals[N\x C\x H\x W]$ and:
%
\begin{eqnarray}
  \mathtt{u}_{lp} &\in & \reals[N\x C\x \frac{H}{2^{J}} \x \frac{W}{2^{J}}] \\
  \mathtt{u}_{j,k} &\in & \reals[N\x C\x \frac{H}{2^{J}}\x \frac{W}{2^{J}}]
\end{eqnarray}

Sometimes we may want to refer to all of the subband coefficients at a single scale
with one variable, in which case we will simply drop the $k$ subscript 
and call them $\mathtt{u}_{j}$. Additionally if we want to refer to all the
coefficients (lowpass and bandpass) we will call them $\mathtt{u}$.

\subsection{$\DTCWT$ Notation}
Unlike the DWT, a $J$ scale $\DTCWT$ gives $6J+1$ coefficients, 6 sets of complex
bandpass coefficients for each scale (representing the oriented bands from 15 to 165
degrees) and 1 set of real lowpass coefficients. 

Although we will only ever use either the DWT or the $\DTCWT$, to avoid
confusion we use a regular font $u$ to refer to the $\DTCWT$ coefficients of
$x$:

\begin{equation}
  \DTCWT_J(x) = u_{lp}, \{u_{j,k} \}_{1\leq j\leq J, 1\leq k\leq 6}
  \label{eq:ch6:dtcwt_coeffs}
\end{equation}

Each of these coefficients then has size:
%
\begin{eqnarray}
  u_{lp} &\in & \reals[N\x C\x \frac{H}{2^{J-1}} \x \frac{W}{2^{J-1}}] \\
  u_{j,k} &\in & \complexes[N\x C\x \frac{H}{2^{J}}\x \frac{W}{2^{J}}]
\end{eqnarray}
%
Note that the lowpass coefficients are twice as large as in a fully decimated
transform, a feature of the redundancy of the $\DTCWT$.

Again if we ever want to refer to all the subbands at a given scale, we will
drop the $k$ subscript and call them $u_j$. Likewise, $u$ refers to the whole
set of $\DTCWT$ coefficients.

When we want to be agnostic of the chosen transform type, we use
$\mathcal{W}$ and $\mathcal{W}^{-1}$ to denote the forward and inverse
transform.


