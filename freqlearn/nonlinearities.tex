\section{Wavelet Based Nonlinearities and Multiple Gain Layers}
So far we have only considered doing the mixing operations in the wavelet
domain. This is a good starting point, and it is nice to see that we can do
apply the gain layer that preserve some of the nice properties of the $\DTCWT$.
Let us call the layer considered so far a \textbf{first order gain layer}.
Recall this is where a $\DTCWT$ is done, followed by a single linear convolution
and then straight away returning to the pixel domain with the inverse $\DTCWT$.
While it is good to see that it is possible to do and even achieves some benefit
over a convolutional layer, the proposed layer \eqref{eq:ch6:end2end} can be
implemented in the pixel domain as a single convolution.

It would be particularly interesting if we could find a sensible nonlinearity to
do in the wavelet domain. This would mean it would be no longer possibile to do the
gain layer in the pixel domain. Further, we could then do multiple mixing
stages in the wavelet domain before returning to the pixel domain.

But what sensible nonlinearity to use? Two particular options are good initial
candidates:
\begin{enumerate}
  \item The ReLU: this is a mainstay of most modern neural networks and has
    proved invaluable in the pixel domain. Perhaps its sparsifying properties
    will work well on wavelet coefficients too. 
  \item Soft thresholding: a technique commonly applied to wavelet
    coefficients for denoising and compression. Many proponents of compressed
    sensing and dictionary learning even like to compare soft thresholding to a
    two sided ReLU \cite{papyan_theoretical_2018, papyan_convolutional_2016}.
\end{enumerate}

In this section we will look at each, see if they add to the first order gain
layer, and see if they open the possibility of having multiple layers in the
wavelet domain. 

\subsection{ReLUs in the Wavelet Domain}
A ReLU could be applied to the real lowpass coefficients with ease, but it does
not generalize so easily to complex coefficients. One option could be to apply
it independently to the real and imaginary coefficients, effectively only
selecting one quadrant of the complex plane.

One potential problem with this is
that applying a ReLU independently to the real and imaginary components

\subsection{Non-Linearity}
  %% Define the variables we will use in this section
\renewcommand{\SigIn}{w}
\renewcommand{\SigOut}{y}

	\begin{figure}[!h]
    \centering
    \subfloat{%
    \begin{tikzpicture}[>=latex]
        \draw[->,thick] (-2,0)--(2,0) node[right]{\SigIn};
        \draw[->,thick] (0,-1)--(0,2) node[above]{\SigOut};

        \draw[->,thick] (0.5,0) node[start,below] {b} -- (1.5,1) node[end,above] {ReLU};
    \end{tikzpicture}
}
\subfloat{%
    \begin{tikzpicture}[tdplot_main_coords,scale=1.2]
        \draw[thick,->] (-2,0,0) -- (2,0,0) node[anchor=north east]{$\SigIn_R$};
        \draw[thick,->] (0,-2,0) -- (0,2,0) node[anchor=north west]{$\SigIn_I$};
        \draw[thick,->] (0,0,-0.5) -- (0,0,2) node[anchor=south]{$\SigOut$};

        \coordinate (Shift1) at (0,0,0.5);
        \coordinate (Shift2) at (0,0,1);
        \draw (0,0) circle [radius=0.5];
        \tdplotsetrotatedcoordsorigin{(Shift1)}
        \draw (0,0) circle [radius=1,tdplot_rotated_coords];
        \tdplotsetrotatedcoordsorigin{(Shift2)}
        \draw (0,0) circle [radius=1.5,tdplot_rotated_coords];

        \draw (0,0.5,0) node[start,below right] {b} -- (0,1.5,1);
        \draw ($1/sqrt(2)*(0.5,-0.5,0)$) -- ($1/sqrt(2)*(1.5,-1.5,sqrt(2))$);
        \draw (.5,0,0) -- (1.5,0,1);
        \draw (-.5,0,0) -- (-1.5,0,1);

    \end{tikzpicture}
}

	\end{figure}
	
As a first attempt at a non-linearity, we will try a modified L2-norm. In 
particular, for input $\SigIn$ (coming from the convolutional layer), the
output $\SigOut$ is:
\begin{equation}
	\SigOut = g(\SigIn, b) = \sqrt{\max(0, \SigIn_R^2 + \SigIn_I^2 - b^2)}
\end{equation}
This is an attempt to modify the ReLU nonlinearity by rotating it
around the complex plane. Really this is more like the complex generalization
of the function $y = \max(0, |x| -b)$, as the ReLU rejects half of the real
line, whereas this function only rejects small magnitude values. I justify
this like so:
\begin{itemize}
  \item If a patch of pixels strongly correlates with
    a filter\footnote{really this should read `strongly correlates with the
    mirror image of a filter', but the meaning is the same}, it will
    produce a large output. 
  \item A large output will be caught by the ReLU and `passed' through, with
    a small subtraction penalty, $b$.
  \item If a patch of pixels does not correlate with a filter, it produces
    a small output, which gets clipped to 0 by the ReLU.
  \item If a patch of pixels negatively correlates with a filter, say if we
    multiply a strongly correlated input with -1, then
    \textbf{this should be distinguishable from zero correlation}. A way to
    do this would be use a soft limiting function ($y=\min(0,x+b)
    + \max(0,x-b)$), or the rectified version of the soft limiter
    ($y=\max(0,|x| - b)$.
  \item The only downside to these non-linearities is that they theoretically
    `fire' on much more of the input, which could mean a dramatic change in
    network behaviour, but really it shouldn't be so drastically different.
    We can verify this by measuring the probability distribution at the 
    input to non-linearities in a real valued CNN. It is easy to show that
    they are roughly normally distributed.
\end{itemize}

