\section{Wavelet-Based Nonlinearities}\label{sec:ch6:nonlinearities}
Returning to the goals from \autoref{sec:ch6:learning}, the experiments from the
previous section have shown that while it is possible to use a wavelet gain
layer ($G$) in place of a convolutional layer ($H$), this may come with a small
performance penalty. Ignoring this effect for the moment, in this section, we
continue with our investigations into learning in the wavelet domain. In
particular, is it possible to replace a pixel domain nonlinearity $\sigma$ with
a wavelet-based one $\sigma_w$?

But what sensible nonlinearity to use? Two particular options are good initial
candidates:
\begin{enumerate}
  \item The ReLU: this is a mainstay of most modern neural networks and has
    proved invaluable in the pixel domain. Perhaps its properties
    will work well on wavelet coefficients too.
  \item Thresholding: a technique commonly applied to wavelet
    coefficients for denoising and compression. Many proponents of compressed
    sensing and dictionary learning even like to compare soft thresholding to a
    two-sided ReLU \cite{papyan_theoretical_2018, papyan_convolutional_2016}.
\end{enumerate}

In this section, we will look at both and see if they improve the gain
layer. If they do, it would the be possible to connect multiple layers in the
wavelet domain, avoiding the necessity to do inverse wavelet transforms after
learning.

\subsection{ReLUs in the Wavelet Domain}
Applying the ReLU to the real lowpass coefficients is not difficult, but it does
not generalize so easily to complex coefficients. The simplest option is to apply
it independently to the real and imaginary coefficients, effectively only
selecting one quadrant of the complex plane:
\begin{align}
  u_{lp} &= \F{max}(0,\ v_{lp}) \\
  u_{j} &= \F{max}(0,\ \real{v_{j}}) + j\F{max}(0,\ \imag{v_j}) \label{eq:ch6:relu_bp}
\end{align}

Another option is to apply it to the magnitude of the bandpass coefficients. Of
course, these are all strictly positive so the ReLU on its own would not do
anything. However, they can be arbitrarily scaled and shifted by using a batch
normalization layer. Then the magnitude could shift to (invalid) negative
values, which can then be rectified by the ReLU.

Dropping the scale subscript $j$ for clarity (we need it for the square root of 
negative 1), let a bandpass coefficient at a given scale be
$v = r_v e^{j\theta_v}$ and define
$\mu_r = \mathbb{E}[r_v]$ and $\sigma_r^2 = \mathbb{E}[(r_v-\mu_r)^2]$, then
applying batch-normalization and the ReLU to the magnitude of $v_j$ means we
get:
\begin{align}
  r_u &= \F{ReLU}(BN(r_v)) = \max(0,\ \gamma \frac{r_v-\mu_r}{\sigma_r} + \beta) \label{eq:ch6:magrelu_bp} \\
  u &= r_u e^{j\theta_v} \label{eq:ch6:magrelu_bp2}
\end{align}
This also works equivalently on the lowpass coefficients, although $v_{lp}$ can
be negative unlike $r_v$:
\begin{equation}
  u_{lp} = \F{ReLU}(BN(v_{lp})) = \max(0, \gamma' \frac{v_{lp} - \mu_{lp}}{\sigma_{lp}} + \beta') \label{eq:ch6:bnrelu_lp}
\end{equation}
%
\subsection{Thresholding}
For $t \in \reals,\ z = re^{j\theta} \in \complexes$ the pointwise hard thresholding is:
\begin{align}
  \mathcal{H}(z, t) &= \left\{ \begin{array}{ll}
    z & r \geq t \\
    0 & r < t\\
  \end{array} \right. \\
  &= \indic(r > t) z
\end{align}
and the pointwise soft thresholding is:
\begin{align}
  \mathcal{S}(z, t) &= \left\{ \begin{array}{ll}
    (r-t)e^{j\theta} & r \geq t \\
    0 & r < t\\
  \end{array} \right. \\
  &= \max(0, r - t)e^{j\theta} \label{eq:ch6:relu_st}
\end{align}
Note that \eqref{eq:ch6:relu_st} is very similar to \eqref{eq:ch6:magrelu_bp} and \eqref{eq:ch6:magrelu_bp2}.
We can rewrite \eqref{eq:ch6:magrelu_bp} by taking the strictly positive terms
$\gamma$, $\sigma$ outside of the $\max$ operator:
\begin{align}
  r_u &= \max(0, \gamma \frac{r_v-\mu_r}{\sigma_r} + \beta) \\
      &= \frac{\gamma}{\sigma_r}\max\left(0, r_v - \left(\mu_r - \frac{\sigma_r\beta}{\gamma}\right)\right) \label{eq:ch6:bnrelu_soft}
\end{align}
then if $t' = \mu_v - \frac{\sigma_r\beta}{\gamma} > 0$, \textbf{doing batch
normalization followed by a ReLU on the magnitude of the complex coefficients is the
same as soft shrinkage with threshold $t'$, scaled by a factor
$\frac{\gamma}{\sigma_r}$}.

The same analogy does not apply to the lowpass
coefficients, as $v_{lp}$ is not strictly positive.

While soft thresholding is similar to batch normalizations and ReLUs, we would also like
to test how well it performs as a wavelet nonlinearity.
To do this, we can:
\begin{itemize}
  \item Learn the threshold $t$
  \item Adapt $t$ as a function of the distribution of activations to achieve a desired sparsity level.
\end{itemize}
In early experiments, we found that trying to set
desired sparsity levels by tracking the standard deviation of the statistics
and setting a threshold as a function of it performed very poorly (causing a
drop in top-1 accuracy of at least 10\%).
Instead, we choose to learn a threshold $t$. We make this an unconstrained
optimization problem by changing \eqref{eq:ch6:relu_st} to:
\begin{equation}
  \mathcal{S}(v, t) = \max(0, r-|t|)e^{j\theta}  \label{eq:ch6:relu_st2}
\end{equation}

Learning a threshold is only possible for soft thresholding, as $\dydx{L}{t}$ is
not defined for hard thresholding. Like batch normalization, we learn
independent thresholds $t$ for each channel.
