\section{Implementation Details}

We use our PyTorch Wavelets package \cite{cotter_pytorch_2018} to perform the $\DTCWT$. As
mentioned in \autoref{sec:ch6:multiple_subbands}, our layer requires a set
of learnable filters:
\begin{equation}
  \{ g_{lp}, g_{j,k} \}_{1\leq j \leq J, 1\leq k\leq 6}
\end{equation}
where each of these filters is a four dimensional array as is standard for a
normal convolutional layer. The lowpass coefficients of a $\DTCWT$ are real, and
therefore, $g_{lp}$ is also real. In contrast, the bandpass coefficients are
complex, and so to get the full use of the phase information of the $\DTCWT$, we
also learn complex bandpass filters. This comes with an overhead which we will 
fully quantify in a later section.

We must choose the spatial sizes of both the lowpass and bandpass
mixing kernels. In our work, we set the spatial support of the complex bandpass
filters $g_{j,k}$ to be $1\x 1$ and the lowpass filters $g_{lp}$ to be $3\x 3$.
Further, we limit ourselves initially to only considering a single scale
transform.  
If we wish, we can learn larger spatial sizes to have more complex
attenuation/magnification of the subbands. We also can use more than
one wavelet scale. 

\begin{figure}[t!]
  \centering
  \small
  \resizebox{\textwidth}{!} {%
  \begin{tikzpicture}[%
    path image/.style={%
      path picture={%
        \node at (path picture bounding box.center) {%
          \includegraphics[height=2.0cm]{#1}
        };
      }
    }, 
    path pic/.style={%
      path picture={%
        \node at (path picture bounding box.center) {%
          \includegraphics[height=0.8cm]{#1}
        };
      }
    }, 
    scale=0.6]
    \tikzcuboid{
    shiftx=-1.5cm,
    shifty=-2.5cm,
    scale=0.5,
    anglex=0, 
    angley=90, 
    anglez=230,
    dimx=3, 
    dimy=3, 
    dimz=6,
    densityx=1, 
    densityy=1, 
    densityz=1,
    shade=false,
    emphedge=true,
    shadeopacity=0,
    emphstyle/.style={rounded corners=0.2pt,line width=0.3mm},
    front/.style={draw=blue!50!white,fill=blue!50!white},%
    right/.style={draw=blue!50!white,fill=blue!50!white},%
    top/.style={draw=blue!50!white,fill=blue!50!white},%
    drawxdims=true,
    dimxval=W,
    drawydims=true,
    dimyval=H,
    drawzdims=true,
    dimzval=C_l,
    }
    \draw (0, .3, 0) node {\large{$x^{(l)}$}};
    \draw [path image=\imgpath/waveys.png, draw=black] (1.5,1.5,0) rectangle (6,0,0);
    \draw (3.75, 1.9, 0) node {\large{$\psi_{j, \theta}$}};
    \draw (1.5,0,0) -- (3,-1.7,0);
    \draw (6,0,0) -- (3.3,-1.7,0);
    \draw [path pic=\imgpath/lowpass.png, draw=black] (2.5,-2.5,10) rectangle (3.5,-1.5,10);
    \draw (3, -1.1, 10) node {\large{$\phi_{j}$}};
    \draw (3.5,-1.5,10) -- (3.5,-1.7,8);

    % \draw (2.4, -1.5, 0) node {\Large{$\conv$}};
    \draw (2.5, -1.5, 3) node {\Large{$\conv$}};
    \draw [->, fill=gray!30,ultra thick] (4, -1.5, 0) -- (5, -1.5, 0);
    \draw [->, fill=gray!30,ultra thick] (4.5, -1.5, 9) -- (5.85, -1.5, 11);

    % Draw the 6 subband activations and their filters
    \tikzcuboid{
    shiftx=3cm,
    shifty=-1.7cm,
    shiftz=0,
    scale=0.3,
    dimx=1, dimy=1, dimz=4,
    densityx=2, densityy=2, densityz=2,
    drawxdims=false,
    drawydims=false,
    drawzdims=false,
    dimzval=12,
    front/.style={draw=yellow!70!white,fill=yellow!70!white},%
    right/.style={draw=yellow!70!white,fill=yellow!70!white},%
    top/.style={draw=yellow!70!white,fill=yellow!70!white},%
    }
    \tikzcuboid{
    shiftz=8/0.3,
    scale=0.3,
    dimx=1, dimy=1, dimz=1,
    densityx=2, densityy=2, densityz=2,
    drawxdims=false,
    drawydims=false,
    drawzdims=false,
    }

    % Draw the 6 subband activations and their filters
    \foreach \i in {1,...,5} {
      \tikzcuboid{
      shiftx=6cm,
      shifty=-2.0cm,
      shiftz=-15+5*\i,
      scale=0.5,
      dimx=2, dimy=2, dimz=3,
      densityx=4, densityy=4, densityz=2,
      drawxdims=false,
      drawydims=false,
      dimzval=C_l,
      drawzdims=false,
      front/.style={draw=blue!50!white,fill=blue!50!white},%
      right/.style={draw=blue!50!white,fill=blue!50!white},%
      top/.style={draw=blue!50!white,fill=blue!50!white},%
      }
      \tikzcuboid{
      shiftx=12.0cm,
      }
      \draw [->, fill=gray!30,ultra thick] (10.0+0.05*\i, -1.5, -7.5+2.5*\i) -- (11.0+0.05*\i, -1.5, -7.5+2.5*\i)
        node[midway, above] {$\sigma_w$};

      \tikzcuboid{
      shiftx=8.5cm,
      shifty=-1.0cm,
      shiftz=-15+5*\i,
      dimx=0.4, dimy=0.4,
      densityx=5, densityy=5, densityz=2,
      front/.style={draw=red!70!white,fill=red!70!white},%
      right/.style={draw=red!70!white,fill=red!70!white},%
      top/.style={draw=red!70!white,fill=red!70!white},%
      }
      \tikzcuboid{
      shifty=-1.5cm,
      }
      \tikzcuboid{
      shifty=-2.0cm,
      }
      \tikzcuboid{
      shifty=-2.5cm,
      }
      % \draw (8cm, -1.8cm, -15+5*\i) node {$\vdots$};
    }
    \tikzcuboid{
    shiftx=6cm,
    shifty=-2.0cm,
    shiftz=15,
    scale=0.5,
    dimx=2, dimy=2, dimz=3,
    densityx=4, densityy=4, densityz=2,
    dimxval=\frac{W}{2},
    drawxdims=true,
    dimyval=\frac{H}{2},
    drawydims=true,
    dimzval=C_l,
    drawzdims=true,
    front/.style={draw=blue!50!white,fill=blue!50!white},%
    right/.style={draw=blue!50!white,fill=blue!50!white},%
    top/.style={draw=blue!50!white,fill=blue!50!white},%
    }
    \tikzcuboid{
    shiftx=12.0cm,
    }
    \draw [->, fill=gray!30,ultra thick] (10+6*0.05, -1.5, 7.5) -- (11.0+6*0.05, -1.5, 7.5)
      node[midway, above] {$\sigma_w$};
    \tikzcuboid{
    shiftx=8.5cm,
    shifty=-1.0cm,
    shiftz=15,
    dimx=0.4, dimy=0.4,
    drawxdims=false,
    drawydims=false,
    drawzdims=false,
    densityx=5, densityy=5, densityz=2,
    front/.style={draw=red!70!white,fill=red!70!white},%
    right/.style={draw=red!70!white,fill=red!70!white},%
    top/.style={draw=red!70!white,fill=red!70!white},%
    }
    \tikzcuboid{
    shifty=-1.5cm,
    }
    \tikzcuboid{
    shifty=-2.0cm,
    }
    \tikzcuboid{
    shifty=-2.5cm,
    }

    % Draw the lowpass activation and its set of filters
    \tikzcuboid{
    shiftx=6cm,
    shifty=-2.0cm,
    shiftz=35,
    scale=0.5,
    dimx=3, dimy=3, dimz=4,
    densityx=4, densityy=4, densityz=2,
    dimxval=W,
    drawxdims=true,
    dimyval=H,
    drawydims=true,
    dimzval=C_l,
    drawzdims=true,
    front/.style={draw=blue!50!white,fill=blue!50!white},%
    right/.style={draw=blue!50!white,fill=blue!50!white},%
    top/.style={draw=blue!50!white,fill=blue!50!white},%
    }
    \tikzcuboid{
    shiftx=12.0cm,
    }
    \draw [->, fill=gray!30,ultra thick] (10.+0.5, -1.5, 17.5) -- (11.0+0.5, -1.5, 17.5)
      node[midway, above] {$\sigma_w$};
    \tikzcuboid{
    shiftx=8.5cm,
    shifty=-2.75cm,
    dimx=0.8, dimy=0.8,
    drawxdims=false,
    drawydims=false,
    drawzdims=false,
    densityx=5, densityy=5, densityz=2,
    front/.style={draw=red!50!white,fill=red!50!white},%
    right/.style={draw=red!50!white,fill=red!50!white},%
    top/.style={draw=red!50!white,fill=red!50!white},%
    }
    \tikzcuboid{
    shifty=-2.00cm,
    }
    \tikzcuboid{
    shifty=-1.25cm,
    }
    \tikzcuboid{
    shifty=-0.5cm,
    }
    % \tikzcuboid{
    % shiftx=6cm,
    % shifty=-2cm,
    % shiftz=0cm,
    % scale=0.5,
    % dimx=2, dimy=2, dimz=24,
    % densityx=2, densityy=2, densityz=2,
    % drawxdims=true,
    % dimxval=\frac{W}{2},
    % drawydims=true,
    % dimyval=\frac{H}{2},
    % drawzdims=true,
    % dimzval=12C_l,
    % front/.style={draw=blue!50!white,fill=blue!50!white},%
    % right/.style={draw=blue!50!white,fill=blue!50!white},%
    % top/.style={draw=blue!50!white,fill=blue!50!white},%
    % }

    % % \draw [->, fill=gray!30,ultra thick] (8.5, -1.5, 0) -- (10.5, -1.5, 0) node[midway, above] (mag) {\large{$\lvert\cdot\rvert$}};
    % \draw [->, fill=gray!30,ultra thick] (8.5, -1.5, 0) -- (10.5, -1.5, 0) node[midway, above] (mag) {\Large{ $\lvert \cdot \rvert$} };
    % \draw [path pic=\imgpath/mag.png, draw=white] (9.25,0,0) rectangle (11.75,2,0);
    % \draw (9.25,0,0) -- (mag.north);
    % \draw (11.75,0,0) -- (mag.north);
    % \draw[->, fill=gray!30, ultra thick] (8.5, -2, 10) -- (10, -3, 3);

    % \tikzcuboid{
    % shiftx=11.5cm,
    % shifty=-2cm,
    % scale=0.5,
    % dimx=2, dimy=2, dimz=12,
    % densityx=2, densityy=2, densityz=2,
    % dimzval=6C_l,
    % drawxdims=false,
    % drawydims=false,
    % front/.style={draw=blue!50!white,fill=blue!50!white},%
    % right/.style={draw=blue!50!white,fill=blue!50!white},%
    % top/.style={draw=blue!50!white,fill=blue!50!white},%
    % }
    % \tikzcuboid{
    % shiftx=11.5cm,
    % shifty=-2cm,
    % shiftz=8,
    % scale=0.5,
    % drawxdims=true,
    % drawydims=true,
    % dimx=2, dimy=2, dimz=4,
    % densityx=2, densityy=2, densityz=2,
    % dimzval=C_l,
    % front/.style={draw=blue!50!white,fill=blue!50!white},%
    % right/.style={draw=blue!50!white,fill=blue!50!white},%
    % top/.style={draw=blue!50!white,fill=blue!50!white},%
    % }
    % \draw (13.2, 0.8, 0) node {\large{$z^{(l+1)}$}};
    % \draw (20.5, 0.3, 0) node {\large{$x^{(l+1)}$}};
    % \draw (14, -1.5, 0) node {\Large{$\conv$}};

    % \tikzcuboid{
    % shiftx=15.2cm,
    % shifty=-1.0cm,
    % shiftz=0,
    % scale=0.5,
    % dimx=0.4, dimy=0.4, dimz=14,
    % densityx=5, densityy=5, densityz=2,
    % drawxdims=false,
    % drawydims=false,
    % drawzdims=false,
    % front/.style={draw=red!50!white,fill=red!50!white},%
    % right/.style={draw=red!50!white,fill=red!50!white},%
    % top/.style={draw=red!50!white,fill=red!50!white},%
    % }
    % \tikzcuboid{
    % shifty=-1.65cm,
    % }
    % \tikzcuboid{
    % shifty=-3.0cm,
    % scale=0.5,
    % drawxdims=true,
    % dimxval=1,
    % drawydims=true,
    % dimyval=1,
    % drawzdims=true,
    % dimzval=7C_l,
    % }
    % \draw (15.5, -1.8, 0) node {$\vdots$};
    % \draw [<->] (15.7, -0.8, -3) -- (15.7, -3, -3) node[near start, right] {$C_{l+1}$};
    % \draw [->, fill=gray!30,ultra thick] (17.5, -1.5, 0) -- (18.5, -1.5, 0)
      % node[midway, above] {$\sigma$};

    % \tikzcuboid{
    % shiftx=19.5cm,
    % shifty=-2.0cm,
    % scale=0.5,
    % dimx=2, dimy=2, dimz=6,
    % densityx=4, densityy=4, densityz=2,
    % drawzdims=true,
    % dimzval=C_{l+1},
    % drawxdims=true,
    % dimxval=\frac{W}{2},
    % drawydims=true,
    % dimyval=\frac{H}{2},
    % front/.style={draw=blue!50!white,fill=blue!50!white},%
    % right/.style={draw=blue!50!white,fill=blue!50!white},%
    % top/.style={draw=blue!50!white,fill=blue!50!white},%
    % }

  \end{tikzpicture}}
  \mycaption{Block Diagram of proposed $\DTCWT$ gain layer for $J=1$}{Activations are shaded
  blue, fixed parameters yellow and learned parameters red. Darker shades of
  blue and red indicate complex valued activations and weights, whereas the
  lighter shades indicate purely real values. The input
  $x^{(l)}\in \mathbb{R}^{C_l\x H\x W}$ is taken into the wavelet domain and each
  subband is mixed independently with $C_{l+1}$ sets of convolutional filters.
  After mixing, a possible wavelet 
  based nonlinearity $\sigma_w$ is applied to the subbands, before returning to
  the pixel domain with an inverse $\DTCWT$.}
  \label{fig:ch6:block_diagram}
\end{figure}


. 

\subsection{Parameter Memory Cost}\label{sec:ch6:memory}
A standard convolutional layer with $C_l$ input channels, $C_{l+1}$ output channels
and kernel size $L\x L$ has $L^2C_{l}C_{l+1}$ parameters, with $L=3$ or $L=5$
common choices for the spatial size.

The number of learnable parameters in our proposed gain layer with
$J=1$, $K=6$, kernel spatial size 3 for the lowpass and 1 for the bandpass:
%
\begin{equation}
  \text{\#params} = (2JK + 3^2)C_lC_{l+1} = 21C_lC_{l+1} \label{eq:ch6:memcost}
\end{equation} 
%
At first glance, we have increased our parameterization by over twofold on a $3\x 3$ 
convolution, but as with the invariant layer, the spatial support of the full
filter is larger than an equivalent one parameterized in the filter domain. Note
that using a second scale with $1\x 1$ filters would increase
\eqref{eq:ch6:memcost} to $33C_lC_{l+1}$.

\subsection{Activation Memory Cost}\label{sec:ch6:act_memory}
A standard convolutional layer needs to save the activation $x^{(l)}$ to
convolve with the backpropagated gradient $\dydx{L}{y^{(l+1)}}$ on the backwards
pass (to give $\dydx{L}{w^{(l)}}$). For an input with $C_l$ channels of spatial
size $H\x W$, this means $HWC_l$ floats must be saved. 

Our layer requires us to save the wavelet coefficients from \eqref{eq:ch6:wave_coeffs}, 
$\{u_{lp}, u_{j,k}\}_{1\leq j \leq J, 1}$ for updating the $g$ terms 
\eqref{eq:ch6:gr_update}, \eqref{eq:ch6:gi_update}.
These are $4:1$ redundant, so require $4HWC_l$ floats to be saved.

If memory is an issue and the computation of the $\DTCWT$ is very fast, then we
only need to save the $x^(l)$ coefficients and can calculate the $u$'s on the
fly during the backwards pass. 

\subsection{Computational Cost}\label{sec:ch6:computation}
A standard convolutional layer with kernel size $L\x L$ needs $L^2C_{l+1}$
multiplies per input pixel (of which there are $C_{l}\x H\x W$).

There is an overhead in doing the wavelet decomposition for each input channel.
A separable 2-D discrete wavelet transform (DWT) with 1-D filters of length $L$
will have $2L\left(1-2^{-2J}\right)$ multiplies per input pixel for a $J$ scale
decomposition. A $\DTCWT$ has 4 DWTs for a 2-D input, so its cost is
$8L\left(1-2^{-2J}\right)$, with $L=6$ a common size for the filters. It is
important to note that unlike the filtering operation, this does not scale with
$C_{l+1}$, the end result being that as $C_{l+1}$ grows, the cost of $C_l$
forward transforms is outweighed by that of the mixing process whose cost is
proportional to $C_l C_{l+1}$.

Because we are using a decimated wavelet decomposition, the sample rate
decreases after each wavelet layer. The benefit of this is that the mixing
process then only works on one quarter the spatial size after the first scale
and one sixteenth the spatial after the second scale. Restricting ourselves to
$J=1$ as we mentioned in \autoref{sec:implementation}, the computational cost is
then:

\begin{equation}
  % \frac{7}{4}C_{l+1} + 48 \label{eq:comp}
  \underbrace{ \frac{7}{4}C_{l+1} }_{\textrm{mixing}} +
  \underbrace{\vphantom{\frac{7}{4}} 36}_{\DTCWT} \quad
  \textrm{multiplies per input pixel}\label{eq:ch5:comp}
\end{equation}
In most CNNs, $C_{l+1}$ is several dozen if not several
hundred, which makes \eqref{eq:ch5:comp} significantly smaller than
$L^2C_{l+1}=9C_{l+1}$ multiplies for $3\x 3$ convolutions.

\subsection{Forward and Backward Algorithm}
There are two layer hyperparameters to choose in our layer:
\begin{itemize}
  \item The number of output channels $C_{l+1}$. This may be restricted by the
    architecture.
  \item The variance of the weight initialization for the mixing matrix $A$.
\end{itemize}

Assuming we have already chosen these values, 
then the forward and backward algorithms can be computed with
Algorithm~\autoref{alg:ch5:inv}.


\subsection{Parameter Memory Cost}
A standard 
Again considering a two scale transform --- instead of learning $w\in \reals[F
\x C\x K\x K]$ we learn complex gains at the two scales, and a real gain for the
real lowpass:

\begin{eqnarray*}
  g_1 &\in& \complexes[F\x C\x 6\x 1\x 1] \\
  g_2 &\in& \complexes[F\x C\x 6\x 1\x 1] \\
  g_{lp} &\in& \reals[F\x C\x 1\x 1]
\end{eqnarray*}

We have set the spatial dimension to be $1\x 1$ to show that this gain is
identical to a $1\x 1$ convolution over the complex wavelet coefficients. If we
wish, we can learn larger spatial sizes to have more complex
attenuation/magnification of the subbands. We also can use more/fewer than
2 wavelet scales.  At first glance, we have increased our parameterization by
a factor of 25 (13 subbands, of which all but the lowpass are complex), but
each one of these gains affects a large spatial size. For the first scale, the
effective size is about $5\x 5$ pixels, for the second scale it is about $15\x
15$.

\subsection{Computational Cost}
A standard convolutional layer needs $K^2 F$ multiplies per input pixel (of
which there are $C\x H\x W$). In comparison, the wavelet gain method does a set
number of operations per pixel for the forward and inverse transforms, and then
applies gains on subsampled activations. For a 2 level $\DTCWT$ the transform
overhead is about 60 multiplies for both the forward and inverse transform. It
is important to note that unlike the filtering operation, this does not scale
with $F$. The learned gains in each subband do scale with the number of output
channels, but can have smaller spatial size (as they have larger effective
sizes) as well as having fewer pixels to operate on (because of the decimation).
The end result is that as $F$ and $C$ grow, the overhead of the $C$ forward and
$F$ inverse transforms is outweighed by cost of $FC$ mixing processes, which
should in turn be significantly less than the cost of $FC$ $K\x K$ standard
convolutions for equivalent spatial sizes.


\subsection{Examples}
\autoref{fig:ch6:example_impulses} show example impulse responses of our layer.
These impulses were generated by randomly initializing both the real and
imaginary parts of $g_2 \in \complexes[6\x 1\x 1]$ from $\mathcal{N}(0,1)$ and
$g_1, g_{lp}$ are set to 0. I.e. each shape has 12 random variables. It is good
to see that there is still a large degree of variability between shapes. Our
experiments have shown that the distribution of the normalized cross-correlation
between 512 of such randomly generated shapes matches the distribution for
random vectors with roughly 11.5 degrees of freedom.

\begin{algorithm}[t]
\caption{$\DTCWT$ gain layer forward and backward passes}\label{alg:ch6:gain}
\begin{algorithmic}[1]
\Procedure{GAINFWD}{$x, w_l, $}
  \State $yl,\ yh \gets \DTCWT(x^l, \mbox{nlevels}=1) $ \Comment{yh has 6
  orientations and is complex}
  \State $U \gets \F{COMPLEXMAG}(yh)$
  \State $yl \gets \F{AVGPOOL2x2}(yl)$  \Comment{Downsample and recentre lowpass
  to match U size}
  \State $Z \gets \F{CONCATENATE}(yl,\ U)$ \Comment{concatenated along the
  channel dimension}
  \State $Y \gets AZ$ \Comment{Mix}
  \State \textbf{save} $Z$ \Comment{For the backwards pass}
  \State \textbf{return} $Y$ 
\EndProcedure
\end{algorithmic}\vspace{10pt}
\begin{algorithmic}[1]
\Procedure{GAINBWD}{$\dydx{L}{Y},\ A$}
  \State \textbf{load} $Z$
  \State $\dydx{L}{A} \gets \dydx{L}{Y} Z^T$ \Comment{The weight gradient}
  \State $\Delta{Z} \gets A^T\dydx{L}{Y}$
  \State $\Delta yl,\ \Delta U \gets \F{UNSTACK}(\Delta Z)$ 
  \State $\Delta yl \gets \F{AVGPOOL2x2BWD}(\Delta yl)$
  \State $\Delta yh \gets \F{COMPLEXMAGBWD}(\Delta U)$
  \State $\dydx{L}{x} \gets \F{\DTCWT BWD}(\Delta yl,\ \Delta yh)$ \Comment{The propagated gradient}
  \State \textbf{return} $\dydx{L}{x},\ \dydx{L}{A}$
\EndProcedure
\end{algorithmic}
\end{algorithm}

