\section{Implementation Details}

We use our PyTorch Wavelets package \cite{cotter_pytorch_2018} to perform the $\DTCWT$. As
mentioned in \autoref{sec:ch6:multiple_subbands}, our layer requires a set
of learnable filters:
\begin{equation}
  \{ g_{lp}, g_{j,k} \}_{1\leq j \leq J, 1\leq k\leq 6}
\end{equation}
where each of these filters is a four dimensional array as is standard for a
normal convolutional layer. The lowpass coefficients of a $\DTCWT$ are real, and
therefore, $g_{lp}$ is also real. In contrast, the bandpass coefficients are
complex, and so to get the full use of the phase information of the $\DTCWT$, we
also learn complex bandpass filters. This comes with an overhead which we will 
fully quantify in a later section.

We must choose the spatial sizes of both the lowpass and bandpass
mixing kernels. In our work, we set the spatial support of the complex bandpass
filters $g_{j,k}$ to be $1\x 1$ and the lowpass filters $g_{lp}$ to be $3\x 3$.
Further, we limit ourselves initially to only considering a single scale
transform.  
If we wish, we can learn larger spatial sizes to have more complex
attenuation/magnification of the subbands. We also can use more than
one wavelet scale. 


\subsection{Parameter Memory Cost}\label{sec:ch6:memory}
A standard convolutional layer with $C_l$ input channels, $C_{l+1}$ output channels
and kernel size $L\x L$ has $L^2C_{l}C_{l+1}$ parameters, with $L=3$ or $L=5$
common choices for the spatial size.

Returning to \eqref{eq:ch6:glp} -- \eqref{eq:ch6:gj6}, let us consider a single
scale ($J=1$) wavelet gain layer. We will keep the spatial size of all the bandpass
kernels the same, and allow for a differently sized lowpass gain. The filters
are then:

\begin{eqnarray*}
  g_{lp} &\in& \reals[F\x C\x k_{lp}\x k_{lp}] \\
  g_{1,1} &\in& \complexes[F\x C\x k_{bp}\x k_{bp}] \\
  g_{1,2} &\in& \complexes[F\x C\x k_{bp}\x k_{bp}] \\
  g_{1,3} &\in& \complexes[F\x C\x k_{bp}\x k_{bp}] \\
  g_{1,4} &\in& \complexes[F\x C\x k_{bp}\x k_{bp}] \\
  g_{1,5} &\in& \complexes[F\x C\x k_{bp}\x k_{bp}] \\
  g_{1,6} &\in& \complexes[F\x C\x k_{bp}\x k_{bp}] 
\end{eqnarray*}
%
or more compactly:
%
\begin{eqnarray*}
  g_{lp} &\in& \reals[F\x C\x k_{lp}\x k_{lp}] \\
  \vec{g}_{1} &\in& \complexes[6\x F\x C\x k_{bp}\x k_{bp}] 
\end{eqnarray*}

We typically set $k_{bp} = 1$ to avoid large parameter costs, but allow $k_{lp}$
to be 1 or 3. For $k_{lp} = 3$, the total parameter cost is then (requiring 2
floats for complex values):
%
\begin{equation}
  \text{\#params} = (2\x 6 + 3^2)C_lC_{l+1} = 21C_lC_{l+1} \label{eq:ch6:memcost}
\end{equation} 
%
At first glance, we have increased our parameterization by over twofold on a $3\x 3$ 
convolution, but as with the invariant layer, the spatial support of the full
filter is larger than an equivalent one parameterized in the filter domain. Note
that using a second scale ($J=2$) with $1\x 1$ filters would increase
\eqref{eq:ch6:memcost} to $33C_lC_{l+1}$. 

Compared to a single scale invariant layer from the previous chapter, this requires
$21C_lC_{l+1} - 7C_lC_{l+1} = 14C_lC_{l+1}$ more parameters, or a threefold
increase. $6C_lC_{l+1}$ of these come from learning a complex gain rather than a
purely real gain for the bandpass, and $8C_lC_{l+1}$ come from using a larger
kernel on the lowpass. 

If we were to set $k_{lp} = 1$, we would minimize the 
relative parameter cost increase to be slightly
less than $2:1$ on the invariant layer.

\subsection{Activation Memory Cost}\label{sec:ch6:act_memory}
A standard convolutional layer needs to save the activation $x^{(l)}$ to
convolve with the backpropagated gradient $\dydx{L}{y^{(l+1)}}$ on the backwards
pass (to give $\dydx{L}{w^{(l)}}$). For an input with $C_l$ channels of spatial
size $H\x W$, this means $HWC_l$ floats must be saved. 

Our layer requires us to save the wavelet coefficients from \eqref{eq:ch6:wave_coeffs}, 
$\{u_{lp}, u_{j,k}\}_{1\leq j \leq J, 1}$ for updating the $g$ terms 
as in \eqref{eq:ch6:gr_update}, \eqref{eq:ch6:gi_update}.
These are $4:1$ redundant, so require $4HWC_l$ floats to be saved.

Note that a single scale layer requires $16/7$ times as many floats to be saved
as compared to the invariant layer of the previous chapter. The extra cost of
this comes from two things. Firstly, we keep the real and imaginary components
for the bandpass (as opposed to only the magnitude), meaning we need $3HWC_l$
floats, rather than $\frac{3}{2}HWC_l$. Additionally, the lowpass was
downsampled in the previous chapter, requiring only $\frac{1}{4}HWC_l$, whereas
we keep the full sample rate costing $HWC_l$.

If memory is an issue and the computation of the $\DTCWT$ is very fast, then we
only need to save the $x^(l)$ coefficients and can calculate the $u$'s on the
fly during the backwards pass. 

\subsection{Computational Cost}\label{sec:ch6:computation}
A standard convolutional layer with kernel size $L\x L$ needs $L^2C_{l+1}$
multiplies per input pixel (of which there are $C_{l}\x H\x W$).

The overhead calculations are the same as in \autoref{sec:ch5:computation}, so
we will omit their derivation here. The mixing is however different, requiring
complex convolution for the bandpass coefficients, and convolution over a higher
resolution lowpass. The bandpass has one quarter spatial resolution at the first
scale, but this is offset by the $4:1$ cost of complex multiplies compared to 
real multiplies. Again assuming we have set $J=1$ and $k_{lp} = 3$ then the total
cost for the gain layer is:

\begin{equation}
  % \frac{7}{4}C_{l+1} + 48 \label{eq:comp}
  \underbrace{\hphantom{1} \frac{6\x 4}{4}C_{l+1} \hphantom{1}}_{\textrm{bandpass}} +
  \underbrace{\hphantom{1}\vphantom{\frac{6}{4}} 3^2 C_{l+1} \hphantom{1}}_{\textrm{lowpass}} + 
  \underbrace{\vphantom{\frac{6}{4}} 36}_{\DTCWT} + 
  \underbrace{\vphantom{\frac{6}{4}} 36}_{\DTCWT^{-1}} = \quad 15C_{l+1} + 72 \quad
  \textrm{mults/pixel}\label{eq:ch5:comp}
\end{equation}

This is larger than both a standard $3\x 3$ convolutional layer, using
$9C_{l+1}$ multiplies per input pixel, and the invariant layer of the previous
chapter which uses $\frac{7}{4}C_{l+1} + 36}$ multiplies per input pixel. 

\subsection{Parameter Initialization}

\subsection{Forward and Backward Algorithm}
Algorithm~\autoref{alg:ch6:inv}.
\begin{algorithm}[tb]
\caption{Locally Invariant Convolutional Layer forward and backward
passes}\label{alg:ch6:inv}
\begin{algorithmic}[1]
\Procedure{GAINLAYERFWD}{$x,\ g_{lp},\ g_1$}
\State $u_{lp},\ u_{1} \gets \DTCWT(x^l, \mbox{nlevels}=1) $ \Comment{$u_{1}$ has 6 orientations and is complex}
  \State \textbf{save} $u_{lp},\ u_{1}$ \Comment{For the backwards pass}
  \State $v_{lp} \gets u_{lp} \conv g_{lp}$ 
  \For{$1\leq k \leq 6$}
  \State \begin{varwidth}[t]{\linewidth}  $v_{1,k} \gets 
      \left( \real{u_{1,k}} \conv \real{g_{1,k}} - \imag{u_{1,k}} \conv \imag{g_{1,k}} \right)\ +$ \par
      \hskip\algorithmicindent \hphantom{abc}$ j\left(\real{u_{1,k}} \conv \imag{g_{1,k}} + \imag{u_{1,k}} \conv \real{g_{1,k}}\right)$;
  \end{varwidth}
  \EndFor
  \State $y \gets \DTCWT^{-1}(v_{lp}, v_{1})$
  \State \textbf{return} $y$
\EndProcedure
\end{algorithmic}\vspace{10pt}
\begin{algorithmic}[1]
  \Procedure{GAINLAYERBWD}{$\dydx{L}{Y},\ g_{lp},\ g_1$}
  \State \textbf{load} $u_{lp},\ u_1$
  \State \textbf{return} $\dydx{L}{x},\ \dydx{L}{A}$
\EndProcedure
\end{algorithmic}
\end{algorithm}


