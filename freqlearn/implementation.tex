\subsection{Implementation Details}

Before analyzing its performance, we compare the implementation properties of
our two new proposed layers (for the DWT and $\DTCWT$) to a standard
convolutional layer.

\subsubsection{Parameter Memory Cost}\label{sec:ch6:memory}
A standard convolutional layer with $C_l$ input channels, $C_{l+1}$ output channels
and kernel size $L\x L$ has $L^2C_{l}C_{l+1}$ parameters, with $L=3$ or $L=5$
common choices for the spatial size.
\begin{equation}
  \text{\#conv params} = 9C_lC_{l+1}
\end{equation}

We must choose the spatial sizes of both the lowpass and bandpass
mixing kernels. In our work, we set:
%
\begin{itemize}
  \item The spatial support of the lowpass filters for the DWT gain 
    layer to be $3\x 3$ and the support of the bandpass layers to be $1\x 1$.
  \item The spatial support of the lowpass filters for the $\DTCWT$ gain layer to
    be $1\x 1$ and the support of the complex bandpass filters to be $1\x 1$.
\end{itemize}
%
Further, we limit ourselves initially to only considering a single scale
transform.  If we wish, we can learn larger spatial sizes to have more complex
attenuation/magnification of the subbands. We also can use more than one wavelet
scale. 

This means that for the DWT layer, the number of parameters is:
%
\begin{equation}
  \text{\#DWT params} = (3 + 3^2)C_lC_{l+1} = 12C_lC_{l+1} \label{eq:ch6:memcost1}
\end{equation} 
%
And for the $\DTCWT$ layer, the number of parameters is:
\begin{equation}
  \text{\#$\DTCWT$ params} = (2\x 6 + 1)C_lC_{l+1} = 13C_lC_{l+1} \label{eq:ch6:memcost2}
\end{equation} 
%
These are both slightly larger than the $9C_lC_{l+1}$ parameters used in a
standard $3\x 3$ convolution, but as \autoref{fig:ch6:examples} shows, the
spatial support of the full filter is larger than an equivalent one
parameterized in the filter domain. 

Note that using a second scale ($J=2$) with $1\x 1$ filters would increase
\eqref{eq:ch6:memcost1} to $15C_lC_{l+1}$ and \eqref{eq:ch6:memcost2} to
$25C_lC_{l+1}$.  

\subsubsection{Activation Memory Cost}\label{sec:ch6:act_memory}
A standard convolutional layer needs to save the activation $x^{(l)}$ to
convolve with the backpropagated gradient $\dydx{L}{y^{(l+1)}}$ on the backwards
pass (to give $\dydx{L}{w^{(l)}}$). For an input with $C_l$ channels of spatial
size $H\x W$, this means
%
\begin{equation}
  \text{\#conv floats} = HWC_l 
\end{equation}

Our layers require us to save the wavelet coefficients $u_{lp}$ and  $u_{j,k}$
for updating the $g$ terms as in \eqref{eq:ch6:g_update} and
\eqref{eq:ch6:gr_update}, \eqref{eq:ch6:gi_update}.  For the critically sampled
DWT, this requires:
%
\begin{equation}
  \text{\#DWT floats} = HWC_l 
\end{equation}
%
to be saved for the backwards pass. For the $4:1$ redundant $\DTCWT$, this 
requires:
%
\begin{equation}
  \text{\#$\DTCWT$ floats} = 4HWC_l 
\end{equation}
%
to be saved for the backwards pass.  You can see this difference from the
difference in the block diagrams in \autoref{fig:ch6:block_diagrams}.

Note that a single scale $\DTCWT$ gain layer requires $16/7$ times as many
floats to be saved as compared to the invariant layer of the previous chapter.
The extra cost of this comes from two things. Firstly, we keep the real and
imaginary components for the bandpass (as opposed to only the magnitude),
meaning we need $3HWC_l$ floats, rather than $\frac{3}{2}HWC_l$. Additionally,
the lowpass was downsampled in the previous chapter, requiring only
$\frac{1}{4}HWC_l$, whereas we keep the full sample rate costing $HWC_l$.

If memory is an issue and the computation of the $\DTCWT$ is very fast, then we
only need to save the $x^(l)$ coefficients and can calculate the $u$'s on the
fly during the backwards pass. Note that a two scale $\DTCWT$ gain layer would
still only require $4HWC_l$ floats.

\subsubsection{Computational Cost}\label{sec:ch6:computation}
A standard convolutional layer with kernel size $L\x L$ needs $L^2C_{l+1}$
multiplies per input pixel (of which there are $C_{l}\x H\x W$).

For the DWT with Daubechies 2 filters, the forward and inverse transform only
require about $6$ multiplies per input pixel. The mixing is then done at a
reduced spatial resolution. For our proposed kernel sizes of $3\x 3$ for the
lowpass and $1\x 1$ for the bandpass, the DWT gain layer requires:

\begin{equation}
  % \frac{7}{4}C_{l+1} + 48 \label{eq:comp}
  \text{\#DWT mults/pixel} = \underbrace{\hphantom{1} \frac{3}{4}C_{l+1} \hphantom{1}}_{\textrm{bandpass}} +
  \underbrace{\hphantom{1}\vphantom{\frac{3}{4}} \frac{3^2}{4} C_{l+1} \hphantom{1}}_{\textrm{lowpass}} + 
  \underbrace{\vphantom{\frac{3}{4}} 6}_{\text{DWT}} + 
  \underbrace{\vphantom{\frac{3}{4}} 6}_{\text{DWT}^{-1}} = \quad 4C_{l+1} + 12 \quad
  \label{eq:ch6:comp_dwt}
\end{equation}

This is smaller than a standard $3\x 3$ convolutional layer using $9C_{l+1}$
multiplies per pixel.

For the $\DTCWT$, the overhead calculations are the same as in
\autoref{sec:ch5:computation}, so we will omit their derivation here. The mixing
is however different, requiring complex convolution for the bandpass
coefficients, and convolution over a higher resolution lowpass. The bandpass has
one quarter spatial resolution at the first scale, but this is offset by the
$4:1$ cost of complex multiplies compared to real multiplies. Again assuming we
have set $J=1$ and $k_{lp} = 3$ then the total cost for the gain layer is:
%
\begin{equation}
  % \frac{7}{4}C_{l+1} + 48 \label{eq:comp}
  \text{\#$\DTCWT$ mults/pixel} = \underbrace{\hphantom{1} \frac{6\x 4}{4}C_{l+1} \hphantom{1}}_{\textrm{bandpass}} +
  \underbrace{\hphantom{1}\vphantom{\frac{6}{4}} C_{l+1} \hphantom{1}}_{\textrm{lowpass}} + 
  \underbrace{\vphantom{\frac{6}{4}} 36}_{\DTCWT} + 
  \underbrace{\vphantom{\frac{6}{4}} 36}_{\DTCWT^{-1}} = \quad 7C_{l+1} + 72 \quad
  \label{eq:ch6:comp_dtcwt}
\end{equation}
This is marginally smaller than a $3\x 3$ convolutional layer.

\subsubsection{Parameter Initialization}
For both layer types we use the Glorot Initialization scheme \cite{glorot_understanding_2010}
with $a=1$: 
%
\begin{equation}
  g_{ij} \drawnfrom U\left[ -\sqrt{\frac{6}{(C_l + C_{l+1})k^2}},\ \sqrt{\frac{6}{(C_l + C_{l+1})k^2}}\
  \right] \label{eq:ch6:glorot}
\end{equation}
where $k$ is the kernel size.
% \subsection{Forward and Backward Algorithm}
% Algorithm~\autoref{alg:ch6:inv}.
% \begin{algorithm}[tb]
\caption{Locally Invariant Convolutional Layer forward and backward
passes}\label{alg:ch6:inv}
\begin{algorithmic}[1]
\Procedure{GAINLAYERFWD}{$x,\ g_{lp},\ g_1$}
\State $u_{lp},\ u_{1} \gets \DTCWT(x^l, \mbox{nlevels}=1) $ \Comment{$u_{1}$ has 6 orientations and is complex}
  \State \textbf{save} $u_{lp},\ u_{1}$ \Comment{For the backwards pass}
  \State $v_{lp} \gets u_{lp} \conv g_{lp}$ 
  \For{$1\leq k \leq 6$}
  \State \begin{varwidth}[t]{\linewidth}  $v_{1,k} \gets 
      \left( \real{u_{1,k}} \conv \real{g_{1,k}} - \imag{u_{1,k}} \conv \imag{g_{1,k}} \right)\ +$ \par
      \hskip\algorithmicindent \hphantom{abc}$ j\left(\real{u_{1,k}} \conv \imag{g_{1,k}} + \imag{u_{1,k}} \conv \real{g_{1,k}}\right)$;
  \end{varwidth}
  \EndFor
  \State $y \gets \DTCWT^{-1}(v_{lp}, v_{1})$
  \State \textbf{return} $y$
\EndProcedure
\end{algorithmic}\vspace{10pt}
\begin{algorithmic}[1]
  \Procedure{GAINLAYERBWD}{$\dydx{L}{Y},\ g_{lp},\ g_1$}
  \State \textbf{load} $u_{lp},\ u_1$
  \State \textbf{return} $\dydx{L}{x},\ \dydx{L}{A}$
\EndProcedure
\end{algorithmic}
\end{algorithm}


