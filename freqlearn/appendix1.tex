\chapter{Invertible Transforms and Optimization} \label{app:ch6:invertible}
To see this, let us consider the work from \cite{rippel_spectral_2015} where
filters are parameterized in the Fourier domain. 

If we define the DFT as the orthonormal version, i.e.\ let:
$$ U_{ab} = \frac{1}{\sqrt{N}} \exp\{ \frac{-2j\pi ab}{N} \} $$
%
then call $X = \DFT{x}$. In matrix form the 2-D DFT is then:

\begin{eqnarray}
  X &=& \DFT{x} = UxU \\
  x &=& \IFT{X} = U^*YU^* 
\end{eqnarray}

When it comes to gradients, these become:
\begin{eqnarray}
  \dydx{L}{X} &=& U \dydx{L}{x} U = \DFT{\dydx{L}{x}} \label{eq:ch6:dft_grad}\\
  \dydx{L}{x} &=& U^* \dydx{L}{X} U^* = \IFT{\dydx{L}{X}}
\end{eqnarray}

Now consider a single filter parameterized in the DFT and spatial domains
presented with the exact same data and with the same \ltwo\ regularization
$\epsilon$ and learning rate $\eta$. Let
the spatial filter at time $t$ be $\vec{w}_t$, the Fourier-parameterized
filter be $\hat{\vec{w}}_t$, and let 

\begin{equation}
  \hat{\vec{w}}_1 = \F{DFT}\{\vec{w}_1\} \label{eq:ch6:initial_condition}
\end{equation}
%
After presenting both systems with the same minibatch of samples $\mathcal{D}$
and calculating the gradient $\dydx{L}{\vec{w}}$ we update both parameters:

\begin{eqnarray}
  \vec{w}_2 & = & \vec{w}_1 - \eta \left(
    \dydx{L}{\vec{w}} + \epsilon \vec{w}_1 \right) \\
    &=& (1-\eta\epsilon)\vec{w}_1 - \eta \dydx{L}{\vec{w}} \\
  \hat{\vec{w}}_2 & = & \hat{\vec{w}}_1 - \eta \left(
     \dydx{L}{\hat{\vec{w}}} + \epsilon \hat{\vec{w}}_1 \right)  \\
     &=& (1-\eta\epsilon)\hat{\vec{w}_1} - \eta \dydx{L}{\hat{\vec{w}}} \\
\end{eqnarray}

Where we have shortened the gradient of the loss evaluated at the current
parameter values to $\delta_{\vec{w}}$ and $\delta_{\hat{\vec{w}}}$.
We can then compare the effect the new parameters would have on the next
minibatch by calculating $\F{DFT}^{-1} \{\hat{\vec{w}}_2 \}$. Using
equations~\ref{eq:ch6:dft_grad} and ~\ref{eq:ch6:initial_condition} we then get:

\begin{eqnarray}
  \IFT{\hat{\vec{w}}_2} &=& \IFT{(1-\eta\epsilon)\hat{\vec{w}_1} - \eta\ \dydx{L}{\hat{\vec{w}}}} \\       
                        & = & (1-\eta\epsilon)\vec{w}_1 - \eta\ \IFT{ \dydx{L}{\hat{\vec{w}}}} \\
                        & = & (1-\eta\epsilon)\vec{w}_1 - \eta \dydx{L}{\vec{w}} \\
                        & = & \vec{w}_2
\end{eqnarray}

% \subsection{Regularization}
% If we use $\ell_1$ then the above doesn't hold.

% \subsection{Optimization}
% If we us adam things r different.
This does not hold for the Adam \cite{kingma_adam:_2014} or Adagrad \cite{}
optimizers, which automatically rescale the learning rates for each parameter
based on estimates of the parameter's variance. Rippel et.\ al.\ use this fact
in their paper \cite{rippel_spectral_2015}.

